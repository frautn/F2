\documentclass[11pt, spanish, a4paper]{article}

% Esto se incluye en cada documento de las guías de ejercicios de Python.

\usepackage[T1]{fontenc}
\usepackage[utf8]{inputenc}

\usepackage[spanish, es-tabla]{babel}
\def\spanishoptions{argentina} % Was macht dass?
% \usepackage{babelbib}
% \selectbiblanguage{spanish}
% \addto\shorthandsspanish{\spanishdeactivate{~<>}}

\usepackage{graphicx}
\graphicspath{{./figuras/}{../LaTeX/}{../figurasLaTeX/}}
% \usepackage{float}

\usepackage[arrowdel]{physics}
\newcommand{\pvec}[1]{\vec{#1}\mkern2mu\vphantom{#1}}
% \usepackage{units}
\usepackage[separate-uncertainty= true, multi-part-units= single, range-units= single, range-phrase= {~a~}, locale= FR]{siunitx}
\usepackage{isotope} % $\isotope[A][Z]{X}\to\isotope[A-4][Z-2]{Y}+\isotope[4][2]{\alpha}

\usepackage{tasks}
\usepackage[inline]{enumitem}
% \usepackage{enumerate}

\usepackage{hyperref}

% \usepackage{amsmath}
% \usepackage{amstext}
% \usepackage{amssymb}

\usepackage{tikz}
\usepackage{tikz-3dplot}
\usepackage{tikz-dimline}
\usetikzlibrary{calc}
% \usetikzlibrary{math}
\usetikzlibrary{arrows.meta}
\usetikzlibrary{snakes}
\usetikzlibrary{decorations}
\usetikzlibrary{decorations.pathmorphing}
\usetikzlibrary{patterns}

\usepackage[hmargin=1cm,vmargin=3cm, top= 0.75cm,nohead]{geometry}

\usepackage{lastpage}
\usepackage{fancyhdr}
\pagestyle{fancyplain}
\fancyhf{}
\setlength\headheight{28.7pt} 
\fancyhead[LE, LO]{\textbf{Física 2} }
% \fancyhead[RE, RO]{\href{https://ingenieria.unlam.edu.ar/}{$\vcenter{\hbox{\includegraphics[height=1cm]{../../../../figurasLaTeX/ambos.pdf}}}$}}  %edg fix path
% \fancyfoot{\href{https://creativecommons.org/licenses/by-nc-sa/4.0/deed.es_ES}{$\vcenter{\hbox{\includegraphics[height=0.4cm]{../../../../figurasLaTeX/by-nc-sa_80x15.pdf}}}$} \href{https://ingenieria.unlam.edu.ar/}{DIIT - UNLaM}}  %edg fix path
% \fancyfoot[C]{ {\tiny Actualizado el \today} }
\fancyfoot[RO, LE]{Pág. \thepage/\pageref{LastPage}}
\renewcommand{\headrulewidth}{0pt}
\renewcommand{\footrulewidth}{0pt}


\begin{document}
\begin{center}
  % \textsc{\large Mecánica general}\\
	\textsc{\large Position vector}
	% \textsc{\large Vectores posición y velocidad. Energía cinética}
\end{center}

If you are able to solve these problems on your own, then you can assume that you have the minimum knowledge about these topics. 

The problems marked with (*) have additional difficulties. Don't hesitate about seeking help from teachers and your classmates if you are not able to complete them.
			

\begin{enumerate}
% 	\section*{Vector posición}
	
	\item
		\begin{minipage}[t][6cm]{0.55\textwidth}
			\textbf{Addition of positions}\\
			\begin{enumerate}
				\item Save in a variable called \verb"a_r" a vector that indicates the position $\vec{r}_a = 3 \hat{e}_x + 0 \hat{e}_y + 5 \hat{e}_z$. 
				\item Save $\vec{r}_b = -5 \hat{e}_x + (-5) \hat{e}_y + 7 \hat{e}_z$ in \verb"b_r".
				\item Subtract the corresponding variables to find $\Delta \vec{r}_{a \to b} = \vec{r}_b - \vec{r}_a$ and save the result in \verb"ab_deltaR".
				\item Save in \verb"c_r" the result from $\vec{r}_a +\Delta \vec{r}_{a \to b}$.
				\item To verify that you did a good work, it's sufficient to display \verb"c_r" and check that $\vec{r}_c = \vec{r}_b$.
			\end{enumerate}
		\end{minipage}
		\begin{minipage}[c][1cm][t]{0.4\textwidth}
			% https://tex.stackexchange.com/questions/117140/easiest-way-to-draw-a-3d-coordinate-system-with-axis-labels-and-ticks-in-tikz
			\begin{tikzpicture}[x=0.5cm,y=0.5cm,z=0.3cm, scale=0.85]
				% The axes
				\draw[-LaTeX] (xyz cs:x=-8.5) -- (xyz cs:x=8.5) node[above] {$x$};
				\draw[-LaTeX] (xyz cs:y=-5.5) -- (xyz cs:y=8.5) node[right] {$z$};
				\draw[-LaTeX] (xyz cs:z=-8.5) -- (xyz cs:z=8.5) node[right] {$y$};
				% The thin ticks
				\foreach \cooZ in {-5,-4,...,8} {
					\draw (-1.5pt,\cooZ) -- (1.5pt,\cooZ);
				}
				\foreach \coo in {-8,-7,...,8}
				{
					\draw (\coo,-1.5pt) -- (\coo,1.5pt);
					\draw (xyz cs:y=-0.15pt,z=\coo) -- (xyz cs:y=0.15pt,z=\coo);
				}
				% The thick ticks
				\foreach \coo in {-5,5}
				{
					\draw[thick] (\coo,-3pt) -- (\coo,3pt) node[below=6pt] {\coo};
					\draw[thick] (-3pt,\coo) -- (3pt,\coo) node[left=6pt] {\coo};
					\draw[thick] (xyz cs:y=-0.3pt,z=\coo) -- (xyz cs:y=0.3pt,z=\coo) node[below=8pt] {\coo};
				}
				% base of versors
				\coordinate (O) at (0,0,0);
				\draw[thick,blue,-LaTeX] (O) -- (1,0,0) node[anchor=north]{$\hat{e}_x$};
				\draw[thick,blue,-LaTeX] (O) -- (0,0,1) node[anchor=west]{$\hat{e}_y$};
				\draw[thick,blue,-LaTeX] (O) -- (0,1,0) node[anchor=east]{$\hat{e}_z$};
				% Dashed lines for the points
				\draw[dashed]
				(xyz cs:z=-5) --
				+(0,7) coordinate (u) --
				(xyz cs:y=7) --
				+(-5,0) --
				++(xyz cs:x=-5,z=-5) coordinate (v) --
				+(0,-7) coordinate (w) --
				cycle;
				\draw[dashed] (u) -- (v);
				\draw[dashed] (-5,7) -- (-5,0) -- (w);
				\draw[dashed] (3,0) |- (0,5);
				% Dots and labels for
				\coordinate (a) at (3,5);
				\coordinate (b) at (v);
				\node[fill,circle,inner sep=1.5pt,label={above right:$(3,0,5)$}] at (a) {};
				\node[fill,circle,inner sep=1.5pt,label={above right:$(-5,-5,7)$}] at (b) {};
				\draw[red,-LaTeX] (O) -- (a) node [midway, above, left]{$\vec{r}_a$};
				\draw[red,-LaTeX] (O) -- (b) node [midway, below]{$\vec{r}_b$};
				% % The origin
				% \node[align=center] at (3,-3) (ori) {(0,0,0)\\\text{origen}};
				% \draw[-LaTeX,help lines] (ori) .. controls (1,-2) and (1.2,-1.5) .. (0,0,0);
			\end{tikzpicture}
		\end{minipage}


	\item
		\begin{minipage}[t][3.5cm]{0.7\textwidth}
			(*) \textbf{Position as a function of a variable}\\
			A particle of mass \(m\) is attached to a ring of radius $R$, and therefore its radius measured from the center of the ring is constant. Then it's enough to know the angle $\varphi$ to describe its position. 
			\begin{enumerate}
				\item Write it using cartesian coordinates.
				\item Calculate the velocity of this particle.
			\end{enumerate}
		\end{minipage}
		\begin{minipage}[c][1.5cm][t]{0.15\textwidth}
			\begin{tikzpicture}[x=0.5cm,y=0.5cm,z=0.3cm, scale=1]
				% The axes
				\draw [-LaTeX] (0,0) -- (5,0) node[above] {$x$};
				\draw [-LaTeX] (0,0) -- (0,5) node[right] {$y$};
				\draw [very thick] (0,0) circle [radius= 4];
				\draw [-LaTeX, dashed] (1,0) arc [radius=1, start angle=0, end angle=300] node[midway,left] {$\varphi$};
				\draw [dashed] (0,0) -- ($(0,0)+(-60:4)$) node [midway,right] {$R$}; 
				\shade [ball color=black!80] ($(0,0)+(-60:4)$) circle(0.5) node [] {\color{white} $m$};
			\end{tikzpicture}
		\end{minipage}


%
%	\section*{Energía cinética}
%
%		\item
%		\begin{minipage}[t][5cm]{0.6\textwidth}
%			\textbf{Péndulo con punto de suspensión libre} [Landau \S5 ej. 2]\\
%			La partícula de masa \(m_2\) pende de una barra rígida de longitud \(\ell\) de masa despreciable.
%			El otro extremo de la misma está engarzada a una barra rígida dispuesta a lo largo del eje \(\hat{x}\).
%			El dispositivo de engarze tiene una  masa \(m_1\).\\
%			Exprese la energía cinética del sistema en función de las coordenadas indicadas por la figura: \(x, \phi\).
%		\end{minipage}
%		\begin{minipage}[c][1cm][t]{0.35\textwidth}
%			\includegraphics[width=\textwidth]{landauS52_fig2.png}
%		\end{minipage}
%
%
%		\item
%			\begin{minipage}[t][3cm]{0.7\textwidth}
%				\textbf{Péndulo doble} [Landau \S5 ej. 1]\\
%				Una barra rígida de longitud \(\ell_1\) tiene una masa despreciable respecto a la de la partícula de masa \(m_1\) fija a su extremo.
%				A su vez de esta última pende otra barra rígida, de longitud \(\ell_2\) que en su extremo tiene otra partícula de masa \(m_2\), también mucho mayor que aquella de la barra.\\
%				Exprese la energía cinética de este sistema en función de las coordenadas indicadas en la figura: \(\phi_1, \phi_2\).\\
%			\end{minipage}
%			\begin{minipage}[c][0.5cm][t]{0.3\textwidth}
%				\includegraphics[width=0.75\textwidth]{landauS52_fig1.png}
%			\end{minipage}
%

\end{enumerate}
\end{document}

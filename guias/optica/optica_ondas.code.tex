\begin{center}
   {\scshape \Huge  Óptica\par}
   \vspace{0.5 cm}
\end{center}

% \ifthenelse{\equal{\unidaddos}{false}}{
% \part{\scshape Óptica}
% \vspace{0.5 cm}
% }

\section{Ondas}
\rfigure
%
\pma{\label{p:P170}
\textit{a}) Ondas de radio de onda media y de frecuencia modulada tienen frecuencias del orden de los $\SI{1200}{kHz}$ y de 120 kHz, respectivamente. Hallar sus
correspondientes longitudes de onda en el vacío. \textit{b}) El oído humano es capaz de percibir sonidos de frecuencias comprendidas entre $\SI{20}{Hz}$ y $\SI{20000}{Hz}$. Determinar las longitudes de onda correspondientes a estas frecuencias, suponiendo que la velocidad del sonido es $\SI{338}{m/s}$. \textit{c}) La radiación de los hornos a microondas o la de las señales Wi-Fi tienen una frecuencia cercana a los $\SI{2.45}{GHz}$. Calcular la longitud de onda de estas radiaciones.\\
\rta{.95}{\textit{a}) Entre $\SI{250}{\metre}$ y $\SI{2500}{\metre}$; \textit{a}) Entre $\SI{0.0169}{\metre}$ y $\SI{16.9}{\metre}$; \textit{c}) $\SI{0.122}{m}$}
}
%
\pma{
Un hombre que se sienta en el borde de un muelle a pescar y cuenta las ondas de agua que golpean un poste de soporte del muelle, en un minuto cuenta 76 ondas. Si una
cresta en particular viaja $\SI{18}{m}$ en $\SI{6}{s}$, ¿cuál es la longitud de onda de estas ondas?\\
\rta{.95}{$\SI{2.37}{m}$}
}
%
\pma{
Una onda armónica en un hilo tiene una amplitud de $\SI{17}{mm}$, una longitud de onda de $\SI{1.96}{m}$ y una velocidad de $\SI{4.1}{m/s}$. Determinar el período, la frecuencia, la frecuencia angular y el número de onda.\\
\rta{.95}{$f = \SI{2.09}{\hertz}$; $T = \SI{0.48}{\second}$; $\omega = \SI{13.13}{\radian/\second}$; $k = \SI{3.2}{\metre^{-1}}$}
}
%
\pma{
La velocidad de las ondas electromagnéticas en el vacío es $\SI{3E8}{m/s}$. Las longitudes de onda de las ondas electromagnéticas visibles se extienden aproximadamente desde $\SI{400}{nm}$ (luz violeta) hasta $\SI{750}{nm}$ (luz roja). Determinar el rango de frecuencias de luz visible.\\
\rta{.95}{Entre $\SI{4.0E14}{\hertz}$ y $\SI{7.5E14}{\hertz}$}
}
%
\pma{
La intensidad promedio de la luz solar sobre la superficie terrestre es de unos $\SI{700}{W/m^2}$. \textit{a}) Calcular la cantidad de energía que incide sobre un panel solar de $\SI{0.5}{m^2}$ de área en 4 horas. \textit{b}) ¿Qué intensidad tendrá la luz solar, si es concentrada por una lupa sobre una superficie 200 veces menor que la propia de la lupa?\\
\rta{.95}{\textit{a}) $\SI{5.04E6}{J}$; \textit{b}) $\SI{1.4E5}{W/m^2}$}
}


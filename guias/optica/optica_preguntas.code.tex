\section{Preguntas sobre óptica para el análisis}
\rfigure
\textit{En esta sección se requiere que se brinden respuestas argumentadas.}

\pma{
Se realiza un experimento de interferencia con dos ranuras, y las franjas se proyectan en una pantalla. Después, todo el aparato se sumerge agua, ¿cómo cambia el patrón de las franjas?
}
%
\pma{
A través de dos ranuras delgadas pasa luz monocromática coherente que se ve en una pantalla alejada. ¿Las franjas brillantes en la pantalla se encontrarán igualmente separadas? Si es así, ¿por qué? Si no, ¿cuáles están más cerca de estar igualmente separadas?
}
%
\pma{
En un patrón de interferencia de dos ranuras sobre una pantalla distante, ¿las franjas brillantes están a la mitad de la distancia que hay entre las franjas oscuras?
}
%
\pma{
Las luces de un automóvil distante, ¿formarían un patrón de interferencia de dos fuentes?
}
%
\pma{
Se iluminan con luz coherente de longitud de onda $\lambda$ dos ranuras estrechas separadas por una distancia $d$. Si $d$ es menor que cierto valor mínimo, no se observan franjas oscuras. Explique lo que sucede. En términos de $\lambda$, indique cuál es este valor mínimo de $d$.
}
%
\pma{
¿Por qué podemos observar fácilmente los efectos de la difracción en el caso de las ondas sonoras y de las ondas en el agua, pero no en el caso de la luz?
}
%
\pma{
A través de una sola ranura de ancho $a$ pasa luz de longitud de onda $\lambda$ y frecuencia $f$. Se observa el patrón de difracción en una pantalla a una distancia $S$ de la ranura. De las acciones siguientes, ¿cuáles reducen la anchura del máximo central? \textit{a}) Disminuir el ancho $a$ de la ranura. \textit{b}) Disminuir la frecuencia $f$ de la luz. \textit{c}) Disminuir la longitud de onda $\lambda$ de la luz. \textit{d}) Disminuir la distancia $S$ de la ranura a la pantalla.
}
%
\pma{
En un experimento de difracción que utiliza ondas con longitud de onda $\lambda$, no habrá mínimos de intensidad (es decir, no habrá franjas oscuras) si la anchura de la rendija es lo suficientemente pequeña. ¿Cuál es el ancho máximo de rendija con el cual ocurre esto?
}
%
% \pma{
% Cuando la luz no polarizada incide en dos polarizadores cruzados, no se transmite luz. Un estudiante afirmó que si se insertaba un tercer polarizador entre los otros dos, habría algo de transmisión. ¿Tiene sentido esto? ¿Cómo podría un tercer filtro incrementar la transmisión?
% }
%

\twocolumn[\colorsection{Preguntas sobre electrostática para el análisis}]
\textit{En esta sección se requiere brindar respuestas argumentadas.}
\setcounter{figure}{0}
%
\begin{Exercise}
    Imagine que tiene dos esferas metálicas ligeras y que cada una de ellas cuelga de un cordón de nailon aislante. Una de las esferas tiene carga neta negativa; en tanto que la otra no tiene carga neta. Si las esferas están cerca una de otra, pero no se tocan, ¿se atraerán mutuamente, se repelerán o no ejercerán fuerza alguna sobre la otra?
\end{Exercise}
%
\begin{Exercise}
Dado el caso anterior: ahora se permite que las esferas entren en contacto. Una vez que se tocan, ¿se atraerán, se repelerán o no ejercerán fuerza alguna sobre la otra?
\end{Exercise}
%
\begin{Exercise}
    Cuando una varilla de vidrio cargada se acerca a un trozo de papel descargado, el papel se siente atraído por la varilla. Explique porqué sucede esto aunque el papel sea un material no conductor y esté descargado.
\end{Exercise}
%
\begin{Exercise}
    Suponga que una carga puntual se encuentra fija en su posición. Entonces, si una partícula pequeña cargada positivamente se coloca en algún punto cercano a la carga fija y se libera, ¿la trayectoria de esta segunda partícula seguirá una línea de campo eléctrico? ¿Por qué?
\end{Exercise}
%
\begin{Exercise}
    Se coloca un protón en un campo eléctrico uniforme y luego se libera. Después se sitúa un electrón en el mismo punto y también se libera. ¿Las dos partículas experimentan la misma fuerza? ¿La misma aceleración? ¿Se mueven en la misma dirección cuando se liberan?
\end{Exercise}
%
\begin{Exercise}
    El potencial (en relación con un punto en el infinito) a media distancia entre dos cargas puntuales de igual magnitud y signos opuestos es igual a cero. Si se trae una carga de prueba desde el infinito hasta ese punto medio, el trabajo neto que es necesario entregar a esa carga, ¿es positivo, negativo o cero? ¿Es posible traer una carga de prueba desde el infinito hasta ese punto medio en forma tal que no se efectúe trabajo en ninguna parte del desplazamiento? Si es así, describa cómo se puede lograr. Si no es posible, explique por qué.
\end{Exercise}
%
\begin{Exercise}
    Es posible tener una configuración de dos cargas puntuales separadas por una distancia finita de manera que la energía potencial eléctrica del arreglo sea la misma que la de las dos cargas separadas por una distancia infinita? ¿Por qué? ¿Y si fueran tres cargas? Explique su razonamiento.
\end{Exercise}
%
\begin{Exercise}
    Como el potencial puede tener cualquier valor que se desee dependiendo de la elección del nivel de referencia de potencial cero, ¿cómo “sabe” un voltímetro qué lectura hacer cuando se conecta entre dos puntos?
\end{Exercise}
%
\begin{Exercise}
    Si se resuelve la integral del campo eléctricostático para una trayectoria cerrada, el resultado siempre será igual a cero, independientemente de la forma de la trayectoria y de dónde se localicen las cargas en relación con esta. Explique por qué es así.
\end{Exercise}
%
\begin{Exercise}
	Un cable coaxial largo consiste en un conductor cilíndrico macizo central y un cilindro hueco que rodea al hilo central, con radio interior $a$ y radio exterior $b$. El cilindro exterior está montado en apoyos aislantes y no tiene carga neta, mientras que el cilindro central tiene una carga uniforme por unidad de longitud $\lambda$. Determinar la carga por unidad de longitud en las superficies interna y externa del cilindro exterior.
\end{Exercise}
%
\begin{Exercise}
	Una esfera hueca, conductora, con radio interior $a$ y radio exterior $b$, tiene una carga neta igual a $\SI{6}{\micro\coulomb}$. Responda las siguientes preguntas utilizando la ley de Gauss: \textit{a}) ¿Cuál es el valor de la carga neta distribuida sobre la superficie interior, de radio $a$? \textit{b}) ¿Cuál es el valor de la carga neta distribuida sobre la superficie exterior, de radio $b$? \textit{c}) Si se introduce una carga de $\SI{-2}{\micro\coulomb}$ en la cavidad interna de la esfera, ¿cuál es el nuevo valor de la carga distribuida sobre la superficie externa de la esfera?
\end{Exercise}

\twocolumn[\colorsection{Preguntas sobre magnetostática para el análisis}]
\textit{En esta sección se requiere brindar respuestas argumentadas.}
\setcounter{figure}{0}
%
\begin{Exercise}
    Explique si las siguientes afirmaciones son verdaderas o falsas:
    \begin{enumerate}[a)]
        \item La fuerza magnética que actúa sobre una partícula cargada móvil es siempre perpendicular a la velocidad de la partícula.
        \item El momento del par que actúa sobre un imán tiende a alinear el momento magnético en la dirección del campo magnético.
        \item Una espira de corriente en un campo magnético uniforme se comporta como un pequeño imán.
        \item El período de una partícula moviéndosa en círculo en un campo magnético es proporcional al radio del círculo.
        \item Los imanes existen en la naturaleza y presentan dos o más polos.
        \item Al igual que los cuerpos electrizados, los polos de distinto tipo de atraen y los de igual tipo se repelen.
    \end{enumerate}
\end{Exercise}
%
\begin{Exercise}
    Contestar cuál opción es la correcta. Al hacer dos cortes en un imán de barra con dos polos en los extremos, dividiéndolo en tres partes iguales, como se muestra en la figura, se obtienen:
    \begin{enumerate}[a)]
        \item Tres imanes completos (cada uno con sus polos Norte y Sur).
        \item Uno con un polo Norte, uno con un polo Sur, y un fragmento no magnetizado.
        \item Dos imanes completos y un fragmento no magnetizado.
        \item Dos fragmentos no magnetizados y un imán completo.
        \item Tres fragmentos no magnetizados.
    \end{enumerate}
FIGURA
\end{Exercise}
%
\begin{Exercise}
    ¿Podría una partícula cargada moverse a través de un campo magnético sin experimentar fuerza alguna?
\end{Exercise}
%
\begin{Exercise}
    La fuerza magnética sobre una partícula cargada en movimiento siempre es perpendicular al campo magnético. ¿La trayectoria siempre es perpendicular a las líneas de campo magnético?
\end{Exercise}
%
\begin{Exercise}
    Una partícula cargada se mueve a través de una región del espacio con velocidad constante. Si el campo magnético externo es igual a cero en esta región, ¿se puede concluir que el campo eléctrico externo también vale cero? (Con ``externo'' nos referimos a aquellos campos que no son producidos por la partícula cargada.) Si el campo eléctrico externo es cero en la región, ¿se puede concluir que el campo magnético externo también es cero?
\end{Exercise}
%
\begin{Exercise}

\end{Exercise}

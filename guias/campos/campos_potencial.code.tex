\twocolumn[\colorsection{Potencial eléctrico}]
\setcounter{figure}{0}
%
\begin{Exercise}
  ¿Cuál es la energía necesaria para ubicar cuatro cargas de $\SI{3.0}{\micro\coulomb}$ en las esquinas de un cuadrado cuyos lados miden $\SI{7.5}{\centi\metre}$?
\end{Exercise}
\begin{Answer}
  $\SI{5.85}{\joule}$
\end{Answer}
%
\begin{Exercise}
  Una carga puntual de $\SI{4.0}{\nano\coulomb}$ está situada en el origen, y otra carga puntual de $\SI{-3.0}{\nano\coulomb}$ está sobre el eje $x$ en la posición $x = \SI{0.20}{\metre}$. ¿Dónde debe situarse sobre el eje $x$, una tercera carga de $\SI{2.0}{\nano\coulomb}$, para que la energía potencial del sistema formado por las tres cargas sea igual a cero?
\end{Exercise}
\begin{Answer}
	\begin{minipage}[t]{.4\textwidth}
    $x = \SI{-0.10}{\metre}$ o $x = \SI{0.074}{\metre}$
  \end{minipage}
\end{Answer}
%
\begin{Exercise}
  Una carga puntual $q_1 = \SI{2.40}{\nano\coulomb}$ se mantiene estacionaria en el origen. Una segunda carga puntual $q_2 = \SI{-4.30}{\nano\coulomb}$ se desplaza desde la posición $\va*{r}_0 =\SI{0.150}{\metre}\vu{x} + \SI{0}{\metre}\vu{y}$ hasta la posición $\va*{r}_f =\SI{0.250}{\metre}\vu{x} + \SI{0.250}{\metre}\vu{y}$. ¿Cuánto varió la energía potencial de la carga $q_2$?
\end{Exercise}
\begin{Answer}
  $\SI{3.57E-7}{\joule}$
\end{Answer}
%
\begin{Exercise}
  Una esfera metálica pequeña tiene una carga neta $q_1 = \SI{2.80}{\micro\coulomb}$ y se mantiene en posición fija por medio de soportes aislantes. Una segunda esfera metálica pequeña con carga neta $q_2 = \SI{7.80}{\micro\coulomb}$ y una masa de $\SI{1.50}{\gram}$ es proyectada hacia $q_1$. Cuando las dos esferas están a una distancia de $\SI{0.800}{\metre}$ una de otra, $q_2$ se mueve hacia $q_1$ con una rapidez de $\SI{22.0}{\metre/\second}$. Suponga que las dos esferas pueden considerarse como cargas puntuales, ¿qué tan cerca de $q_1$ llega $q_2$?
\end{Exercise}
\begin{Answer}
  $\SI{0.323}{\metre}$
\end{Answer}
%
\begin{Exercise}
  \textbf{electronvolt:} Un electronvolt ($\SI{1}{eV}$) es una unidad de energía que equivale a la variación de energía potencial de un electrón que se desplaza a través de una diferencia de potencial de $\SI{1}{\volt}$. \textit{a}) Verifique la siguiente equivalencia: \[\SI{1}{eV} = \SI{1.602E-19}{\joule}~.\] \textit{b}) Calcule la energía (en eV y en J) de un electrón que ha sido acelerado desde el reposo, a través de una diferencia de potencial de $\SI{100}{\volt}$. \textit{c}) Calcule la velocidad que alcanza ese electrón.
\end{Exercise}
\begin{Answer}
	\begin{minipage}[t]{.4\textwidth}
    \textit{b}) $\SI{100}{eV} = \SI{1.602E-17}{\joule}$\\ \textit{c}) $\SI{5.93E6}{\metre/\second}$
  \end{minipage}
\end{Answer}
%
\begin{Exercise}
  Calcule el potencial eléctrico en el centro de un cuadrado de $\SI{1}{\metre}$ de lado, si en sus vértices se ubican las siguientes cargas: $q_1 = \SI{10}{\nano\coulomb}$; $q_2 = \SI{-20}{\nano\coulomb}$; $q_3 = \SI{30}{\nano\coulomb}$ y $q_4 = \SI{20}{\nano\coulomb}$.
\end{Exercise}
\begin{Answer}
  $\SI{509}{\volt}$
\end{Answer}
%
\begin{Exercise}\label{p:potencial01}
  Para la distribución de cargas mostrada en la figura \ref{f:potencial01}, donde $q_1 = \SI{3.1}{\micro\coulomb}$ y $q_2 = \SI{2.4}{\micro\coulomb}$ están sobre el plano $xy$, calcule: \textit{a}) el potencial eléctrico en el origen de coordenadas, \textit{b}) el potencial eléctrico en la posición $\va*{r} =\SI{0.25}{\metre}\vu{z}$.
\end{Exercise}
\begin{Answer}
	\begin{minipage}[t]{.4\textwidth}
    \textit{a}) $\SI{1.98E5}{\volt}$\\ \textit{b}) $\SI{1.4E5}{\volt}$
  \end{minipage}
\end{Answer}
%
\begin{center}
  \begin{tikzpicture}[scale=0.5]
    %Axis
    \draw[axis] (-1,0) -- (5,0) node [below, pos=1.1] {$x$};
    \draw[axis] (0,-1) -- (0,5) node [left, pos=1.05] {$y$};
    \fill [black](2.5,0) circle(5pt) node[above] {$q_1$} node[below] {$0.25\text{ m}$};
    \fill [black](0,2.5) circle(5pt) node[right] {$q_2$} node[left] {$0.25\text{ m}$};
  \end{tikzpicture}
  \captionof{figure}{Problema \ref{p:potencial01}\label{f:potencial01}}
\end{center}
%
\begin{Exercise}\label{p:potencial02}
  Un dipolo de cargas $\pm q$ y separación $d$ ($p=qd$) está colocado sobre el eje $\vu{i}$ como se muestra en la figura \ref{f:potencial02}. \textit{a}) Verifique que el potencial en el punto $P$ es:
  \begin{flalign*}
    V_P &= \dfrac{1}{4\pi\varepsilon_o} \dfrac{p}{x^2-\dfrac{d^2}{4}}
  \end{flalign*}
  \textit{b}) A partir de la expresión del ítem \textit{a}, verifique que el trabajo necesario para transportar una carga $Q$ muy distante hasta un punto situado sobre el eje $\vu{i}$, a una distancia $a$ del centro del dipolo es:
  \begin{flalign*}
    W = QV_a &= \dfrac{Q}{4\pi\varepsilon_o} \dfrac{p}{a^2-\dfrac{d^2}{4}}
  \end{flalign*}
  \textit{c}) Verifique que el potencial en $P$ cuando $x \gg d$ puede ser aproximado por:
  \begin{flalign*}
    V_P &= \dfrac{1}{4\pi\varepsilon_o} \dfrac{p}{x^2}
  \end{flalign*}
  \textit{d}) A partir del resultado anterior y usando $\va*{E} = -\nabla V$, obtenga la siguiente expresión para el campo eléctrico en el punto $P$ cuando $x \gg d$:
  \begin{flalign*}
    \va*{E}_P &= \dfrac{1}{2\pi\varepsilon_o} \dfrac{p}{x^3}\vu{i}
  \end{flalign*}
\end{Exercise}
%
\begin{center}
  \begin{tikzpicture}[scale=0.5]
    %Axis
    \draw[axis] (-5,0) -- (7 ,0) node [below, pos=1] {$i$};
    \draw[axis] (0,-0.5) -- (0,2) node [left, pos=1] {$j$};
    \fill [black](-2.5,0) circle(5pt) node[above] {$-q$};
    \fill [black](2.5,0) circle(5pt) node[above] {$q$};
    \fill [red](4.5,0) circle(5pt) node[above] {$P$};
    \draw [{latex}-{latex}] (0,-1) -- (4.5, -1) node [midway, below] {$x$};
  \end{tikzpicture}
  \captionof{figure}{Problema \ref{p:potencial02}\label{f:potencial02}}
\end{center}
%

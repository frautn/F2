\newcommand{\anio}{2024}
\newcommand{\comision}{2\textsuperscript{do} 31}

% Imprimir lista de ejercicios seleccionados para hacer en clase.
\newcommand{\seleccionados}{false}


\renewcommand{\thefigure}{\arabic{section}.\arabic{figure}}
\renewcommand{\theequation}{\arabic{section}.\arabic{equation}}

\renewcommand\spanishtablename{Tabla}


%---------------------------
%
% Section headers with colors.
%
%---------------------------
% \usepackage{stackengine}
% \usepackage[most]{tcolorbox}

\definecolor{topcolor}{RGB}{0,121,138}


% \newcommand{\colorofsection}{cyan!90!white}
\newcommand{\colorofsection}{topcolor}
\newcommand{\colorsection}[1]{
    \tcbset{on line,
        boxsep=5pt, left=0pt,right=0pt,top=0pt,bottom=0pt,
        colframe=white, colback=\colorofsection, sharp corners=southwest,
        leftrule=0pt, highlight math style={enhanced}
        }

    \refstepcounter{section}%
    \bigskip\bigskip
    {\noindent\def\stackalignment{l}%
    \stackunder[-2pt]{\tcbox{\textcolor{white}{\textbf{\Large\thesection.\hspace{5pt}#1}}}}{\textcolor{\colorofsection}{\rule{\linewidth}{1pt}}}\medskip}
    \addcontentsline{toc}{section}{\thesection\hspace{5pt}#1}
}

\newcommand{\colorsectionnonumber}[1]{
    \tcbset{on line,
        boxsep=5pt, left=0pt,right=0pt,top=0pt,bottom=0pt,
        colframe=white, colback=\colorofsection, sharp corners=southwest,
        leftrule=0pt, highlight math style={enhanced}
        }

    \bigskip\bigskip
    {\noindent\def\stackalignment{l}%
    \stackunder[-2pt]{\tcbox{\textcolor{white}{\textbf{\Large #1}}}}{\textcolor{\colorofsection}{\rule{\linewidth}{1pt}}}\medskip}
    \addcontentsline{toc}{section}{#1}
}

%---------------------------
%
% Exercises and answers.
%
%---------------------------
% \usepackage[lastexercise,answerdelayed]{exercise}
% \usepackage{multicol}
\counterwithin{Exercise}{section}
\renewcommand{\ExerciseHeader}{
    \noindent\textbf{\large
    \ExerciseHeaderNB\ExerciseHeaderTitle
    \ExerciseHeaderOrigin}}
\renewcommand{\AnswerHeader}{\noindent\medskip{\textbf{\ExerciseHeaderNB\hspace{5pt}}}}

%---------------------------
%
% Tikz
%
%---------------------------

\tikzset{axis/.style={blue, thick,-latex}}

% Optica:
% \newcommand\planemirror{} % just for safety
\def\planemirror[#1](#2)(#3)(#4)(#5){%
  % Synopsis
  % \planemirror[fill options](center)(length)(angle)(thickness)
  \fill[#1]   (#2) + ({-0.5*#3*cos(#4)},{-0.5*#3*sin(#4)})
  --++ ({0.5*#3*cos(#4)},{0.5*#3*sin(#4)})
  --++ ({#5*sin(#4)},{-#5*cos(#4)}) --++ ({-#3*cos(#4)},{-#3*sin(#4)}) -- cycle;
  \draw (#2) + ({-#3*cos(#4)/2},{-#3*sin(#4)/2}) --++ ({0.5*#3*cos(#4)},{0.5*#3*sin(#4)});
}
% \newcommand\ray{} % just for safety
\def\ray(#1)(#2)(#3)(#4){%
  % Synopsis
  % \ray[](starting point)(end point)(position of arrow)(xscale)
  \draw[red, thick] (#1) -- (#2) node[color=red, currarrow, pos=#3, xscale=#4, sloped] {};
}

%---------------------------
%
% Babinet
%
%---------------------------

\newcommand{\object}[1]{%
	\begin{scope}[canvas is xz plane at y=1.2]
		\draw[line join=round, thick, fill=black!40] (#1,-1.2) rectangle (#1+0.1,1.2);
	\end{scope}
	%
	\begin{scope}[canvas is xy plane at z=1.2]
		\draw[line join=round, thick, fill=black!25](#1,-1.2) rectangle (#1+0.1,1.2);
	\end{scope}
	%
	\begin{scope}[canvas is yz plane at x=#1]
		\draw[line join=round, thick, fill=black] (-1.2,-1.2) rectangle (1.2,1.2);
		\draw[line join=round, thick, fill=white] (0,0.3) -- (0.3,-0.15) -- (-0.3,-0.15) -- cycle;
		\draw[line join=round, thick, fill=black!40] (0.04,0.24) -- (-0.1667,-0.07) -- (0.2467,-0.07) -- (0.3,-0.15) -- (-0.3,-0.15) -- (0,0.3) -- cycle;
		\draw[thick] (-0.3,-0.15) -- (-0.1667, -0.07);
	\end{scope}
}
\newcommand{\objecttriangle}[1]{%
	\begin{scope}[canvas is yz plane at x=#1]
		\draw[line join=round, thick, fill=black] (0,0.3) -- (0.3,-0.15) -- (-0.3,-0.15) -- cycle;
		\draw[line join=round, thick, fill=black!40] (0,0.3) -- (0.1333, 0.38) -- (0.4333,-0.07)
		-- (0.3,-0.15) -- cycle;
	\end{scope}
}
\newcommand{\image}[1]{%
	\def\point{0.3}
	\def\inside{0.15}
	\begin{scope}[canvas is yz plane at x=#1]
		\draw[line join=round, thick, fill=black] (-1.2,-1.2) rectangle (1.2,1.2);
		\draw[line join=round, thick, fill=red] (0,\point) -- (\inside*0.5,\inside*0.707) -- (\point*0.707,\point*0.5) -- (\inside,0) -- (\point*0.707, -\point*0.5) -- (\inside*0.5, -\inside*0.707) -- (0,-\point) -- (-\inside*0.5, -\inside*0.707) -- (-\point*0.707, -\point*0.5) -- (-\inside,0) -- (-\point*0.707,\point*0.5) -- (-\inside*0.5,\inside*0.707) -- cycle;
	\end{scope}
}
\newcommand{\lens}[1]{%
	\begin{scope}[canvas is yz plane at x=#1+0.1]
		\draw[cyan!30] (0,0) circle (30pt);
	\end{scope}
	\begin{scope}[canvas is yz plane at x=#1]
		\shade[left color=cyan!0,right color=cyan!30]
		(0,0) circle (30pt);
		\draw[cyan!30] (0,0) circle (30pt);
	\end{scope}
}
%---------------------------


%---------------------------
%
% Genera una lista dinámica con
% ejercicios seleccionados.
%
%---------------------------

\let\svaddtocontents\addtocontents
\makeatletter
\newcommand\defineList[1]{%
 \expandafter\def\csname add#1line\endcsname##1##2##3{\addtocontents {##1}{%
  \protect \csname #1line\endcsname {##2}{##3}}}
 \expandafter\def\csname write#1\endcsname{%
  \renewcommand\addtocontents[2]{\relax}%
  \setcounter{section}{0}\noindent%
  \expandafter\def\csname #1line\endcsname####1####2{\expandafter\csname####1\endcsname{####2}}%
  \@starttoc{#1}%
  \setcounter{section}{0}%
  \let\addtocontents\svaddtocontents%
 }%
 \csname add#1line\endcsname{#1}{begin}{itemize}%
 \AtEndDocument{\csname add#1line\endcsname{#1}{end}{itemize}}
}
\newcommand\addToList[2]{\csname add#1line\endcsname{#1}{item}{#2}}
\newcommand\printList[1]{\csname write#1\endcsname}
\makeatother

%---------------------------

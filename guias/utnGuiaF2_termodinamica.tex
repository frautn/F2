\documentclass[a4paper,11pt,dvipsnames,twocolumn]{article}
\usepackage[a4paper,left=2cm,right=2cm,top=2.5cm,bottom=2.5cm]{geometry}

\setlength{\columnseprule}{0.4pt}
\setlength{\columnsep}{20pt}
% \renewcommand{\columnseprulecolor}{\color{red}}
\usepackage[utf8]{inputenc}
\usepackage[spanish]{babel}
\usepackage{amsmath}
\usepackage{enumerate}
\usepackage{caption}
\usepackage{graphicx}
\usepackage{xcolor}

\usepackage[locale=FR, per-mode=fraction, separate-uncertainty=true]{siunitx}
\usepackage{physics}
\ExplSyntaxOn
\msg_redirect_name:nnn { siunitx } { physics-pkg } { none }
\ExplSyntaxOff

\usepackage{stackengine}  % Needed for colorsection.
\usepackage[most]{tcolorbox}  % Needed for colorsection.
\usepackage[lastexercise,answerdelayed]{exercise}
\usepackage{multicol}  % Respuestas en dos columnas.

\usepackage{circuitikz,pgfplots}  %pgfplots needed for axis environment.
\usepackage{tikz-3dplot}
\usetikzlibrary{babel}  % Solves the problems produced by changes to category codes made by some babel modules.
\usetikzlibrary{decorations, decorations.pathmorphing}
\usetikzlibrary{shapes.geometric}
\usetikzlibrary{arrows, arrows.meta}
\usetikzlibrary{calc}
\usetikzlibrary{patterns}
% \usetikzlibrary{3d}  % Babinet.

% Tikzset needed for the two loops graph in preguntas.
\usetikzlibrary{decorations.markings}
\tikzset{
    set arrow inside/.code={\pgfqkeys{/tikz/arrow inside}{#1}},
    set arrow inside={end/.initial=>, opt/.initial=},
    /pgf/decoration/Mark/.style={
        mark/.expanded=at position #1 with
        {
            \noexpand\arrow[\pgfkeysvalueof{/tikz/arrow inside/opt}]{\pgfkeysvalueof{/tikz/arrow inside/end}}
        }
    },
    arrow inside/.style 2 args={
        set arrow inside={#1},
        postaction={
            decorate,decoration={
                markings,Mark/.list={#2}
            }
        }
    },
}


\newcommand{\anio}{2025}
\newcommand{\comision}{2\textsuperscript{do} 31}

% Imprimir lista de ejercicios seleccionados para hacer en clase.
\newcommand{\seleccionados}{false}


\renewcommand{\thefigure}{\arabic{section}.\arabic{figure}}
\renewcommand{\theequation}{\arabic{section}.\arabic{equation}}

\renewcommand\spanishtablename{Tabla}


%---------------------------
%
% Section headers with colors.
%
%---------------------------
% \usepackage{stackengine}
% \usepackage[most]{tcolorbox}

\definecolor{topcolor}{RGB}{0,121,138}


% \newcommand{\colorofsection}{cyan!90!white}
\newcommand{\colorofsection}{topcolor}
\newcommand{\colorsection}[1]{
    \tcbset{on line,
        boxsep=5pt, left=0pt,right=0pt,top=0pt,bottom=0pt,
        colframe=white, colback=\colorofsection, sharp corners=southwest,
        leftrule=0pt, highlight math style={enhanced}
        }

    \refstepcounter{section}%
    \bigskip\bigskip
    {\noindent\def\stackalignment{l}%
    \stackunder[-2pt]{\tcbox{\textcolor{white}{\textbf{\Large\thesection.\hspace{5pt}#1}}}}{\textcolor{\colorofsection}{\rule{\linewidth}{1pt}}}\medskip}
    \addcontentsline{toc}{section}{\thesection\hspace{5pt}#1}
}

\newcommand{\colorsectionnonumber}[1]{
    \tcbset{on line,
        boxsep=5pt, left=0pt,right=0pt,top=0pt,bottom=0pt,
        colframe=white, colback=\colorofsection, sharp corners=southwest,
        leftrule=0pt, highlight math style={enhanced}
        }

    \bigskip\bigskip
    {\noindent\def\stackalignment{l}%
    \stackunder[-2pt]{\tcbox{\textcolor{white}{\textbf{\Large #1}}}}{\textcolor{\colorofsection}{\rule{\linewidth}{1pt}}}\medskip}
    \addcontentsline{toc}{section}{#1}
}

%---------------------------
%
% Exercises and answers.
%
%---------------------------
% \usepackage[lastexercise,answerdelayed]{exercise}
% \usepackage{multicol}
\counterwithin{Exercise}{section}
\renewcommand{\ExerciseHeader}{
    \noindent\textbf{\large
    \ExerciseHeaderNB\ExerciseHeaderTitle
    \ExerciseHeaderOrigin}}
\renewcommand{\AnswerHeader}{\noindent\medskip{\textbf{\ExerciseHeaderNB\hspace{5pt}}}}

%---------------------------
%
% Tikz
%
%---------------------------

\tikzset{axis/.style={blue, thick,-latex}}

% Optica:
% \newcommand\planemirror{} % just for safety
\def\planemirror[#1](#2)(#3)(#4)(#5){%
  % Synopsis
  % \planemirror[fill options](center)(length)(angle)(thickness)
  \fill[#1]   (#2) + ({-0.5*#3*cos(#4)},{-0.5*#3*sin(#4)})
  --++ ({0.5*#3*cos(#4)},{0.5*#3*sin(#4)})
  --++ ({#5*sin(#4)},{-#5*cos(#4)}) --++ ({-#3*cos(#4)},{-#3*sin(#4)}) -- cycle;
  \draw (#2) + ({-#3*cos(#4)/2},{-#3*sin(#4)/2}) --++ ({0.5*#3*cos(#4)},{0.5*#3*sin(#4)});
}
% \newcommand\ray{} % just for safety
\def\ray(#1)(#2)(#3)(#4){%
  % Synopsis
  % \ray[](starting point)(end point)(position of arrow)(xscale)
  \draw[red, thick] (#1) -- (#2) node[color=red, currarrow, pos=#3, xscale=#4, sloped] {};
}

%---------------------------
%
% Babinet
%
%---------------------------

\newcommand{\object}[1]{%
	\begin{scope}[canvas is xz plane at y=1.2]
		\draw[line join=round, thick, fill=black!40] (#1,-1.2) rectangle (#1+0.1,1.2);
	\end{scope}
	%
	\begin{scope}[canvas is xy plane at z=1.2]
		\draw[line join=round, thick, fill=black!25](#1,-1.2) rectangle (#1+0.1,1.2);
	\end{scope}
	%
	\begin{scope}[canvas is yz plane at x=#1]
		\draw[line join=round, thick, fill=black] (-1.2,-1.2) rectangle (1.2,1.2);
		\draw[line join=round, thick, fill=white] (0,0.3) -- (0.3,-0.15) -- (-0.3,-0.15) -- cycle;
		\draw[line join=round, thick, fill=black!40] (0.04,0.24) -- (-0.1667,-0.07) -- (0.2467,-0.07) -- (0.3,-0.15) -- (-0.3,-0.15) -- (0,0.3) -- cycle;
		\draw[thick] (-0.3,-0.15) -- (-0.1667, -0.07);
	\end{scope}
}
\newcommand{\objecttriangle}[1]{%
	\begin{scope}[canvas is yz plane at x=#1]
		\draw[line join=round, thick, fill=black] (0,0.3) -- (0.3,-0.15) -- (-0.3,-0.15) -- cycle;
		\draw[line join=round, thick, fill=black!40] (0,0.3) -- (0.1333, 0.38) -- (0.4333,-0.07)
		-- (0.3,-0.15) -- cycle;
	\end{scope}
}
\newcommand{\image}[1]{%
	\def\point{0.3}
	\def\inside{0.15}
	\begin{scope}[canvas is yz plane at x=#1]
		\draw[line join=round, thick, fill=black] (-1.2,-1.2) rectangle (1.2,1.2);
		\draw[line join=round, thick, fill=red] (0,\point) -- (\inside*0.5,\inside*0.707) -- (\point*0.707,\point*0.5) -- (\inside,0) -- (\point*0.707, -\point*0.5) -- (\inside*0.5, -\inside*0.707) -- (0,-\point) -- (-\inside*0.5, -\inside*0.707) -- (-\point*0.707, -\point*0.5) -- (-\inside,0) -- (-\point*0.707,\point*0.5) -- (-\inside*0.5,\inside*0.707) -- cycle;
	\end{scope}
}
\newcommand{\lens}[1]{%
	\begin{scope}[canvas is yz plane at x=#1+0.1]
		\draw[cyan!30] (0,0) circle (30pt);
	\end{scope}
	\begin{scope}[canvas is yz plane at x=#1]
		\shade[left color=cyan!0,right color=cyan!30]
		(0,0) circle (30pt);
		\draw[cyan!30] (0,0) circle (30pt);
	\end{scope}
}
%---------------------------


%---------------------------
%
% Genera una lista dinámica con
% ejercicios seleccionados.
%
%---------------------------

\let\svaddtocontents\addtocontents
\makeatletter
\newcommand\defineList[1]{%
 \expandafter\def\csname add#1line\endcsname##1##2##3{\addtocontents {##1}{%
  \protect \csname #1line\endcsname {##2}{##3}}}
 \expandafter\def\csname write#1\endcsname{%
  \renewcommand\addtocontents[2]{\relax}%
  \setcounter{section}{0}\noindent%
  \expandafter\def\csname #1line\endcsname####1####2{\expandafter\csname####1\endcsname{####2}}%
  \@starttoc{#1}%
  \setcounter{section}{0}%
  \let\addtocontents\svaddtocontents%
 }%
 \csname add#1line\endcsname{#1}{begin}{itemize}%
 \AtEndDocument{\csname add#1line\endcsname{#1}{end}{itemize}}
}
\newcommand\addToList[2]{\csname add#1line\endcsname{#1}{item}{#2}}
\newcommand\printList[1]{\csname write#1\endcsname}
\makeatother

%---------------------------


% \includeonly{termo_portada.code, termo_calorimetria.code,
%   termo_preguntas_calorimetria.code,
%   termo_transmision.code,
%   termo_preguntas_transmision.code, termo_primerppio.code,
%   termo_segundoppio.code, termo_preguntas_principios.code,
%   termo_adicionales.code}

\graphicspath{{./termodinamica/img/}}

\pgfplotsset{compat=1.18}

\setlength{\parindent}{0pt}

\begin{document}
% \addcontentsline{toc}{section}{Unnumbered Section}

  \begin{titlepage}
    % \begin{figure}[ht]
    \begin{center}
    \vspace{1.5cm}
    % Aquí se inserta el escudo o emblema:
    \begin{tikzpicture}[scale=1, every node/.style={scale=1}]

    \definecolor{topcolor}{RGB}{0,121,138};
    \definecolor{botcolor}{RGB}{0,96,154};

    \fill[color=topcolor] (1.8273,1.065) arc (30.235:149.765:2.115) -- cycle;
    \fill[color=botcolor] (1.8273,-1.065) arc (-30.235:-149.765:2.115) -- cycle;

    \draw[color=botcolor, thick] (-1.755,-0.625) -- (1.755, -0.625);
    \fill[color=botcolor] (-1.755,-0.0929) -- (1.755, -0.0929) -- (1.755, -0.5405) -- (-1.755,-0.5405) -- cycle;

    % \fill [black](0,-0.7) circle(2pt);

    % \draw (-1.7539,0.93) rectangle (-0.9592,0);
    \fill[] (-1.7539,0.93) arc (180:360:0.39735) -- (-1.1178,0.93) arc (360:180:0.23875) -- cycle;
    \fill[] (-1.7539,0) arc (180:0:0.39735) -- (-1.1178,0) arc (0:180:0.23875) -- cycle;
    \fill (-1.7539,0.3866) rectangle (-0.9592,0.5434);
    \fill (-1.43495,0.93) rectangle (-1.27815,0);

    \fill[color=topcolor] (-0.7795,0.3275) arc[start angle=180, end angle=360,x radius=0.40585, y radius=0.3275] -- (0.0322,0.8506) arc[start angle=0, end angle=180,x radius=0.08925, y radius=0.0794] -- (-0.1463,0.3275) arc[start angle=0, end angle=-180,x radius=0.22735, y radius=0.1604] -- (-0.603, 0.8506) arc[start angle=0, end angle=180,x radius=0.08925, y radius=0.0794] -- cycle;

    \fill[color=topcolor] (0.4003,0.0794) arc[start angle=180, end angle=360,x radius=0.08925, y radius=0.0794] -- (0.5788,0.7515) -- (0.7986,0.7515) arc[start angle=-90, end angle=90,x radius=0.0794, y radius=0.08925] -- (0.1805,0.93) arc[start angle=90, end angle=270,x radius=0.0794, y radius=0.08925] -- (0.4003, 0.7515) -- cycle;

    \fill[color=topcolor] (0.9455,0.0794) arc[start angle=180, end angle=360,x radius=0.08925, y radius=0.0794] -- (1.124,0.608) -- (1.57,0.0267) arc[start angle=-153.217, end angle=0,x radius=0.09411, y radius=0.07059] -- (1.755,0.8506) arc[start angle=0, end angle=180,x radius=0.08925, y radius=0.0794] -- (1.5765,0.3191) -- (1.124,0.9004) arc[start angle=26.783, end angle=180,x radius=0.09411, y radius=0.07059] -- cycle;

    \node [scale=0.8,color=white] at (-1.3822,-0.31) {{\fontfamily{cmss}\selectfont \textbf{A}}};
    \node [scale=0.8,color=white] at (-1.0751,-0.31) {{\fontfamily{cmss}\selectfont \textbf{V}}};
    \node [scale=0.8,color=white] at (-0.768,-0.31) {{\fontfamily{cmss}\selectfont \textbf{E}}};
    \node [scale=0.8,color=white] at (-0.4609,-0.31) {{\fontfamily{cmss}\selectfont \textbf{L}}};
    \node [scale=0.8,color=white] at (-0.1536,-0.31) {{\fontfamily{cmss}\selectfont \textbf{L}}};
    \node [scale=0.8,color=white] at (0.1536,-0.31) {{\fontfamily{cmss}\selectfont \textbf{A}}};
    \node [scale=0.8,color=white] at (0.4605,-0.31) {{\fontfamily{cmss}\selectfont \textbf{N}}};
    \node [scale=0.8,color=white] at (0.7676,-0.31) {{\fontfamily{cmss}\selectfont \textbf{E}}};
    \node [scale=0.8,color=white] at (1.0749,-0.31) {{\fontfamily{cmss}\selectfont \textbf{D}}};
    \node [scale=0.8,color=white] at (1.3822,-0.31) {{\fontfamily{cmss}\selectfont \textbf{A}}};

    \node [scale=1] at (0,-0.845) {{\fontfamily{cmss}\selectfont UDB Física}};

\end{tikzpicture}


    \end{center}
    % \end{figure}

    \begin{center}
        {\LARGE UNIVERSIDAD TECNOLÓGICA NACIONAL}\par\medskip
        \vspace*{0.25cm}
        {\LARGE Facultad Regional Avellaneda}\par\medskip
        \vspace*{1cm}
        {\Huge Física II - \comision}\par\medskip
        \vspace*{0.5cm}
        {\LARGE Guía de problemas de la unidad I}\par\bigskip
        \vspace*{1cm}
        {\Huge \bf \color[RGB]{0,121,138} Termodinámica\par\medskip}
    \end{center}

    \vspace{1cm}

    % \newpage
    % \pagenumbering{roman}
    \begin{center}
        \begin{minipage}[t]{.7\textwidth}
            \renewcommand*{\contentsname}{Contenidos}
            \tableofcontents
        \end{minipage}
        \vspace*{\fill}
    \end{center}
    \begin{center}
        Año \anio
    \end{center}

    \end{titlepage}

    \newpage
    \pagenumbering{arabic}


  \ifthenelse{\equal{\seleccionados}{true}}
    {%Imprime una lista de ejercicios seleccionados
        Ejercicios seleccionados para hacer en clase.\par\medskip

\defineList{xyz-calorimetria}
\defineList{xyz-transmision}
\defineList{xyz-primerppio}
\defineList{xyz-segundoppio}
\defineList{xyz-preguntas}

Calorimetria:

\printList{xyz-calorimetria}

Transmisión:

\printList{xyz-transmision}

Primer principio:

\printList{xyz-primerppio}

Segundo principio:

\printList{xyz-segundoppio}

Preguntas:

\printList{xyz-preguntas}

    }{}

  \twocolumn[\colorsection{Calorimetría}]
\setcounter{figure}{0}
% %
% \begin{Exercise}
%   \textit{a}) La temperatura corporal normal promedio medida en la boca es $\SI{310}{\kelvin}$. ¿Cuál es la temperatura en grados Celsius y en Fahrenheit? \textit{b}) Durante un ejercicio muy vigoroso, la temperatura del cuerpo puede elevarse hasta $\SI{40}{\celsius}$. ¿Cuál es la temperatura en Kelvin y en grados Fahrenheit? \textit{c}) La temperatura de la superficie del cuerpo es aproximadamente $\SI{7}{\celsius}$ más baja que la temperatura interna. Exprese esta diferencia de temperatura en Kelvin y en grados Fahrenheit. \textit{d}) La sangre almacenada a $\SI{4}{\celsius}$ dura aproximadamente 3 semanas, mientras que la sangre almacenada a $\SI{-160}{\celsius}$ tiene una duración de 5 años. Exprese ambas temperaturas en las escalas Fahrenheit y Kelvin.
% \end{Exercise}
% \begin{Answer}
% 	\begin{minipage}[t]{.4\textwidth}
%     \textit{a}) $\SI{36.9}{\celsius}$ y $\SI{98.4}{^\circ F}$\\ \textit{b}) $\SI{313}{\kelvin}$ y $\SI{104}{^\circ F}$\\ \textit{c}) $\SI{7}{\kelvin}$ y $\SI{13}{^\circ F}$\\ \textit{d}) $\SI{4}{\celsius}$: $\SI{277}{\kelvin}$ y $\SI{39.2}{^\circ F}$; $\SI{-160}{\celsius}$: $\SI{113}{\kelvin}$ y $\SI{-256}{^\circ F}$
%   \end{minipage}
% \end{Answer}
%
\begin{Exercise}
  \textit{a}) Calcule la única temperatura a la que los termómetros Fahrenheit y Celsius coinciden. \textit{b}) Calcule la única temperatura a la que los termómetros Fahrenheit y Kelvin coinciden.
\end{Exercise}
\begin{Answer}
	\begin{minipage}[t]{.4\textwidth}
    \textit{a}) $\SI{-40}{\celsius}=\SI{-40}{^\circ F}$\\ \textit{b}) $\SI{575}{^\circ F}=\SI{575}{\kelvin}$
  \end{minipage}
\end{Answer}
%
\begin{Exercise}
  \textit{a}) Un termómetro de gas a volumen constante tiene una presión de $\SI{1000}{\pascal}$ a $\SI{15}{\celsius}$. Si la presión se incrementa a $\SI{2000}{\pascal}$, ¿cuál es la temperatura en grados Celsius? \textit{b}) Un termómetro de gas registra una presión absoluta de $\SI{325}{mmHg}$, estando en contacto con agua en el punto triple. ¿Qué presión indicará en contacto con agua en su punto de ebullición normal?
\end{Exercise}
\begin{Answer}
	\begin{minipage}[t]{.4\textwidth}
    \textit{a}) $\SI{303}{\celsius}$\\ \textit{b}) $\SI{444}{mmHg}$
  \end{minipage}
\end{Answer}
%
\begin{Exercise}
  Usando un termómetro de gas de volumen constante, un experimentador determinó que la presión del gas cuando el termómetro se encuentra a la temperatura del punto triple del agua ($\SI{0.01}{\celsius}$) es $\SI{4.8E4}{\pascal}$; y en el  punto de ebullición normal del agua ($\SI{100}{\celsius}$) es $\SI{6.5E4}{\pascal}$. \textit{a}) Suponiendo que la presión varía linealmente con la temperatura, use estos datos para calcular la temperatura Celsius en la que la presión del gas sería cero (es decir, obtenga la temperatura Celsius del cero absoluto). \textit{b}) ¿El gas de este termómetro obedece con precisión la ecuación $T_2/T_1=p_2/p_1$? Si es así y la presión a $\SI{100}{\celsius}$ fuera $\SI{6.5E4}{\pascal}$, ¿qué presión habría medido el experimentador a $\SI{0.01}{\celsius}$?
\end{Exercise}
\begin{Answer}
	\begin{minipage}[t]{.4\textwidth}
    \textit{a}) $\SI{-282.4}{\celsius}$\\ \textit{b}) $\SI{4.6E4}{\pascal}$
  \end{minipage}
\end{Answer}
%
\begin{Exercise}\label{p:calorimetria01}
  Un técnico mide el calor específico de un líquido desconocido sumergiendo en él una resistencia eléctrica. La energía eléctrica se convierte en calor transferido al líquido durante $\SI{120}{\second}$ a una tasa constante de $\SI{65.0}{\watt}$. La masa del líquido es $\SI{0.780}{\kilogram}$ y su temperatura aumenta de $\SI{18.55}{\celsius}$ a $\SI{22.54}{\celsius}$. Calcule el calor específico promedio del líquido en este intervalo de temperatura. Suponga que la cantidad de calor que se transfiere al recipiente es despreciable y que no se transfiere calor al entorno.
\end{Exercise}
\begin{Answer}
  $\SI{2.51E3}{\joule.\kilogram^{-1}.\kelvin^{-1}}$
\end{Answer}
%
\begin{Exercise}\label{p:calorimetria02}
  \ifthenelse{\equal{\seleccionados}{true}}
    {\addToList{xyz-calorimetria}{\ExerciseHeaderNB}}{}
  En un experimento se suministra calor a una muestra sólida de $\SI{500}{g}$ a una tasa de $\SI{10.0}{\kilo\joule/\minute}$ mientras se registra su temperatura en función del tiempo. La gráfica de sus datos se muestra en la figura \ref{f:calorimetria02}. \textit{a}) Calcule el calor latente de fusión del sólido. \textit{b}) Determine los calores específicos de los estados sólido y líquido del material.
\end{Exercise}
\begin{Answer}
	\begin{minipage}[t]{.4\textwidth}
    \textit{a}) $ L_f = \SI{3.00E4}{\joule/\kilogram}$\\ \textit{b}) $c_\text{sólido}=\SI{1.33E3}{\joule.\kilogram^{-1}.\kelvin^{-1}}$ y $c_\text{líquido}=\SI{1.00E3}{\joule.\kilogram^{-1}.\kelvin^{-1}}$
  \end{minipage}
\end{Answer}
%
\begin{center}
  \begin{tikzpicture}[scale=0.8]
      \begin{axis}[
                   every major x tick/.append style={thick,blue},
                   clip=false,
                   grid=both,
                   minor x tick num=3,        %un minor tick es decir 0.5
                   minor y tick num=3,
                   xmin=0, xmax=4,           %min y max para los ejes, NO PARA EL DOMINIO
                   ymin=-10, ymax=50,
                   %axis y line=center,        %alinea el eje al centro de la figura
                   %axis x line=middle,        %sino pone 2 ejes x
                   xtick  align=center,
                   xlabel={tiempo~[min]},
                   ylabel={temperatura~[$^\circ\text{C}$]}
                  ];
      \addplot [color=red, very thick] [id=s,samples= 180, domain=0:1]  {-5+15*x};
      \addplot [color=red, very thick] [id=s,samples= 180, domain=1:2.5]  {10};
      \addplot [color=red, very thick] [id=s,samples= 180, domain=2.5:4]  {10+20*(x-2.5)};
      \end{axis}
    \end{tikzpicture}
    \captionof{figure}{Problema \ref{p:calorimetria02}\label{f:calorimetria02}}
  \end{center}
%
\begin{Exercise}
  \ifthenelse{\equal{\seleccionados}{true}}
    {\addToList{xyz-calorimetria}{\ExerciseHeaderNB}}{}
  Una pieza metálica de $\SI{6.00}{\kilogram}$ de cobre sólido a una temperatura inicial $T$ se coloca con $\SI{2.00}{\kilogram}$ de hielo que se encuentran inicialmente a $\SI{-20.0}{\celsius}$. El hielo está en un contenedor aislado de masa despreciable. Después de que se alcanza el equilibrio térmico, se observan $\SI{1.20}{\kilogram}$ de hielo y $\SI{0.80}{\kilogram}$ de agua líquida. ¿Cuál era la temperatura inicial de la pieza de cobre?
\end{Exercise}
\begin{Answer}
  $T=\SI{150}{\celsius}$
\end{Answer}
%
\begin{Exercise}
  \ifthenelse{\equal{\seleccionados}{true}}
  {\addToList{xyz-calorimetria}{\ExerciseHeaderNB}}{}
  Una olla de cobre con una masa de $\SI{0.500}{\kilogram}$ contiene $\SI{0.170}{\kilogram}$ de agua, y ambas están a una temperatura de $\SI{20.0}{\celsius}$. Un bloque de $\SI{0.250}{\kilogram}$ de hierro a $\SI{85.0}{\celsius}$ se deja caer en la olla. Encuentre la temperatura final del sistema, suponiendo que no hay pérdida de calor a los alrededores.
\end{Exercise}
\begin{Answer}
  $\SI{27.5}{\celsius}$
\end{Answer}
%
\begin{Exercise}
  En un recipiente adiabático de masa despreciable, $\SI{0.2}{\kilogram}$ de hielo a una temperatura inicial de $\SI{-40}{\celsius}$ se mezclan con una masa $m$ de agua que tiene una temperatura inicial de $\SI{80}{\celsius}$. Si la temperatura final del sistema es $\SI{20}{\celsius}$, ¿cuál es la masa $m$ del agua que estaba inicialmente a $\SI{80}{\celsius}$?
\end{Exercise}
\begin{Answer}
  $m=\SI{0.4}{\kilogram}$
\end{Answer}
%
\begin{Exercise}
  El calor específico molar de cierta sustancia varía con la temperatura según la siguiente ecuación empírica: $C = a + bT$, donde $a=\SI{29.5}{\joule.\mole^{-1}.\kelvin^{-1}}$ y $b=\SI{8.20E-3}{\joule.\mole^{-1}.\kelvin^{-2}}$. ¿Cuánto calor se necesita para modificar la temperatura de $\SI{3.00}{\mole}$ de la sustancia de $\SI{27.0}{\celsius}$ a $\SI{227}{\celsius}$? (Sugerencia: Integre la ecuación $dQ = nCdT$.)
\end{Exercise}
\begin{Answer}
  $Q=\SI{19700}{\joule}$
\end{Answer}
%
\begin{Exercise}
  Un calorímetro de cobre cuya masa es $\SI{0.446}{\kilogram}$ contiene $\SI{0.095}{\kilogram}$ de hielo, y el sistema está inicialmente en equilibrio a $\SI{0}{\celsius}$. Si se agregan $\SI{0.035}{\kilogram}$ de vapor de agua a $\SI{100.0}{\celsius}$ y $\SI{1}{atm}$ de presión, \textit{a}) ¿qué temperatura final alcanzará el calorímetro y su contenido?, \textit{b}) ¿cuántos kilogramos habrá de hielo, de agua líquida y de vapor a dicha temperatura final?
\end{Exercise}
\begin{Answer}
	\begin{minipage}[t]{.4\textwidth}
    \textit{a}) $\SI{86.2}{\celsius}$\\ \textit{b}) sin hielo, sin vapor, $\SI{0.130}{\kilogram}$ de agua en estado líquido.
  \end{minipage}
\end{Answer}
%
\begin{Exercise}
  Un calorímetro cuyo equivalente en agua es $\SI{20}{\gram}$, contiene $\SI{100}{\gram}$ de agua a $\SI{20}{\celsius}$. Se agregan $\SI{50}{\gram}$ de una sustancia desconocida a una temperatura de $\SI{90}{\celsius}$, obteniéndose una temperatura final de equilibrio de $\SI{24}{\celsius}$. Calcular el calor específico de la sustancia desconocida.
\end{Exercise}
\begin{Answer}
  $\SI{0.145}{cal.\gram^{-1}.\celsius^{-1}}$
\end{Answer}
%
\begin{Exercise}
  \ifthenelse{\equal{\seleccionados}{true}}
  {\addToList{xyz-calorimetria}{\ExerciseHeaderNB}}{}
  Un calorímetro contiene $\SI{40}{\gram}$ de agua a $\SI{22}{\celsius}$ y se le agregan $\SI{50}{\gram}$ de agua a $\SI{50}{\celsius}$, obteniéndose una temperatura final de $\SI{35}{\celsius}$. \textit{a}) Calcular el equivalente en agua del calorímetro. \textit{b}) En un nuevo experimento, este mismo calorímetro contiene $\SI{100}{\gram}$ de agua a una temperatura de $\SI{22}{\celsius}$, y se agregan $\SI{80}{\gram}$ de aluminio a $\SI{90}{\celsius}$. Calcular la temperatura de equilibrio. \textit{Dato:} Calor específico del aluminio: $\SI{0.22}{cal.\gram^{-1}.\celsius^{-1}}$.
\end{Exercise}
\begin{Answer}
	\begin{minipage}[t]{.4\textwidth}
    \textit{a}) $\SI{17.7}{\gram}$\\ \textit{b}) $\SI{30.8}{\celsius}$
  \end{minipage}
\end{Answer}
%

  \twocolumn[\colorsection{Transmisión del calor}]
\setcounter{figure}{0}
%
\begin{Exercise}
  \ifthenelse{\equal{\seleccionados}{true}}
  {\addToList{xyz-transmision}{\ExerciseHeaderNB}}{}
  Un extremo de una varilla metálica aislada se mantiene a $\SI{100}{\celsius}$, y el otro se mantiene a $\SI{0}{\celsius}$ con una mezcla de hielo y agua. La varilla tiene $\SI{60}{\centi\metre}$ de longitud y el área de su sección transversal es $\SI{1.25}{\square\centi\metre}$. El calor conducido por la varilla funde $\SI{8.5}{\gram}$ de hielo cada $\SI{10}{\minute}$. Calcule la conductividad térmica del metal.
\end{Exercise}
\begin{Answer}
  $\SI{227}{\watt.\metre^{-1}.\kelvin^{-1}}$
\end{Answer}
%
% \begin{Exercise}
%   A través de una ventana de vidrio de $\SI{1}{\square\metre}$ de área y $\SI{5}{\milli\metre}$ de espesor fluye calor a razón de $\SI{1600}{cal/\second}$, siendo la temperatura interior de $\SI{15}{\celsius}$ y la exterior de $\SI{25}{\celsius}$. Si la temperatura exterior aumenta a $\SI{35}{\celsius}$, ¿por cuál de las siguientes ventanas de área $A$ $\si{\square\metre}$ y espesor $w$ $\si{\milli\metre}$ la cambiaría para mantener el mismo flujo calórico?\\
% \begin{tabular*}{0.8\textwidth}{ccc}
%   \textit{a}) $A/w=0.40$ & \textit{b}) $A/w=0.13$ & \textit{c}) $A/w=0.20$\\
%   \textit{d}) $A/w=0.17$ & \textit{e}) $A/w=0.11$ & \textit{f}) $A/w=0.10$\\
% \end{tabular*}
% \end{Exercise}
% \begin{Answer}
%   Opción \textit{f})
% \end{Answer}
%
\begin{Exercise}
  Un método experimental para medir la conductividad térmica de un material aislante consiste en construir una caja del material y medir el aporte de potencia a un calentador eléctrico dentro de la caja, que mantiene el interior a una temperatura medida por encima de la temperatura de la superficie exterior. Suponga que en un aparato como el mencionado se requiere un aporte de potencia de $\SI{180}{\watt}$ para mantener la superficie interior de la caja $\SI{65.0}{\celsius}$ arriba de la temperatura de la superficie exterior. El área total de la caja es de $\SI{2.18}{\square\metre}$, y el espesor de la pared es de $\SI{3.90}{\centi\metre}$. Calcule la conductividad térmica del material en unidades del SI.
\end{Exercise}
\begin{Answer}
  $\SI{0.0495}{\watt.\metre^{-1}.\kelvin^{-1}}$
\end{Answer}
%
\begin{Exercise}\label{p:transmision00}
  \ifthenelse{\equal{\seleccionados}{true}}
  {\addToList{xyz-transmision}{\ExerciseHeaderNB}}{}
  En una casa se tiene una pared de ladrillos de $\SI{3}{\metre} \times \SI{4}{\metre}$, y espesor $\SI{15}{\centi\metre}$, que separa un ambiente a $\SI{25}{\celsius}$ del exterior a $\SI{5}{\celsius}$. Esta pared contiene una ventana que consiste en solo un panel de vidrio de $\SI{1.5}{\metre} \times \SI{1.5}{\metre} \times \SI{5}{\milli\metre}$. \textit{a}) Calcular la corriente de calor total a través del concreto y la ventana, sin incluir efectos de convección. \textit{b}) ¿Cuál es el porcentaje de calor que se pierde a través de la ventana respecto del total? \textit{c}) Repetir los cálculos para una situación donde la ventana es cubierta por una lámina de papel de espesor $\SI{0.750}{\milli\metre}$. \textit{Datos}: conductividad térmica del ladrillo = $\SI{0.6}{\watt.\metre^{-1}.\kelvin^{-1}}$; conductividad térmica del vidrio = $\SI{1}{\watt.\metre^{-1}.\kelvin^{-1}}$; conductividad térmica del papel = $\SI{0.050}{\watt.\metre^{-1}.\kelvin^{-1}}$.
\end{Exercise}
\begin{Answer}
	\begin{minipage}[t]{.4\textwidth}
    \textit{a}) $\SI{9780}{\watt}$\\ \textit{b}) 92\%\\ \textit{c}) $\SI{3030}{\watt}$ ($74.3$\%)
  \end{minipage}
\end{Answer}
%
%No se informan las conductividades, con la intención de que los estudiantes las busquen.
\begin{Exercise}
  \ifthenelse{\equal{\seleccionados}{true}}
  {\addToList{xyz-transmision}{\ExerciseHeaderNB}}{}
  Dos barras, una de latón y otra de cobre, están unidas extremo con extremo. La longitud de la barra de latón es $\SI{0.2}{\metre}$ y la de cobre es $\SI{0.8}{\metre}$. La sección transversal de cada segmento tiene un área de $\SI{0.005}{\square\metre}$. El extremo libre del segmento de latón está en contacto con agua hirviendo y el extremo libre del segmento de cobre se encuentra en contacto con una mezcla de hielo y agua, en ambos casos a presión atmosférica normal. Los lados de las varillas están aislados, por lo que no hay pérdida de calor a los alrededores. \textit{a}) ¿Cuál es la temperatura del punto en el que los segmentos de latón y de cobre se unen? \textit{b}) ¿Qué masa de hielo se funde en $\SI{5}{\minute}$ debido el calor conducido por la varilla compuesta?
\end{Exercise}
\begin{Answer}
	\begin{minipage}[t]{.4\textwidth}
    \textit{a}) $\SI{53.1}{\celsius}$; \textit{b}) $\SI{115}{\gram}$
  \end{minipage}
\end{Answer}
%
\begin{Exercise}\label{p:transmision02}
  El gráfico de la figura \ref{f:transmision02} representa la temperatura en función de la posición dentro de una barra cuya área transversal mide $\SI{3.00}{\centi\metre\squared}$ y su longitud total es $\SI{30}{\centi\metre}$. La barra está compuesta por dos materiales homogéneos, y conecta dos fuentes de calor a temperaturas constantes, transmitiendo $\SI{1.44}{cal/s}$. Determinar las conductividades térmicas de los materiales que forman la barra.
\end{Exercise}
\begin{Answer}
	\begin{minipage}[t]{.4\textwidth}
    $\SI{0.12}{cal/(cm.\celsius.s)}$ y $\SI{0.72}{cal/(cm.\celsius.s)}$
  \end{minipage}
\end{Answer}
%
\begin{center}
  \begin{tikzpicture}[scale=1]
    \begin{axis}[
      % every major x tick/.append style={thick,blue},
      clip=false,
      grid=both,
      minor x tick num=5,
      minor y tick num=3,
      xmin=0, xmax=30,
      ymin=0, ymax=100,
      xtick  align=center,
      xlabel={$x$~[cm]},
      ylabel={$T$~[$^\circ$C]}
      ];
      \draw [color=blue, very thick](0,90)--(15,30);
      \draw [color=blue, very thick](15,30)--(30,20);
    \end{axis}
  \end{tikzpicture}
  \captionof{figure}{Problema \ref{p:transmision02}\label{f:transmision02}}
\end{center}
%
\begin{Exercise}
  Se sueldan varillas de cobre, latón y acero para formar una figura en forma de Y. El área de la sección transversal de cada varilla es de $\SI{2}{\square\centi\metre}$. El extremo libre de la varilla de cobre se mantiene a $\SI{100}{\celsius}$, y los extremos libres de las varillas de latón y acero a $\SI{0}{\celsius}$. Suponga que no hay pérdida de calor por los laterales de las varillas, cuyas longitudes son: $\SI{13}{\centi\metre}$, $\SI{18}{\centi\metre}$ y $\SI{24}{\centi\metre}$ para la de cobre, latón y acero respectivamente. \textit{a}) ¿Qué temperatura tiene el punto de unión? \textit{b}) Calcule la corriente de calor en cada una de las tres varillas.
\end{Exercise}
\begin{Answer}
	\begin{minipage}[t]{.4\textwidth}
    \textit{a}) $\SI{78.4}{\celsius}$\\ \textit{b}) $H_\text{latón} = \SI{9.50}{\watt}$, $H_\text{acero} = \SI{3.28}{\watt}$ y $H_\text{cobre} = \SI{12.8}{\watt}$
  \end{minipage}
\end{Answer}
%
\begin{Exercise}
  Una pared de ladrillos de $\SI{20}{\centi\metre}$ de espesor y conductividad térmica $\SI{5E-4}{cal/(\second.\centi\metre.\celsius)}$, separa una habitación en la que el aire está a una temperatura de $\SI{15}{\celsius}$, del exterior donde el aire se encuentra a una temperatura de $\SI{-5}{\celsius}$. Si el coeficiente de convección interior es de $\SI{1E-4}{cal/(\second.\centi\square\metre.\celsius)}$ y el doble de  éste en el exterior, calcular: \textit{a}) La temperatura de la superficie interior de la pared. \textit{b}) La temperatura de la superficie exterior de la pared.
\end{Exercise}
\begin{Answer}
  \begin{minipage}[t]{.4\textwidth}
    \textit{a}) $\SI{11.4}{\celsius}$\\ \textit{b}) $\SI{-3.16}{\celsius}$
  \end{minipage}
\end{Answer}
%
\begin{Exercise}\label{p:transmision01}
  Una pared exterior está compuesta por una capa externa de madera de $\SI{3.0}{\centi\metre}$ de espesor y una capa interna de espuma de poliestireno de $\SI{2.2}{\centi\metre}$ de espesor. Considere que la conductividad térmica de la madera es $k_\text{m} = \SI{0.08}{\watt.\metre^{-1}.\kelvin^{-1}}$ y la del poliestireno es $k_\text{p} = \SI{0.01}{\watt.\metre^{-1}.\kelvin^{-1}}$. La temperatura del aire en el interior es $\SI{19}{\celsius}$ y la del aire en el exterior es $\SI{-10}{\celsius}$, y los coeficientes de convección del aire en el interior y del aire en el exterior valen $\SI{5}{\watt/(\square\metre.\kelvin)}$ y $\SI{12}{\watt/(\square\metre.\kelvin)}$ respectivamente. \textit{a}) Calcular la rapidez del flujo de calor por metro cuadrado a través de esta pared. \textit{b}) Calcular la temperatura en la superficie de contacto entre la madera y la espuma de poliestireno. \textit{c}) Realizar un gráfico de la temperatura en función de la posición, en la dirección del flujo de calor.
\end{Exercise}
\begin{Answer}
	\begin{minipage}[t]{.4\textwidth}
    \textit{a}) $\SI{10.2}{\watt/\square\metre}$\\ \textit{b}) $\SI{-5.36}{\celsius}$
  \end{minipage}
\end{Answer}
%
\begin{Exercise}
  Un carpintero construye una cabaña rústica que tiene un piso de dimensiones $\SI{3.50}{\metre} \times \SI{3.00}{\metre}$. Sus paredes, que miden $\SI{2.50}{\metre}$ de alto y $\SI{1.80}{\centi\metre}$ de grosor, están hechas de una madera cuya conductividad térmica vale $\SI{0.517}{cal/(\hour.\centi\metre.\celsius)}$, y serán aisladas con un material sintético de conductividad térmica igual a $\SI{0.947}{cal/(\hour.\centi\metre.\celsius)}$. Se desea instalar una estufa que entregue una potencia calorífica de $\SI{1100}{kcal/\hour}$ para mantener el interior a una temperatura de $\SI{25.0}{\celsius}$ cuando la temperatura exterior es $\SI{2.00}{\celsius}$. Despreciando la pérdida de calor a través del techo y del piso, calcule el espesor mínimo necesario del material aislante. Considere que el  coeficiente de convección del aire $\SI{2.00E-4}{cal/(\second.\centi\square\metre.\celsius)}$ tanto en el interior como en el exterior.
\end{Exercise}
\begin{Answer}
  $\SI{0.51}{\centi\metre}$
\end{Answer}
%
\begin{Exercise}
  En un edificio de oficinas se está considerando reemplazar ventanas de un solo panel de vidrio de $\SI{3}{\milli\metre}$ de espesor por ventanas de doble panel de vidrio de $\SI{3}{\milli\metre}$ de espesor, separados por $\SI{5}{\milli\metre}$ de aire estanco. En ambos casos la conductividad térmica del vidrio es $\SI{1}{\watt/(\metre.\kelvin)}$ y en el caso de doble panel la conductividad térmica del aire estanco es $\SI{0.025}{\watt/(\metre.\kelvin)}$. El coeficiente de transmisión interior y exterior vale $\SI{20}{\watt/(\metre\squared.\kelvin)}$ y se puede despreciar la contribución por convección en el aire estanco. Las oficinas se calefaccionan con energía eléctrica cuyo costo es $\SI{116.63}{\$/kWh}$, para mantener la temperatura interior $\SI{10}{\celsius}$ por encima de la temperatura exterior, durante 220 horas mensuales. ¿Cuánto dinero se ahorrarían cada mes por cada metro cuadrado de ventana que se reemplace?
\end{Exercise}
\begin{Answer}
	\begin{minipage}[t]{.4\textwidth}
    \$ 1650
  \end{minipage}
\end{Answer}
%
\begin{Exercise}
  Calcule la tasa de radiación de energía por unidad de área de un cuerpo negro a: \textit{a}) $\SI{273}{\kelvin}$ y \textit{b}) $\SI{2730}{\kelvin}$.
\end{Exercise}
\begin{Answer}
	\begin{minipage}[t]{.4\textwidth}
    \textit{a}) $\SI{315}{\watt/\square\metre}$\\ \textit{b}) $\SI{3.15E6}{\watt/\square\metre}$
  \end{minipage}
\end{Answer}
%
\begin{Exercise}
  \ifthenelse{\equal{\seleccionados}{true}}
  {\addToList{xyz-transmision}{\ExerciseHeaderNB}}{}
  La emisividad del tungsteno es $0.350$. Una esfera de tungsteno con un radio de $\SI{1.5}{\centi\metre}$ se suspende dentro de una cavidad grande, cuyas paredes están a $\SI{290}{\kelvin}$. ¿Qué aporte de potencia se requiere para mantener la esfera a una temperatura de $\SI{3000}{\kelvin}$, si se desprecia la conducción de calor por los soportes?
\end{Exercise}
\begin{Answer}
  $\SI{4540}{\watt}$
\end{Answer}
%
\begin{Exercise}
  Calcular cuánto calor neto pierde cada hora una persona desnuda en forma de radiación, cuando se encuentra en un ambiente a $\SI{15}{\celsius}$. Suponer que la superficie libre que emite (y recibe) calor es de $\SI{2.5}{\metre\squared}$, su temperatura es $\SI{33}{\celsius}$, y se comporta como un cuerpo negro.
\end{Exercise}
\begin{Answer}
	\begin{minipage}[t]{.4\textwidth}
    $\SI{963}{\kilo\joule}$
  \end{minipage}
\end{Answer}
%
\begin{Exercise}
  La tasa de energía radiante que llega del Sol a la atmósfera superior de la Tierra es cercana a $\SI{1.5}{\kilo\watt/\square\metre}$. La distancia promedio de la Tierra al Sol es $\SI{1.5E11}{\metre}$ y el radio del Sol es $\SI{6.96E8}{\metre}$. \textit{a}) Calcule la tasa de radiación de energía por unidad de área de la superficie solar. \textit{b}) Si el Sol irradia como cuerpo negro ideal, ¿qué temperatura tiene en su superficie?
\end{Exercise}
\begin{Answer}
	\begin{minipage}[t]{.4\textwidth}
    \textit{a}) $\approx\SI{70}{\mega\watt/\square\metre}$\\ \textit{b}) $\approx\SI{5900}{\kelvin}$
  \end{minipage}
\end{Answer}
%
  \twocolumn[\colorsection{Primer principio de la termodinámica}]
\setcounter{figure}{0}
%
\begin{Exercise}
  Un tanque de $\SI{20.0}{\liter}$ contiene $\SI{4.86E-4}{\kilogram}$ de helio a $\SI{18.0}{\celsius}$. La masa molar del helio es $\SI{4.00}{\gram/\mole}$. \textit{a}) ¿Cuántos moles de helio hay en el tanque? \textit{b}) ¿Cuál es la presión en el tanque en pascales y en atmósferas?
\end{Exercise}
\begin{Answer}
	\begin{minipage}[t]{.4\textwidth}
    \textit{a}) $\SI{0.122}{\mole}$\\ \textit{b}) $\SI{14750}{\pascal}$ o $\SI{0.146}{atm}$
  \end{minipage}
\end{Answer}
%
\begin{Exercise}
  \ifthenelse{\equal{\seleccionados}{true}}
  {\addToList{xyz-primerppio}{\ExerciseHeaderNB}}{}
  Un tanque cilíndrico tiene un pistón ajustado que permite modificar el volumen del tanque. Originalmente, el tanque contiene $\SI{0.110}{\cubic\metre}$ de aire a $\SI{0.355}{atm}$ de presión. Se tira lentamente del pistón hasta aumentar el volumen del aire a $\SI{0.390}{\cubic\metre}$. Si la temperatura permanece constante, ¿qué valor final tiene la presión?
\end{Exercise}
\begin{Answer}
  $\SI{0.100}{atm}$
\end{Answer}
%
\begin{Exercise}
  Un matraz de $\SI{1.50}{\liter}$, provisto de una llave de paso, contiene etano gaseoso ($\text{C}_2\text{H}_6$) a $\SI{300}{\kelvin}$ y presión atmosférica ($\SI{101.3}{\kilo\pascal}$). La masa molar del etano es $\SI{30.1}{\gram/\mole}$. El sistema se calienta a $\SI{490}{\kelvin}$, con la llave abierta a la atmósfera. Luego se cierra la llave y el matraz se enfría a su temperatura  original. \textit{a}) Calcule la presión final del etano en el matraz. \textit{b}) ¿Cuántos gramos de etano quedan en el matraz?
\end{Exercise}
\begin{Answer}
	\begin{minipage}[t]{.4\textwidth}
    \textit{a}) $\SI{62}{\kilo\pascal}$\\ \textit{b}) $\SI{1.12}{\gram}$
  \end{minipage}
\end{Answer}
%
\begin{Exercise}
  Dos moles de gas ideal se calientan a presión constante desde $\SI{27.0}{\celsius}$ hasta $\SI{107}{\celsius}$. Calcule el trabajo efectuado por el gas.
\end{Exercise}
\begin{Answer}
  $\SI{1330}{\joule}$
\end{Answer}
%
\begin{Exercise}
  \ifthenelse{\equal{\seleccionados}{true}}
  {\addToList{xyz-primerppio}{\ExerciseHeaderNB}}{}
  Seis moles de gas ideal están en un cilindro provisto en un extremo con un pistón móvil. La temperatura inicial del gas es $\SI{27.0}{\celsius}$ y se desplaza el pistón manteniendo la presión del gas constante. Calcule la temperatura final del gas una vez que haya efectuado $\SI{2.4}{\kilo\joule}$ de trabajo.
\end{Exercise}
\begin{Answer}
  $\SI{75.1}{\celsius}$
\end{Answer}
%
\begin{Exercise}\label{p:primerppio01}
  La gráfica de la figura \ref{f:primerppio01} muestra un diagrama $p$-$V$ del aire en un pulmón cuando una persona inhala y luego exhala una respiración profunda. Estas gráficas, obtenidas en la práctica clínica, normalmente están algo curvadas, pero modelamos una como un conjunto de líneas rectas de la misma forma general. (Importante: La presión indicada es la presión manométrica, no la presión absoluta). \textit{a}) ¿Cuántos joules de trabajo neto realiza el pulmón de esta persona durante una respiración completa? \textit{b}) El proceso que aquí se representa es algo diferente de los que se han estudiado, ya que el cambio de presión se debe a los cambios en la cantidad de gas en el pulmón, y no a los cambios de temperatura. (Piense en su propia respiración, sus pulmones no se expanden porque se han calentado). Si la temperatura del aire en el pulmón permanece en un valor razonable de $\SI{20}{\celsius}$, ¿cuál es el número máximo de moles en el pulmón de esta persona durante una respiración?
\end{Exercise}
\begin{Answer}
	\begin{minipage}[t]{.4\textwidth}
    \textit{a}) $\SI{1}{\joule}$\\ \textit{b}) $\SI{0.06}{\mole}$
  \end{minipage}
\end{Answer}
%
\begin{center}
  \begin{tikzpicture}[scale=1]
    \begin{axis}[
                 every major x tick/.append style={thick,blue},
                 clip=false,
                 grid=both,
                 minor x tick num=3,        %un minor tick es decir 0.5
                 minor y tick num=3,
                 xmin=0, xmax=1.6,           %min y max para los ejes, NO PARA EL DOMINIO
                 ymin=0, ymax=13, 
                 %axis y line=center,        %alinea el eje al centro de la figura
                 %axis x line=middle,        %sino pone 2 ejes x
                 xtick  align=center,
                 xlabel={$V$~[L]},
                 ylabel={$p$~[mmHg]}         
                ];
    \addplot [color=blue, thick] [id=pulmon1,samples= 180, domain=0.1:0.4]  {1+80/3*(x-0.1)};
    \addplot [color=blue, thick] [id=pulmon2,samples= 180, domain=0.4:1.4]  {9+2*(x-0.4)};
    \addplot [color=blue, thick] [id=pulmon3,samples= 180, domain=1:1.4]  {2+90/4*(x-1)};
    \addplot [color=blue, thick] [id=pulmon4,samples= 180, domain=0.1:1]  {1+10/9*(x-0.1)};
    \draw [red, -{Stealth}] (0.15,4)--(0.25,20/3) node[midway,sloped,above] {inhalación};
    \draw [red, -{Stealth}] (0.6,10)--(0.85,10.5) node[midway,sloped,above] {inhalación};
    \draw [red, -{Stealth}] (1.15,7.25)--(1.05,5) node[midway,sloped,above] {exhalación};
    \draw [red, -{Stealth}] (0.7,2.33)--(0.4,2) node[midway,sloped,above] {exhalación};
    \end{axis}
  \end{tikzpicture}
  \captionof{figure}{Problema \ref{p:primerppio01}\label{f:primerppio01}}
\end{center}
%
\begin{Exercise}
  Durante el tiempo en que $\SI{0.305}{\mole}$ de un gas ideal experimentan una compresión isotérmica a  $\SI{22}{\celsius}$, su entorno efectúa $\SI{468}{\joule}$ de trabajo sobre él. \textit{a}) Si la presión final es $\SI{1.76}{atm}$, ¿cuál fue la presión inicial? \textit{b}) Realice una gráfica $p$-$V$ para este proceso.
\end{Exercise}
\begin{Answer}
	\begin{minipage}[t]{.4\textwidth}
    \textit{a}) $\SI{0.941}{atm}$
  \end{minipage}
\end{Answer}
%
\begin{Exercise}
  Cuando se hierve agua a una presión de $\SI{2.00}{atm}$, el calor de vaporización es $\SI{2.20}{\mega\joule/\kilogram}$ y la temperatura del punto de ebullición es $\SI{120}{\celsius}$. A esta presión, $\SI{1.00}{\kilogram}$ de agua ocupa un volumen de $\SI{1.00E-3}{\cubic\metre}$, y $\SI{1.00}{\kilogram}$ de vapor de agua ocupa un volumen de $\SI{0.824}{\cubic\metre}$. Calcule el incremento en la energía interna del agua cuando se forma $\SI{1.00}{\kilogram}$ de vapor de agua a esta temperatura.
\end{Exercise}
\begin{Answer}
	\begin{minipage}[t]{.4\textwidth}
    $\Delta U = \SI{2.03E6}{\joule}$ (es menor al calor recibido)
  \end{minipage}
\end{Answer}
%
\begin{Exercise}\label{p:primerppio02}
  \ifthenelse{\equal{\seleccionados}{true}}
  {\addToList{xyz-primerppio}{\ExerciseHeaderNB}}{}
  Considere el ciclo cerrado $a\rightarrow b \rightarrow c \rightarrow d \rightarrow a$ mostrado en la figura \ref{f:primerppio02}. \textit{a}) Encuentre una expresión para el trabajo total efectuado por el sistema en este proceso. \textit{b}) Encuentre una expresión para el trabajo total efectuado por el sistema si el  ciclo se recorre en sentido opuesto.
\end{Exercise}
\begin{Answer}
	\begin{minipage}[t]{.4\textwidth}
    \textit{a}) $W_{abcda} = (p_1-p_0)(V_1-V_0)$\\ \textit{b}) $W_{adcba} = -W_{abcda}$
  \end{minipage}
\end{Answer}
%
\begin{center}
  \begin{tikzpicture}[scale=1]
    \begin{axis}[
      axis x line=bottom,
      axis y line=left,
      xmin=0, xmax=6,           %min y max para los ejes, NO PARA EL DOMINIO
      ymin=0, ymax=9, 
      xlabel={$V$},
      ylabel={$p$},
      xtick={1,5},
      xticklabels={$V_0$,$V_1$},
      ytick={2,7},
      yticklabels={$p_0$,$p_1$}
      ]
    % \draw [color=blue, very thick][-latex](160pt,20pt)..controls(100pt,25pt) and (60pt,45pt)..(42pt,58pt);
    \draw [color=blue, very thick][-latex](5,2)--(3,2);
    \draw [color=blue, very thick](3,2)--(1,2);
    \draw [color=blue, very thick][-latex] (1,2)--(1,4.5);
    \draw [color=blue, very thick] (1,4.5)--(1,7);
    \draw [color=blue, very thick][-latex] (1,7)--(3,7);
    \draw [color=blue, very thick] (3,7)--(5,7);
    \draw [color=blue, very thick][-latex] (5,7)--(5,4.5);
    \draw [color=blue, very thick] (5,4.5)--(5,2);
    \draw [dashed, thick] (1,0)--(1,2);
    \draw [dashed, thick] (5,0)--(5,2);
    \draw [dashed, thick] (0,2)--(1,2);
    \draw [dashed, thick] (0,7)--(1,7);
    \filldraw(1,2)circle(2pt) (1,7)circle (2pt)(5,7)circle(2pt)(5,2)circle(2pt);
    \draw (1,2) node [below left] {$a$};
    \draw (1,7) node [above left] {$b$};
    \draw (5,7) node [above right] {$c$};
    \draw (5,2) node [below right] {$d$};
    \end{axis}
  \end{tikzpicture}
  \captionof{figure}{Problema \ref{p:primerppio02}\label{f:primerppio02}}
\end{center}
%
\begin{Exercise}\label{p:primerppio03}
  En la figura \ref{f:primerppio03} se muestra el diagrama $p$-$V$ del proceso $a\rightarrow b \rightarrow c$ que implica $\SI{0.0175}{\mole}$ de un gas ideal. \textit{a}) ¿Cuál fue la temperatura más baja que alcanzó el gas en  este proceso? ¿Dónde ocurrió? \textit{b}) ¿Cuánto trabajo realizó o recibió el gas en el proceso $a \rightarrow b$? \textit{c}) ¿Cuánto trabajo realizó o recibió el gas en el proceso $b \rightarrow c$? \textit{d}) Si se entregaron $\SI{215}{\joule}$ de calor al gas durante $a\rightarrow b \rightarrow c$, ¿cuántos de esos Joules se destinaron a la energía interna?
\end{Exercise}
\begin{Answer}
	\begin{minipage}[t]{.4\textwidth}
    \textit{a}) $\SI{278}{\kelvin}$ en el estado $a$\\ \textit{b}) $\SI{0}{\joule}$\\ \textit{c}) realizó $\SI{162}{\joule}$ de trabajo\\ \textit{d}) $\SI{53}{\joule}$
  \end{minipage}
\end{Answer}
%
\begin{center}
  \begin{tikzpicture}[scale=1]
    \begin{axis}[
      every major x tick/.append style={thick,blue},
      clip=false,
      grid=both,
      minor x tick num=3,
      minor y tick num=3,
      xmin=0, xmax=7,
      ymin=0, ymax=0.6,
      xtick  align=center,
      xlabel={$V$~[L]},
      ylabel={$p$~[atm]}
      ];
      \draw [color=blue, very thick][-latex](2,0.2)--(2,0.5);
      \draw [color=blue, very thick][-latex](2,0.5)--(6,0.3);
      \draw (2,0.2) node [below left] {$a$};
      \draw (2,0.5) node [above left] {$b$};
      \draw (6,0.3) node [right] {$c$};
      \fill [black](2,0.2) circle(2pt);
      \fill [black](2,0.5) circle(2pt);
      \fill [black](6,0.3) circle(2pt);
    \end{axis}
  \end{tikzpicture}
  \captionof{figure}{Problema \ref{p:primerppio03}\label{f:primerppio03}}
\end{center}
%
\begin{Exercise}\label{p:primerppio04}
  \ifthenelse{\equal{\seleccionados}{true}}
  {\addToList{xyz-primerppio}{\ExerciseHeaderNB}}{}
  El gráfico en la figura \ref{f:primerppio04} muestra un diagrama $p$-$V$ de $\SI{3.25}{\mole}$ de helio ideal gaseoso. La parte $c \rightarrow a$ de este proceso es isotérmica. \textit{a}) Calcule el calor intercambiado por el helio en los procesos $a \rightarrow b$, $b \rightarrow c$ y $c \rightarrow a$. \textit{b}) Calcule la variación de energía interna del He en esos mismos procesos.
\end{Exercise}
\begin{Answer}
	\begin{minipage}[t]{.4\textwidth}
    \textit{a}) $Q_{ab}=\SI{60.0}{\kilo\joule}$, $Q_{bc}=\SI{-36.0}{\kilo\joule}$ y $Q_{ca}=\SI{-11.1}{\kilo\joule}$\\ \textit{b}) $\Delta U_{ab}=\SI{36.0}{\kilo\joule}$, $\Delta U_{bc}=\SI{-36.0}{\kilo\joule}$ y $\Delta U_{ca}=\SI{0}{\kilo\joule}$
  \end{minipage}
\end{Answer}
%
\begin{center}
  \begin{tikzpicture}[scale=1]
    \begin{axis}[
      %ticks=none,
      axis x line=bottom,
      axis y line=left,
      xmin=0, xmax=5,           %min y max para los ejes, NO PARA EL DOMINIO
      ymin=1, ymax=9, 
      xlabel={$V$~[$\si{\cubic\metre}$]},
      ylabel={$p$~[$10^5 \si{\pascal}$]},
      xtick={1,4},
      xticklabels={0.010,0.040},
      ytick={2},
      yticklabels={2.0}
      %yticklabels={},
      %extra x ticks={0.04},
      %extra y ticks={2}
      ];
    \addplot [-latex, color=blue, very thick] [samples= 180, domain=1:2.5]  {8};
    \addplot [color=blue, very thick] [samples= 180, domain=2.5:4]  {8};
    \addplot [color=blue, very thick] [samples= 180, domain=1:2.5]  {8/x};
    \addplot [latex-, color=blue, very thick] [samples= 180, domain=2.5:4]  {8/x};
    \addplot[color = black, dashed, thick] coordinates {(4, 0) (4, 2) (0, 2)};
    \addplot[color = black, dashed, thick] coordinates {(1, 0) (1, 8)};
    \draw [color=blue, very thick][-latex](4,8)--(4,5);
    \draw [color=blue, very thick](4,5)--(4,2);
    \draw (1,8) node [above left] {$a$};
    \draw (4,8) node [above right] {$b$};
    \draw (4,2) node [right] {$c$};
    \fill [black](1,8) circle(2pt);
    \fill [black](4,8) circle(2pt);
    \fill [black](4,2) circle(2pt);
    \end{axis}
  \end{tikzpicture}
  \captionof{figure}{Problema \ref{p:primerppio04}\label{f:primerppio04}}
\end{center}
%
\begin{Exercise}\label{p:primerppio05}
  \textit{a}) Una tercera parte de un mol de gas He evoluciona a lo largo de la trayectoria $a \rightarrow b \rightarrow c$ representada por la línea continua en la figura \ref{f:primerppio05}. Suponga que el gas se puede tratar como ideal. ¿Cuánto calor intercambia gas? \textit{b}) Si, en vez de ello, el gas pasa  del estado $a$ al estado $c$ evolucionando a lo largo de la línea horizontal punteada en la figura, ¿cuánto calor intercambia el gas en este caso?
\end{Exercise}
\begin{Answer}
	\begin{minipage}[t]{.4\textwidth}
    \textit{a}) $\SI{3200}{\joule}$\\ \textit{b}) $\SI{2000}{\joule}$
  \end{minipage}
\end{Answer}
%
\begin{center}
  \begin{tikzpicture}[scale=1]
    \begin{axis}[
      every major x tick/.append style={thick,blue},
      clip=false,
      grid=both,
      minor x tick num=3,
      minor y tick num=3,
      xmin=0, xmax=1.1,
      ymin=0, ymax=4.5,
      xtick  align=center,
      xlabel={$V$~[$10^{-2} \si{\cubic\metre}$]},
      ylabel={$p$~[$10^5 \si{\pascal}$]}
      ];
      \draw [color=blue, very thick][-latex](0.2,1)--(0.4,2.5);
      \draw [color=blue, very thick](0.4,2.5)--(0.6,4);
      \draw [color=blue, very thick][-latex](0.6,4)--(0.8,2.5);
      \draw [color=blue, very thick](0.8,2.5)--(1,1);
      \draw [color=blue, dashed, very thick][-latex](0.2,1)--(0.6,1);
      \draw [color=blue, dashed, very thick](0.6,1)--(1,1);
      \draw (0.2,1) node [below left] {$a$};
      \draw (0.6,4) node [above left] {$b$};
      \draw (1,1) node [below right] {$c$};
      \fill [black](0.2,1) circle(2pt);
      \fill [black](0.6,4) circle(2pt);
      \fill [black](1,1) circle(2pt);
    \end{axis}
  \end{tikzpicture}
  \captionof{figure}{Problema \ref{p:primerppio05}\label{f:primerppio05}}
\end{center}
%
\begin{Exercise}
  \ifthenelse{\equal{\seleccionados}{true}}
  {\addToList{xyz-primerppio}{\ExerciseHeaderNB}}{}
  Se colocan $\SI{0.20}{\mole}$ de un gas ideal diatómico en un recipiente a $\SI{3.0}{atm}$ y $\SI{500}{\kelvin}$. Se efectúa con el gas el siguiente ciclo: \textit{i}) Se expande isotérmicamente desde el estado \textit{A} hasta duplicar su volumen llegando al estado \textit{B}. \textit{ii}) A volumen constante se reduce su temperatura hasta $\SI{300}{\kelvin}$, alcanzando el estado \textit{C}. \textit{iii}) A presión constante se reduce su volumen hasta el volumen inicial (estado \textit{D}). \textit{iv}) El gas aumenta su temperatura a volumen constante hasta el estado \textit{A}. Para este ciclo se pide: \textit{a}) Realizar un diagrama $p-V$ indicando en qué procesos el gas realiza o recibe trabajo y en cuáles absorbe o cede calor. \textit{b}) Calcular el trabajo neto realizado por el gas. \textit{c}) Calcular el cociente entre el trabajo realizado neto y el calor absorbido por el gas.
\end{Exercise}
\begin{Answer}
	\begin{minipage}[t]{.4\textwidth}
    \textit{b}) $W = \SI{327}{\joule}$\\ \textit{c}) $0.16$
  \end{minipage}
\end{Answer}
%
\begin{Exercise}
  Un mol de un gas ideal diatómico se comprime lentamente a un tercio de su volumen original. En esta compresión, la magnitud del trabajo realizado sobre el gas es $\SI{600}{\joule}$. \textit{a}) Si el proceso es isotérmico, ¿cuál es el valor del calor $Q$ para el gas? ¿El flujo de calor es hacia adentro o hacia afuera del gas? \textit{b}) Si el proceso es isobárico, ¿cuál es el cambio en la energía interna del gas? ¿Aumenta o disminuye su energía interna?
\end{Exercise}
\begin{Answer}
	\begin{minipage}[t]{.4\textwidth}
    \textit{a}) $Q = \SI{-600}{\joule}$, el gas cede calor\\ \textit{b}) $\Delta U = \SI{-1500}{\joule}$, disminuye
  \end{minipage}
\end{Answer}
%
\begin{Exercise}
  Un cilindro con pistón contiene $\SI{0.15}{\mole}$ de nitrógeno a $\SI{0.18}{\mega\pascal}$ y $\SI{300}{\kelvin}$, que se puede tratar como un gas ideal. Primero, el gas se comprime isobáricamente a la mitad de su volumen original. Luego se expande adiabáticamente hasta su volumen original. Por último, se calienta isocóricamente hasta su presión original. Calcule el trabajo del gas en este ciclo.
\end{Exercise}
\begin{Answer}
	\begin{minipage}[t]{.4\textwidth}
    $W = \SI{-74.4}{\joule}$
  \end{minipage}
\end{Answer}
%

  \twocolumn[\colorsection{Segundo principio de la termodinámica}]
\setcounter{figure}{0}
%
\begin{Exercise}
  En un calorímetro ideal (adiabático y de capacidad calorífica despreciable) se introducen $\SI{50}{\gram}$ de agua a $\SI{15}{\celsius}$ y $\SI{50}{\gram}$ de agua a $\SI{5}{\celsius}$, y se espera hasta que el sistema (la mezcla) alcance el equilibrio térmico. \textit{a}) Calcule la variación de entropía del sistema. \textit{b}) ¿Cuál es la variación de entropía del Universo en este proceso?
\end{Exercise}
\begin{Answer}
	\begin{minipage}[t]{.4\textwidth}
    \textit{a}) $\SI{0.016}{cal/\kelvin}$\\ \textit{b}) $\SI{0.016}{cal/\kelvin}$
  \end{minipage}
\end{Answer}
%
\begin{Exercise}
  Un bloque de hielo de $\SI{4.5}{\kilogram}$ a $\SI{0}{\celsius}$ cae en el océano y se funde. La temperatura media del océano es $\SI{3.5}{\celsius}$, incluyendo las aguas profundas. ¿En qué medida la fusión de este hielo cambia la entropía del mundo? ¿La aumenta o la disminuye? (La variación de la temperatura del océano mientras este bloque de hielo se funde es despreciable.)
\end{Exercise}
\begin{Answer}
	\begin{minipage}[t]{.4\textwidth}
    $\Delta S = \SI{71.3}{\joule/\kelvin}$
  \end{minipage}
\end{Answer}
%
\begin{Exercise}
  Un motor diesel realiza $\SI{2200}{\joule}$ de trabajo mecánico y cede $\SI{4300}{\joule}$ de calor en cada ciclo. \textit{a}) ¿Cuánto calor debe suministrarse al motor en cada ciclo? \textit{b}) Calcule la eficiencia térmica del motor.
\end{Exercise}
\begin{Answer}
	\begin{minipage}[t]{.4\textwidth}
    \textit{a}) $\SI{6500}{\joule}$\\ \textit{b}) 34\%
  \end{minipage}
\end{Answer}
%
\begin{Exercise}\label{p:segundoppio01}
  \ifthenelse{\equal{\seleccionados}{true}}
  {\addToList{xyz-segundoppio}{\ExerciseHeaderNB}}{}
  La figura \ref{f:segundoppio01} muestra un esquema de una máquina térmica cuyo depósito de alta temperatura está a una temperatura $T_c = \SI{620}{\kelvin}$, recibe $Q_\text{abs} = \SI{550}{\joule}$ de calor a esta temperatura en cada ciclo y cede $Q_\text{ced} = \SI{-395}{\joule}$ al depósito frío, a una temperatura $T_f = \SI{378}{\kelvin}$. \textit{a}) ¿Cuánto trabajo mecánico realiza la máquina en cada ciclo? \textit{b}) ¿Cuál es la eficiencia térmica de la máquina? \textit{c}) ¿Se trata de una máquina ideal o real?
\end{Exercise}
\begin{Answer}
	\begin{minipage}[t]{.4\textwidth}
    \textit{a}) $W = \SI{155}{\joule}$\\ \textit{b}) $e = 0.282$\\ \textit{c}) la máquina es real
  \end{minipage}
\end{Answer}
%
\begin{center}
  \begin{tikzpicture}[scale=1]
      
    % It needs:
    % \documentclass[dvipsnames]{article}
    % \usepackage{xcolor}
    % \usetikzlibrary{arrows.meta}
    
    \draw [rounded corners, thick, fill=pink] (0,4) rectangle (3,5);
    \draw [rounded corners, thick, fill=cyan] (0,0) rectangle (3,1);
    \draw [thick, fill=LimeGreen] (1.5,2.5) circle [radius=0.5];
    \draw (1.5,4.5) node {$T_c$};
    \draw (1.5,0.5) node {$T_f$};
    \draw (1.5,2.5) node {gas};

    \draw [color=pink] [-{Triangle[width=25pt,length=8pt]}, line width=18pt](1.5,3.95) -- (1.5, 3);
    \draw (1.5,3.5) node {$Q_c$};
    \draw [color=cyan] [-{Triangle[width=25pt,length=8pt]}, line width=18pt](1.5,1.97) -- (1.5, 1);
    \draw (1.5,1.5) node {$Q_f$};
    \draw [color=LimeGreen] [-{Triangle[width=25pt,length=8pt]}, line width=18pt](2.02,2.5) -- (3.5, 2.5);
    \draw (2.75,2.5) node {$W$};

  \end{tikzpicture}
  \captionof{figure}{Problema \ref{p:segundoppio01}\label{f:segundoppio01}}
\end{center}
%
\begin{Exercise}\label{p:segundoppio02}
  En la tabla \ref{t:segundoppio02} se muestra el calor absorbido, el calor cedido (en valor absoluto) y el trabajo realizado en cada ciclo de cuatro máquinas térmicas diseñadas para funcionar entre dos reservorios a temperaturas constantes de $\SI{800}{\kelvin}$ y $\SI{400}{\kelvin}$. ¿Cuáles de estas máquinas son posibles, debido a que no violan ningún principio termodinámico?\\
  \begin{table}[h]
    \caption{Ejercicio \ref{p:segundoppio02}}\label{t:segundoppio02}
    \centering
    \begin{tabular}{cccc}
        \hline
        \textbf{Máquina} & \textbf{$Q$ abs. [J]} & \textbf{$Q$ ced. [J]} & \textbf{$W$ [J]}\\
        \hline
        \textit{a} & 1000 & 600 & 600\\
        \textit{b} & 1000 & 400 & 600\\
        \textit{c} & 1000 & 500 & 500\\
        \textit{d} & 1000 & 700 & 300\\
        \hline
    \end{tabular}
  \end{table}
\end{Exercise}
\begin{Answer}
	\begin{minipage}[t]{.4\textwidth}
    Máquinas \textit{c} y \textit{d}
  \end{minipage}
\end{Answer}
%
\begin{Exercise}
  Un refrigerador tiene un coeficiente de rendimiento de $2.10$. Durante cada ciclo, absorbe $\SI{3.4E4}{\joule}$ de calor del depósito frío. \textit{a}) ¿Cuánta energía mecánica se requiere en cada ciclo para operar el refrigerador? \textit{b}) Durante cada ciclo, ¿cuánto calor se desecha al depósito caliente?
\end{Exercise}
\begin{Answer}
	\begin{minipage}[t]{.4\textwidth}
    \textit{a}) $\SI{1.62E4}{\joule}$\\ \textit{b}) $\SI{5.02E4}{\joule}$
  \end{minipage}
\end{Answer}
%
\begin{Exercise}
  Un acondicionador de aire tiene un coeficiente de rendimiento de $2.9$ en un día caluroso, y utiliza $\SI{850}{\watt}$ de potencia eléctrica. \textit{a}) ¿Cuántos Joules de calor extrae el sistema de aire acondicionado de la habitación en cada minuto? \textit{b}) ¿Cuántos Joules de calor entrega el sistema de aire acondicionado al aire caliente del exterior en cada minuto?
\end{Exercise}
\begin{Answer}
	\begin{minipage}[t]{.4\textwidth}
    \textit{a}) $\SI{148}{\kilo\joule}$\\ \textit{b}) $\SI{199}{\kilo\joule}$
  \end{minipage}
\end{Answer}
%
\begin{Exercise}
  \ifthenelse{\equal{\seleccionados}{true}}
  {\addToList{xyz-segundoppio}{\ExerciseHeaderNB}}{}
  Una máquina de Carnot opera entre $\SI{500}{\celsius}$ y $\SI{100}{\celsius}$ con un suministro de calor de $\SI{250}{\joule}$ por ciclo. \textit{a}) ¿Cuánto calor se entrega al depósito frío en cada ciclo? \textit{b}) ¿Qué número mínimo de ciclos se requieren para que la máquina levante una piedra de $\SI{500}{\kilogram}$ a una altura de $\SI{100}{\metre}$?
\end{Exercise}
\begin{Answer}
	\begin{minipage}[t]{.4\textwidth}
    \textit{a}) $\SI{121}{\joule}$\\ \textit{b}) $\SI{3.79E3}{ciclos}$
  \end{minipage}
\end{Answer}
%
\begin{Exercise}
  Una máquina de Carnot tiene una eficiencia del 59\% y realiza $\SI{25}{\kilo\joule}$ de trabajo en cada ciclo. \textit{a}) ¿Cuánto calor extrae la máquina de su fuente de calor en cada ciclo? \textit{b}) Suponga que la máquina expulsa calor a una temperatura ambiente de $\SI{20}{\celsius}$. ¿Cuál es la temperatura de su fuente de calor?
\end{Exercise}
\begin{Answer}
  \begin{minipage}[t]{.4\textwidth}
    \textit{a}) $\SI{42.37}{\kilo\joule}$; \textit{b}) $\SI{442}{\celsius}$
  \end{minipage}
\end{Answer}
%
\begin{Exercise}\label{p:segundoppio05}
  Calcule la eficiencia térmica de una máquina en la que $n$ moles de un gas ideal diatómico realizan el ciclo $1 \rightarrow 2 \rightarrow 3 \rightarrow 4 \rightarrow 1$ que se muestra en la figura \ref{f:segundoppio05}.
\end{Exercise}
\begin{Answer}
  \begin{minipage}[t]{.4\textwidth}
    $e = 2/19$
  \end{minipage}
\end{Answer}
%
\begin{center}
  \begin{tikzpicture}[scale=0.9]
    \begin{axis}[
      % ticks=none,
      axis x line=bottom,
      axis y line=left,
      xmin=0.5, xmax=2.5,
      ymin=0.5, ymax=2.5, 
      xlabel={$V$},
      ylabel={$p$},
      xtick={1,2},
      xticklabels={$V_0$,$2V_0$},
      ytick={1,2},
      yticklabels={$p_0$,$2p_0$}
      ];
    \addplot[color = black, dashed, thick] coordinates {(0, 2) (1, 2)};
    \addplot[color = black, dashed, thick] coordinates {(0, 1) (1, 1)};
    \addplot[color = black, dashed, thick] coordinates {(1, 0) (1, 1)};
    \addplot[color = black, dashed, thick] coordinates {(2, 0) (2, 1)};
    \draw [color=blue, very thick][-latex](1,2)--(1.5,2);
    \draw [color=blue, very thick](1.5,2)--(2,2);
    \draw [color=blue, very thick][-latex](2,2)--(2,1.5);
    \draw [color=blue, very thick](2,1.5)--(2,1);
    \draw [color=blue, very thick][-latex](2,1)--(1.5,1);
    \draw [color=blue, very thick](1.5,1)--(1,1);
    \draw [color=blue, very thick][-latex](1,1)--(1,1.5);
    \draw [color=blue, very thick](1,1.5)--(1,2);
    \draw (1,1) node [below left] {1};
    \draw (1,2) node [above left] {2};
    \draw (2,2) node [above right] {3};
    \draw (2,1) node [below right] {4};
    \fill [black](1,1) circle(2pt);
    \fill [black](1,2) circle(2pt);
    \fill [black](2,2) circle(2pt);
    \fill [black](2,1) circle(2pt);
    \end{axis}
  \end{tikzpicture}
  \captionof{figure}{Problema \ref{p:segundoppio05}\label{f:segundoppio05}}
\end{center}
%
\begin{Exercise}\label{p:segundoppio06}
  Calcule la eficiencia térmica de una máquina en la que $n$ moles de un gas ideal diatómico realizan el ciclo mostrado en la figura \ref{f:segundoppio06}.
\end{Exercise}
\begin{Answer}
	\begin{minipage}[t]{.4\textwidth}
    $e = \ln{(3)}/(2\ln{(3)}+7/2)$
  \end{minipage}
\end{Answer}
%
\begin{center}
  \begin{tikzpicture}[scale=0.9]
    \begin{axis}[
      % ticks=none,
      axis x line=bottom,
      axis y line=left,
      xmin=0.5, xmax=2.5,
      ymin=0.5, ymax=3.5, 
      xlabel={$T$},
      ylabel={$p$},
      xtick={1,2},
      xticklabels={$T_0$,$2T_0$},
      ytick={1,3},
      yticklabels={$p_0$,$3p_0$}
      %extra x ticks={0.04},
      %extra y ticks={2}
      ];
    \addplot[color = black, dashed, thick] coordinates {(0, 3) (1, 3)};
    \addplot[color = black, dashed, thick] coordinates {(0, 1) (1, 1)};
    \addplot[color = black, dashed, thick] coordinates {(1, 0) (1, 1)};
    \addplot[color = black, dashed, thick] coordinates {(2, 0) (2, 1)};
    % \addplot[color = black, dashed, thick] coordinates {(9, 0) (9, 1.5)};
    \draw [color=blue, very thick][-latex](1,3)--(1.5,3);
    \draw [color=blue, very thick](1.5,3)--(2,3);
    \draw [color=blue, very thick][-latex](2,3)--(2,2);
    \draw [color=blue, very thick](2,2)--(2,1);
    \draw [color=blue, very thick][-latex](2,1)--(1.5,1);
    \draw [color=blue, very thick](1.5,1)--(1,1);
    \draw [color=blue, very thick][-latex](1,1)--(1,2);
    \draw [color=blue, very thick](1,2)--(1,3);
    \fill [black](1,1) circle(2pt);
    \fill [black](1,3) circle(2pt);
    \fill [black](2,3) circle(2pt);
    \fill [black](2,1) circle(2pt);
    \end{axis}
  \end{tikzpicture}
  \captionof{figure}{Problema \ref{p:segundoppio06}\label{f:segundoppio06}}
\end{center}
%
\begin{Exercise}
  Un cilindro contiene oxígeno a una presión de $\SI{2.00}{atm}$, en un volumen de $\SI{4.00}{\liter}$ y a una temperatura de $\SI{300}{\kelvin}$. Suponga que el oxígeno se puede tratar como gas ideal mientras se somete a los siguientes procesos: \textit{i}) Calentamiento a presión constante desde el estado inicial (estado 1) al estado 2, donde $T = \SI{450}{\kelvin}$. \textit{ii}) Enfriamiento a volumen constante hasta $\SI{250}{\kelvin}$ (estado 3). \textit{iii}) Compresión a temperatura constante hasta un volumen de $\SI{4.00}{\liter}$ (estado 4). \textit{iv}) Calentamiento a volumen constante hasta $\SI{300}{\kelvin}$, regresando el sistema al estado 1. Determine la eficiencia de este dispositivo como máquina térmica y compárela con la de una máquina de ciclo de Carnot que opera entre las mismas temperaturas mínima y máxima de $\SI{250}{\kelvin}$ y $\SI{450}{\kelvin}$.
\end{Exercise}
\begin{Answer}
	\begin{minipage}[t]{.4\textwidth}
    $7.5\%$ (el máximo posible es $44.4\%$)
  \end{minipage}
\end{Answer}
%
\begin{Exercise}\label{p:segundoppio04}
  Una máquina térmica opera siguiendo aproximadamente el ciclo de la figura \ref{f:segundoppio04}. Dos moles de helio gaseoso, que puede ser considerado como gas ideal, son utilizados como la sustancia de trabajo en esta máquina, que alcanza una temperatura máxima de $\SI{327}{\celsius}$ cuando llega al equilibrio con la fuente de temperatura alta. El proceso $bc$ es isotérmico reversible manteniendo al gas en contacto con la fuente de temperatura, la presión en los estados $a$ y $c$ es $\SI{100}{\kilo\pascal}$, y en el estado $b$ es $\SI{300}{\kilo\pascal}$. La temperatura de la fuente fría es la misma que alcanza el gas cuando llega al estado \textit{a}. \textit{a}) ¿Cuánto calor entra en el gas y cuánto sale del gas en cada ciclo? \textit{b}) ¿Cuánto trabajo efectúa la máquina en cada ciclo? \textit{c}) ¿Qué eficiencia tiene esta máquina? \textit{d}) Calcule el porcentaje que representa la eficiencia de esta máquina respecto de la máxima eficiencia posible que puede lograrse con los depósitos caliente y frío que se usan en este ciclo ($e_\text{máquina}/e_\text{máximo}\times 100\%$). \textit{e}) Si se invierte el sentido del ciclo para transformarla en una máquina refrigeradora, ¿la máquina que se obtiene es posible? (Verificar que se cumplan ambos principios de la termodinámica.)
\end{Exercise}
\begin{Answer}
	\begin{minipage}[t]{.4\textwidth}
    \textit{a}) Entran $\SI{20.9}{\kilo\joule}$ y salen $\SI{16.6}{\kilo\joule}$\\ \textit{b}) $\SI{4.3}{\kilo\joule}$\\ \textit{c}) 21\%\\ \textit{d}) La eficiencia de esta máquina es 31\% del máximo posible.
  \end{minipage}
\end{Answer}
%
\begin{center}
  \begin{tikzpicture}[scale=0.9]
    \begin{axis}[
      ticks=none,
      axis x line=bottom,
      axis y line=left,
      xmin=0.01, xmax=0.12,
      ymin=50, ymax=350, 
      xlabel={$V$},
      ylabel={$p$},
      ];
    \addplot [color=blue, very thick] [samples= 180, domain=0.033:0.066]  {100};
    \addplot [latex-, color=blue, very thick] [samples= 180, domain=0.066:0.1]  {100};
    \addplot [-latex, color=blue, very thick] [samples= 180, domain=0.033:0.066]  {100*0.1/x};
    \addplot [color=blue, very thick] [samples= 180, domain=0.066:0.1]  {100*0.1/x};
    \draw [color=blue, very thick][-latex](0.033,100)--(0.033,200);
    \draw [color=blue, very thick](0.033,200)--(0.033,300);
    \draw (0.033,100) node [left] {$a$};
    \draw (0.033,300) node [left] {$b$};
    \draw (0.1,100) node [above right] {$c$};
    \fill [black](0.033,100) circle(2pt);
    \fill [black](0.033,300) circle(2pt);
    \fill [black](0.1,100) circle(2pt);
    \end{axis}
  \end{tikzpicture}
  \captionof{figure}{Problema \ref{p:segundoppio04}\label{f:segundoppio04}}
\end{center}
%
\begin{Exercise}\label{p:segundoppio03}
  \ifthenelse{\equal{\seleccionados}{true}}
  {\addToList{xyz-segundoppio}{\ExerciseHeaderNB}}{}
  El diagrama $p$-$V$ de la figura \ref{f:segundoppio03} muestra un ciclo de una máquina térmica que usa $\SI{0.250}{\mole}$ de un gas ideal para el cual el coeficiente de dilatación adiabática es $\gamma = 1.40$. La parte curva $ab$ del ciclo corresponde a un proceso adiabático. \textit{a}) ¿Cuánto calor absorbe este gas por ciclo, y en qué parte del ciclo ocurre? \textit{b}) ¿Cuánto calor cede este gas por ciclo, y en qué parte del ciclo ocurre? \textit{c}) ¿Cuánto trabajo realiza esta máquina en un ciclo? \textit{d}) ¿Cuál es la eficiencia térmica de la máquina?
\end{Exercise}
\begin{Answer}
	\begin{minipage}[t]{.4\textwidth}
    \textit{a}) $\SI{5480}{\joule}$\\ \textit{b}) $\SI{3720}{\joule}$\\ \textit{c}) $\SI{1760}{\joule}$\\ \textit{d}) 32\%
  \end{minipage}
\end{Answer}
%
\begin{center}
  \begin{tikzpicture}[scale=0.9]
    \begin{axis}[
      %ticks=none,
      axis x line=bottom,
      axis y line=left,
      xmin=0, xmax=10,
      ymin=0, ymax=15, 
      xlabel={$V$~[$\si{\cubic\metre}$]},
      ylabel={$p$~[atm]},
      xtick={2,9},
      xticklabels={0.002,0.009},
      ytick={1.5},
      yticklabels={1.5}
      %yticklabels={},
      %extra x ticks={0.04},
      %extra y ticks={2}
      ];
    \addplot [color=blue, very thick] [samples= 180, domain=2:5.5]  {1.5};
    \addplot [latex-, color=blue, very thick] [samples= 180, domain=5.5:9]  {1.5};
    \addplot [-latex, color=blue, very thick] [samples= 180, domain=2:4.5]  {1.5*(9/x)^1.5};
    \addplot [color=blue, very thick] [samples= 180, domain=4.5:9]  {1.5*(9/x)^1.5};
    \addplot[color = black, dashed, thick] coordinates {(2, 0) (2, 1.5) (0, 1.5)};
    \addplot[color = black, dashed, thick] coordinates {(9, 0) (9, 1.5)};
    \draw [color=blue, very thick][-latex](2,1.5)--(2,7.5);
    \draw [color=blue, very thick](2,7.5)--(2,14.3);
    \draw (2,14.3) node [left] {$a$};
    \draw (9,1.5) node [above right] {$b$};
    \draw (2,1.5) node [above left] {$c$};
    \fill [black](2,14.3) circle(2pt);
    \fill [black](9,1.5) circle(2pt);
    \fill [black](2,1.5) circle(2pt);
    \end{axis}
  \end{tikzpicture}
  \captionof{figure}{Problema \ref{p:segundoppio03}\label{f:segundoppio03}}
\end{center}
%
\begin{Exercise}
  Una planta generadora de energía eléctrica de $\SI{1000}{\mega\watt}$, alimentada con carbón, tiene una eficiencia térmica del 40\%. \textit{a}) ¿Cuál es la rapidez de suministro de calor a la planta? \textit{b}) La planta quema carbón de piedra (antracita), que tiene un calor de combustión de $\SI{2.65E7}{\joule/\kilogram}$. ¿Cuánto carbón consume la planta al día, si opera de manera continua? \textit{c}) El depósito frío hacia donde la planta cede calor es un río cercano. La temperatura del río es $\SI{18.0}{\celsius}$ antes de llegar a la planta de energía y $\SI{18.5}{\celsius}$ después de recibir el calor de desecho de la planta. Calcule el caudal del río en $\si{\cubic\metre/\second}$.
\end{Exercise}
\begin{Answer}
	\begin{minipage}[t]{.4\textwidth}
    \textit{a}) $\SI{2500}{\mega\watt}$\\ \textit{b}) $\SI{8150}{ton}$\\ \textit{c}) $\SI{720}{\cubic\metre/\second}$
  \end{minipage}
\end{Answer}
%
\begin{Exercise}\label{p:segundoppio07}
  La figura \ref{f:segundoppio07} muestra el ciclo de Stirling idealizado, donde el proceso $1 \rightarrow 2$ es una expansión isotérmica a alta temperatura ($T_\text{c}$) y el proceso $3 \rightarrow 4$ es una compresión isotérmica a baja temperatura ($T_\text{f}$). \textit{a}) En un motor Stirling, las transferencias de calor en $4 \rightarrow 1$ y $2 \rightarrow 3$ no implican fuentes de calor externas, sino que usan regeneración: la misma sustancia que transfiere calor al gas del interior del cilindro en el proceso $4 \rightarrow 1$ absorbe calor del gas en el proceso $2 \rightarrow 3$. Por lo tanto, los calores transferidos $Q_{41}$ y $Q_{23}$ no afectan a la eficiencia del motor. Explique esta afirmación demostrando que se cumple $Q_{41} = - Q_{23}$. \textit{b}) Deduzca la eficiencia de este ciclo en términos de las temperaturas $T_\text{c}$ y $T_\text{f}$, teniendo en cuenta que representa a un motor Stirling que utiliza regeneración.
\end{Exercise}
\begin{Answer}
	\begin{minipage}[t]{.4\textwidth}
    \textit{b}) $e = 1 - T_f/T_c$
  \end{minipage}
\end{Answer}
%
\begin{center}
  \begin{tikzpicture}[scale=0.9]
    \begin{axis}[
      % ticks=none,
      axis x line=bottom,
      axis y line=left,
      xmin=0, xmax=5,
      ymin=0, ymax=450, 
      xlabel={$V$},
      ylabel={$p$},
      xtick={2,4},
      xticklabels={$V_1$,$V_2$},
      ytick=\empty,
      yticklabels={}
      %extra x ticks={0.04},
      %extra y ticks={2}
      ];
    \addplot [-latex, color=red, very thick] [samples= 180, domain=2:3] {600/x};
    \addplot [color=red, very thick] [samples= 180, domain=3:4] {600/x};
    \addplot [color=red, dashed, thick] [samples= 180, domain=1.4:2] {600/x};
    \addplot [color=red, dashed, thick] [samples= 180, domain=4:5] {600/x};
    \addplot [latex-, color=blue, very thick] [samples= 180, domain=3:4] {300/x};
    \addplot [color=blue, very thick] [samples= 180, domain=2:3] {300/x};
    \addplot [color=blue, dashed, thick] [samples= 180, domain=0.7:2] {300/x};
    \addplot [color=blue, dashed, thick] [samples= 180, domain=4:5] {300/x};
    
    \addplot[-latex, color = black, very thick] coordinates {(2, 150) (2, 225)};
    \addplot[color = black, very thick] coordinates {(2, 225) (2, 300)};
    \addplot[-latex, color = black, very thick] coordinates {(4, 150) (4, 112)};
    \addplot[color = black, very thick] coordinates {(4, 112) (4, 75)};

    \addplot[color = black, dashed, thick] coordinates {(2, 0) (2, 150)};
    \addplot[color = black, dashed, thick] coordinates {(4, 0) (4, 75)};
    
    \draw (2,150) node [below left] {4};
    \draw (2,300) node [below left] {1};
    \draw (4,150) node [above right] {2};
    \draw (4,75) node [below right] {3};
    \fill [black](2,150) circle(2pt);
    \fill [black](2,300) circle(2pt);
    \fill [black](4,150) circle(2pt);
    \fill [black](4,75) circle(2pt);
    \end{axis}
  \end{tikzpicture}
  \captionof{figure}{Problema \ref{p:segundoppio07}\label{f:segundoppio07}}
\end{center}
%
\begin{Exercise}\label{p:segundoppio08}
  El ciclo de la figura \ref{f:segundoppio08} aproxima el funcionamiento de una máquina térmica constituida por $\SI{0.55}{\mole}$ de un gas ideal diatómico que intercambia calor con dos depósitos que mantienen sus temperaturas constantes. El gas comienza en el estado $a$ con un volumen de $\SI{2.3}{\liter}$ y a la temperatura del reservorio caliente, $\SI{520}{\kelvin}$. Se expande adiabáticamente hasta un volumen de $\SI{9.0}{\liter}$, llegando al estado $b$ cuya temperatura es la del reservorio frío, $\SI{300}{\kelvin}$. Luego se comprime isotérmicamente hasta el estado $c$ con un volumen de $\SI{1.5}{\liter}$. Por último evoluciona sobre la trayectoria rectilínea que une los estados $c$ y $a$ para completar el ciclo. Calcular la eficiencia térmica de esta máquina. 
\end{Exercise}
\begin{Answer}
	\begin{minipage}[t]{.4\textwidth}
    $0.254$
  \end{minipage}
\end{Answer}
%
\begin{center}
  \begin{tikzpicture}[scale=0.9]
    \begin{axis}[
      ticks=none,
      axis x line=bottom,
      axis y line=left,
      xmin=0, xmax=10, 
      ymin=0, ymax=12.5, 
      xlabel={$V$},
      ylabel={$p$},
      % xtick={2,9},
      % xticklabels={1.5,9},
      % ytick={1.5},
      % yticklabels={1.5}
      %yticklabels={},
      %extra x ticks={0.04},
      %extra y ticks={2}
      ];
    \addplot [-latex, color=blue, very thick] [samples= 180, domain=2.3:4.5]  {1.5*(9/x)^1.5};
    \addplot [color=blue, very thick] [samples= 180, domain=4.5:9]  {1.5*(9/x)^1.5};
    \addplot [latex-, color=blue, very thick] [samples= 180, domain=3:9]  {1.5*9/x};
    \addplot [color=blue, very thick] [samples= 180, domain=1.5:3]  {1.5*9/x};
    % \addplot[color = black, dashed, thick] coordinates {(2, 0) (2, 1.5) (0, 1.5)};
    % \addplot[color = black, dashed, thick] coordinates {(9, 0) (9, 1.5)};
    \draw [color=blue, very thick][-latex](1.5,9)--(2,10.63);
    \draw [color=blue, very thick](2,10.63)--(2.3,11.61);
    \draw (2.3,11.61) node [right] {$a$};
    \draw (9,1.5) node [above right] {$b$};
    \draw (1.5,9) node [left] {$c$};
    \fill [black](2.3,11.61) circle(2pt);
    \fill [black](9,1.5) circle(2pt);
    \fill [black](1.5,9) circle(2pt);
    \end{axis}
  \end{tikzpicture}
  \captionof{figure}{Problema \ref{p:segundoppio08}\label{f:segundoppio08}}
\end{center}
%

  %   \twocolumn[\colorsection{Problemas adicionales}]
\setcounter{figure}{0}
%
% \begin{Exercise}\label{p:termoadicionales01}
%   La figura \ref{f:termoadicionales01} es un esquema de una pared formada por dos planchas paralelas de $\SI{5}{\centi\metre}$ y $\SI{4}{\centi\metre}$ de grosor separadas por una intercámara (de aire) y con coeficientes de conductividad térmica de $\SI{209}{\watt.\metre^{-1}.\kelvin^{-1}}$ y $\SI{83.6}{\watt.\metre^{-1}.\kelvin^{-1}}$ respectivamente. Siendo $\SI{100}{\celsius}$ y $\SI{10}{\celsius}$ las temperaturas de las caras opuestas respectivas, determine la temperatura de la intercámara.
% \begin{Answer}
%   \begin{minipage}[t]{.4\textwidth}
%     $\SI{70}{\celsius}$
%   \end{minipage}
% \end{Answer}
% %
% \begin{figure}[ht!]
%   \begin{center}
%     \includegraphics[scale=1]{termodinamica/img/termo_transmision_intercamara.png}
%     \caption{Problema \ref{p:termoadicionales01}.}\label{f:termoadicionales01}
%   \end{center}
% \end{figure}
% %
% \begin{Exercise}\label{p:termoadicionales02}
% La figura \ref{f:termoadicionales02} muestra un corte longitudinal de una pared compuesta por $\SI{15}{\centi\metre}$ de hormigón y $\SI{3}{\centi\metre}$ de un revestimiento. La superficie libre del revestimiento se encuentra en contacto con aire a una temperatura $T_1 = \SI{20}{\celsius}$ y la superficie libre del hormigón con aire a $T_2 = \SI{0}{\celsius}$. La conductividad térmica de este hormigón es $\SI{0.65}{kcal.\metre^{-1}.h^{-1}.\celsius^{-1}}$. Se sabe que la cantidad de calor que se transmite a través de esta
% pared por unidad de tiempo y por unidad de área es $\SI{48}{kcal.h^{-1}.\metre^{-2}}$. \textit{a}) ¿Cuánto vale el coeficiente de conductividad térmica del revestimiento? \textit{b}) ¿Cuál es la temperatura en la superficie de contacto entre el revestimiento y el hormigón? \textit{c}) Resuelva nuevamente el problema pero ahora incluyendo convección a ambos lados de la pared. Considere que el coeficiente de convección del aire en el interior es $\SI{5}{\watt/(\square\metre.\kelvin)}$ y el del aire en el exterior es $\SI{12}{\watt/(\square\metre.\kelvin)}$.
% \textit{d}) Realice un gráfico de la temperatura como función de la posición dentro de la pared.
% \begin{Answer}
%   \begin{minipage}[t]{.4\textwidth}
%     \textit{a}) $\SI{0.162}{kcal.\metre^{-1}.h^{-1}.\celsius^{-1}}$\\ \textit{b}) $\SI{11.1}{\celsius}$\\ \textit{c})
%   \end{minipage}
% \end{Answer}
% %
% \begin{figure}[h!]
%   \begin{center}
%     \includegraphics[scale=1]{termodinamica/img/termo_transmision_hormigon.png}
%     \caption{Problema \ref{p:P902}.}\label{f:termoadicionales02}
%   \end{center}
% \end{figure}
%
\begin{Exercise}
  Se coloca una cantidad de gas en un cilindro metálico que tiene un émbolo móvil en un extremo. No hay fugas de gas del cilindro a medida que el pistón se mueve. La fuerza externa aplicada al pistón se puede variar para cambiar la presión del gas a medida que se mueve el pistón para cambiar el volumen del gas. Un manómetro unido a la pared interior del cilindro mide la presión del gas y se puede calcular el volumen del gas a partir de una medición de la posición del pistón en el cilindro. Se comienza con una presión de $\SI{1.0}{atm}$ y un volumen de gas de $\SI{3.0}{\liter}$. Manteniendo la  presión constante, se aumenta el volumen a $\SI{5.0}{\liter}$. Luego, manteniendo el volumen constante en $\SI{5.0}{\liter}$, se aumenta la presión a $\SI{3.0}{atm}$. A continuación, se disminuye la presión linealmente en función del volumen hasta que el volumen sea de $\SI{3.0}{\liter}$ y la presión sea de $\SI{2.0}{atm}$. Finalmente, se mantiene el volumen constante a $\SI{3.0}{\liter}$ y se disminuye la presión a $\SI{1.0}{atm}$, devolviendo el gas a su presión y volumen iniciales. Las paredes del cilindro son buenos conductores del calor, a través de las cuales se realizan los flujos de calor necesarios para estas transformaciones. A estas presiones relativamente altas, usted sospecha que la ecuación de gas ideal no se aplicará con mucha precisión. Usted no sabe qué gas está en el cilindro o si es monoatómico, diatómico o poliatómico. ¿Cuál es el calor neto para el gas durante este ciclo? ¿Hay flujo calor neto hacia adentro o hacia afuera del gas?
\end{Exercise}
\begin{Answer}
  \begin{minipage}[t]{.4\textwidth}
    $Q = \SI{-304}{\joule}$. Hay un flujo neto de calor hacia afuera del gas
  \end{minipage}
\end{Answer}
%
\begin{Exercise}
  En cierto proceso, un sistema desprende $\SI{0.215}{\mega\joule}$ de calor, al tiempo que se contrae bajo una presión externa constante de $\SI{0.95}{\mega\pascal}$. La energía interna del sistema es la misma al principio y al final del proceso. Calcule el cambio de volumen del sistema.
\end{Exercise}
\begin{Answer}
  \begin{minipage}[t]{.4\textwidth}
    $\Delta V = \SI{-0.226}{\cubic\metre}$
  \end{minipage}
\end{Answer}
%
\begin{Exercise}\label{p:termoadicionales07}
  Considere un ciclo Diesel idealizado como el que se muestra en la figura \ref{f:termoadicionales07}, que inicia  (punto \textit{a}) con aire a una temperatura $T_a$. El aire puede tratarse como gas ideal. Calcule la eficiencia si $T_a = \SI{300}{\kelvin}$, $T_c = \SI{950}{\kelvin}$, $\gamma = 1.4$ y $r = 15$. \textcolor{red}{AGREGAR PREGUNTA ESTILO COMO SE DEBE MODIFICAR PARA AUMENTAR SU RENDIMIENTO.}
\end{Exercise}
\begin{Answer}
  \begin{minipage}[t]{.4\textwidth}
    $0.657$
  \end{minipage}
\end{Answer}
%
\begin{center}
  \begin{tikzpicture}[scale=1]
    \begin{axis}[
      % ticks=none,
      axis x line=bottom,
      axis y line=left,
      xmin=0.5, xmax=5.5,
      ymin=0, ymax=650,
      xlabel={$V$},
      ylabel={$p$},
      xtick={1,5},
      xticklabels={$V_0$,$rV_0$},
      ytick=\empty,
      % yticklabels={}
      %extra x ticks={0.04},
      %extra y ticks={2}
      ];
    \addplot [latex-, color=blue, very thick] [samples= 180, domain=2.5:5] {600/x^1.4};
    \addplot [color=blue, very thick] [samples= 180, domain=1:2.5] {600/x^1.4};
    \addplot [-latex, color=blue, very thick] [samples= 180, domain=2:3.5] {1583.41/x^1.4};
    \addplot [color=blue, very thick] [samples= 180, domain=3.5:5] {1583.41/x^1.4};

    \addplot[-latex, color = blue, very thick] coordinates {(1,600) (1.5, 600)};
    \addplot[color = blue, very thick] coordinates {(1.5,600) (2, 600)};
    \addplot[-latex, color = blue, very thick] coordinates {(5,166.355) (5,110)};
    \addplot[color = blue, very thick] coordinates {(5,110) (5, 63.04)};

    \addplot[color = black, dashed, thick] coordinates {(1, 0) (1, 600)};
    \addplot[color = black, dashed, thick] coordinates {(5, 0) (5, 63)};

    \draw (1,600) node [above left] {$b$};
    \draw (2,600) node [above right] {$c$};
    \draw (5,166.4) node [above right] {$d$};
    \draw (5,63) node [below right] {$a$};
    \fill [black](1,600) circle(2pt);
    \fill [black](2,600) circle(2pt);
    \fill [black](5,166.4) circle(2pt);
    \fill [black](5,63) circle(2pt);
    \end{axis}
  \end{tikzpicture}
  \captionof{figure}{Problema \ref{p:termoadicionales07}\label{f:termoadicionales07}}
\end{center}
%
\begin{Exercise}
  {\color{red} ¿Eliminar?} Un recipiente rígido y perfectamente aislado tiene una membrana que divide su volumen en mitades. Un lado contiene un gas a una temperatura absoluta $T_0$ y presión $p_0$, mientras que la otra mitad está completamente vacía. De repente, se forma un pequeño orificio en la membrana, permitiendo que el gas se filtre hacia la otra mitad hasta que termina por ocupar el doble de su volumen original. En términos de $T_0$ y $p_0$, ¿cuál será la nueva temperatura y presión del gas cuando se distribuye equitativamente en ambas mitades del recipiente?
\end{Exercise}
%
\begin{Exercise}
  Si se conocen los estados inicial y final de un sistema y el cambio correspondiente de energía interna, ¿podría determinarse si dicho cambio se debió a trabajo o a transferencia de calor?
\end{Exercise}
%
\begin{Exercise}
  {\color{red} ¿Eliminar?} Usted sostiene un globo inflado sobre un conducto de aire caliente de su casa y observa que se expande lentamente. Después, usted lo aleja del conducto y lo deja enfriar a la temperatura ambiente. Durante la expansión, ¿cuál era mayor, el calor agregado al globo o el trabajo efectuado por el aire dentro de este? (Suponga que el aire es un gas ideal). Una vez que el globo  regresa a la temperatura ambiente, ¿cómo se compara el calor neto ganado o perdido por el aire dentro del globo con el trabajo neto efectuado sobre el aire circundante o con el trabajo realizado por este último?
\end{Exercise}
%
\begin{Exercise}
  {\color{red} Agregar precisión a la pregunta.} Suponga que trata de enfriar su cocina dejando abierta la puerta del refrigerador. ¿Qué sucede? ¿Por qué? ¿El resultado sería el mismo si se dejara abierta una hielera llena de hielo? Explique las diferencias, si las hay.
\end{Exercise}
%
\begin{Exercise}
  Las máquinas térmicas reales, como el motor de gasolina de un auto, siempre tienen fricción entre sus piezas móviles, aunque los lubricantes la reducen al mínimo. ¿Una máquina térmica con piezas totalmente sin fricción sería 100\% eficiente? ¿La respuesta depende de si la máquina realiza un ciclo de Carnot o no?
\end{Exercise}
%
\begin{Exercise}
  ¿La Tierra y el Sol están en equilibrio térmico? ¿Hay cambios de entropía asociados a la transmisión de energía del Sol a la Tierra?
\end{Exercise}
%
\begin{Exercise}
  ¿Qué eficiencia tendría una máquina de Carnot que opera con $T_\text{caliente} = T_\text{fría}$? ¿Y si  $T_\text{fría} = \SI{0}{\kelvin}$ y $T_\text{caliente}$ fuera cualquier temperatura mayor que $\SI{0}{\kelvin}$? Interprete sus respuestas.
\end{Exercise}
%
  \twocolumn[\colorsection{Preguntas para el análisis}]
\textit{En esta sección se requiere brindar respuestas argumentadas.}
\setcounter{figure}{0}
%
\begin{Exercise}
  Explique por qué no tendría sentido utilizar un termómetro de vidrio de tamaño normal, para medir la temperatura del agua caliente contenida en un dedal.
\end{Exercise}
%
\begin{Exercise}
  Si usted calienta el aire dentro de un recipiente rígido y sellado hasta que su temperatura en la escala Kelvin se duplique, la presión del aire en el recipiente también se duplica. ¿Esto es cierto si se duplica la temperatura Celsius del aire en el recipiente?
\end{Exercise}
%
\begin{Exercise}
  \textit{a}) Suponga que en el problema \ref{p:calorimetria01} no es posible despreciar la transferencia de calor del líquido al recipiente en ese experimento. ¿El resultado obtenido en dicho experimento, es mayor o menor que el calor específico promedio real del líquido? \textit{b}) Ahora suponga que se puede despreciar el intercambio de calor con el recipiente, pero no es posible despreciar el intercambio de calor con el entorno. ¿El resultado obtenido en dicho experimento, es mayor o menor que el calor específico promedio real del líquido?
\end{Exercise}
%
\begin{Exercise}
  Si en el problema \ref{p:transmision00} se tuviera en cuenta la convección del aire a ambos lados de la pared, ¿la corriente de calor sería mayor, menor o la misma que la calculada en dicho problema?
\end{Exercise}
%
\begin{Exercise}
  Se desea cubrir las paredes metálicas de un horno con un material aislante. ¿Cuáles de los siguientes valores cambian y cuáles se mantienen igual si en lugar de aplicar el revestimiento en el interior del horno se lo aplica sobre el exterior de las paredes?: \textit{i}) la potencia transmitida a través de las paredes; \textit{ii}) la temperatura de la superficie interior; \textit{iii}) la temperatura de la superficie exterior. ¿Sus respuestas son las mismas si se incluye o no convección en los cálculos?
\end{Exercise}
%
\begin{Exercise}\label{p:preguntastermo01}
  \ifthenelse{\equal{\seleccionados}{true}}
  {\addToList{xyz-preguntas}{\ExerciseHeaderNB}}{}
  En la figura \ref{f:preguntastermo01} se muestra la evolución de la temperatura en función del calor intercambiado con el entorno para una muestra de un líquido desconocido. Si $\text{C}_{\text{l}}$ es el calor específico de este líquido y $\text{C}_{\text{s}}$ es el calor específico de la misma sustancia en estado sólido, ¿cuál de las siguientes opciones es la única correcta?\\
\renewcommand{\arraystretch}{1.5}
  \begin{tabular}{p{2.5cm} p{2.5cm} p{2.5cm}}
     \textit{a}) $\text{C}_{\text{s}}=\text{C}_{\text{l}}$ & \textit{b}) $\text{C}_{\text{s}}=3\text{C}_{\text{l}}$ & \textit{c}) $\text{C}_{\text{s}}=\text{C}_{\text{l}}/3$ \\
     \textit{d}) $\text{C}_{\text{s}}=9\text{C}_{\text{l}}$ & e) $\text{C}_{\text{s}}=\text{C}_{\text{l}}/9$ \\
  \end{tabular} \\
\end{Exercise}
%
\begin{center}
    \begin{tikzpicture}[scale=0.8]
        \begin{axis}[
            ticks= none,
            every major x tick/.append style={thick,blue},
            clip=false,
            grid=both,
            minor x tick num=2,        
            minor y tick num=2,
            xmin=0, xmax=7,   
            ymin=0, ymax=6.5,
            %xtick  align=center,
            xlabel={$|Q|$},
            ylabel={$T$}
        ];
        \addplot [color=red, thick] [id=p36a,samples= 180, domain=0:1]  {5-3*x};
        \addplot [color=red, thick] [id=p36b,samples= 180, domain=1:3]  {2};
        \addplot [color=red, thick] [id=p36c,samples= 180, domain=3:7]  {2-1/3*(x-3)};
        \end{axis}
    \end{tikzpicture}
    \captionof{figure}{Pregunta \ref{p:preguntastermo01}\label{f:preguntastermo01}}
\end{center}
%
\begin{Exercise}
  ¿Es correcto afirmar que si se disminuye el área de un cuerpo, el calor intercambiado disminuye en la misma proporción independientemente del mecanismo considerado?
\end{Exercise}
%
\begin{Exercise}
  \textit{a}) Un bloque de metal frío se siente más frío que uno de madera a la misma temperatura. ¿Por qué? \textit{b}) Un bloque de metal caliente se siente más caliente que uno de madera a la misma temperatura. ¿Por qué? \textit{c}) ¿Hay alguna temperatura a la que ambos bloques se sientan igualmente calientes o fríos? ¿Cuál es esta?
\end{Exercise}
%
\begin{Exercise}
  En algunas situaciones suele considerarse que uno de los mecanismos de transmisión del calor es ``más importante'' que los otros. \textit{a}) Mencione un caso en que la conducción es el mecanismo primordial y de algunas razones para que esta aproximación pueda considerarse correcta. \textit{b}) Ídem \textit{a}) para convección. \textit{c}) Ídem \textit{a}) para radiación.
\end{Exercise}
%
\begin{Exercise}\label{p:preguntastermo02}
  \ifthenelse{\equal{\seleccionados}{true}}
  {\addToList{xyz-preguntas}{\ExerciseHeaderNB}}{}
  El gráfico de la figura \ref{f:preguntastermo02} representa la temperatura en función a la distancia a la fuente caliente dentro de una pared formada por dos materiales, el material $A$ con un espesor $l$ y el material $B$ con espesor $2l$. Para el gráfico en cuestión analice las siguientes afirmaciones y diga si son verdaderas o falsas y por qué: \textit{a}) La conductividad térmica de $A$ es mayor que la de $B$. \textit{b}) La temperatura de unión en la superficie de contacto antre $A$ y $B$ es menor que el promedio de las temperaturas en los lados de la pared. \textit{c}) El flujo calórico es mayor en el material de mayor conductividad.
\end{Exercise}
%
\begin{center}
  \begin{tikzpicture}[scale=0.9]
    \begin{axis}[
      %ticks=none,
      axis x line=bottom,
      axis y line=left,
      xmin=0, xmax=3.5,
      ymin=0, ymax=5,
      xlabel={Posición},
      ylabel={Temperatura},
      xtick={1,3},
      xticklabels={$l$,$3l$},
      ytick={},
      yticklabels={}
      ];
    \draw [color=blue, very thick](0,0.5)--(0,4.5);
    \draw [color=blue, very thick](1,0.5)--(1,4.5);
    \draw [color=blue, very thick](3,0.5)--(3,4.5);

    % No puedo usar foreach adentro de axis, y si lo uso afuera no
    % respeta la escala.
    \draw [color=blue!30](0,0.75)--(0.5,0.5);
    \draw [color=blue!30](0,1)--(1,0.5);
    \draw [color=blue!30](0,1.25)--(1,0.75);
    \draw [color=blue!30](0,1.5)--(1,1);
    \draw [color=blue!30](0,1.75)--(1,1.25);
    \draw [color=blue!30](0,2)--(1,1.5);
    \draw [color=blue!30](0,2.25)--(1,1.75);
    \draw [color=blue!30](0,2.5)--(1,2);
    \draw [color=blue!30](0,2.75)--(1,2.25);
    \draw [color=blue!30](0,3)--(1,2.5);
    \draw [color=blue!30](0,3.25)--(1,2.75);
    \draw [color=blue!30](0,3.5)--(1,3);
    \draw [color=blue!30](0,3.75)--(1,3.25);
    \draw [color=blue!30](0,4)--(1,3.5);
    \draw [color=blue!30](0,4.25)--(1,3.75);
    \draw [color=blue!30](0,4.5)--(1,4);
    \draw [color=blue!30](0.5,4.5)--(1,4.25);

    \draw [color=blue!30](2.875,4.35)--(3,4.25);
    \draw [color=blue!30](2.5625,4.35)--(3,4);
    \draw [color=blue!30](2.25,4.35)--(3,3.75);
    \draw [color=blue!30](1.9375,4.35)--(3,3.5);
    \draw [color=blue!30](1.625,4.35)--(3,3.25);
    \draw [color=blue!30](1.3125,4.35)--(3,3);
    \draw [color=blue!30](1,4.35)--(3,2.75);
    \draw [color=blue!30](1,4.10)--(3,2.5);
    \draw [color=blue!30](1,3.85)--(3,2.25);
    \draw [color=blue!30](1,3.60)--(3,2.0);
    \draw [color=blue!30](1,3.35)--(3,1.75);
    \draw [color=blue!30](1,3.10)--(3,1.5);
    \draw [color=blue!30](1,2.85)--(3,1.25);
    \draw [color=blue!30](1,2.6)--(3,1.);
    \draw [color=blue!30](1,2.35)--(3,0.75);
    \draw [color=blue!30](1,2.10)--(3,0.5);
    \draw [color=blue!30](1,1.85)--(2.6875,0.5);
    \draw [color=blue!30](1,1.6)--(2.375,0.5);
    \draw [color=blue!30](1,1.35)--(2.0625,0.5);
    \draw [color=blue!30](1,1.1)--(1.75,0.5);
    \draw [color=blue!30](1,0.85)--(1.4375,0.5);

    \draw [color=black, very thick](0,4)--(1,3.5);
    \draw [color=black, very thick](1,3.5)--(3,0.75);

    \draw (0.5,3) node [font=\fontsize{17}{0}] {$A$};
    \draw (2.5,3) node [font=\fontsize{17}{0}] {$B$};
    \end{axis}
    % \foreach \y in {1,...,5}
    %     \draw [color=blue!30](0,\y)--(1.8,\y-0.5);
  \end{tikzpicture}
  \captionof{figure}{Problema \ref{p:preguntastermo02}\label{f:preguntastermo02}}
\end{center}
%
\textbf{Se agregarán más preguntas próximamente.}
% \begin{Exercise}\label{p:preguntastermo03}
%   En la figura \ref{f:preguntastermo03} se muestran dos evoluciones de un mismo gas ideal. ¿Cómo se comparan el calor $Q_{a\rightarrow b \rightarrow c}$ a lo largo de la línea sólida con el calor $Q_{a\rightarrow c}$ a lo largo de la línea punteada?
% \end{Exercise}
% %
% \begin{center}
%   \begin{tikzpicture}[scale=0.9]
%     \begin{axis}[
%       ticks=none,
%       axis x line=bottom,
%       axis y line=left,
%       clip=false,
%       xmin=0, xmax=1.1,
%       ymin=0, ymax=4.5,
%       xtick  align=center,
%       xlabel={$V$},
%       ylabel={$p$}
%       ];
%       \draw [color=blue, very thick][-latex](0.2,1)--(0.4,2.5);
%       \draw [color=blue, very thick](0.4,2.5)--(0.6,4);
%       \draw [color=blue, very thick][-latex](0.6,4)--(0.8,2.5);
%       \draw [color=blue, very thick](0.8,2.5)--(1,1);
%       \draw [color=blue, dashed, very thick][-latex](0.2,1)--(0.6,1);
%       \draw [color=blue, dashed, very thick](0.6,1)--(1,1);
%       \draw (0.2,1) node [below left] {$a$};
%       \draw (0.6,4) node [above left] {$b$};
%       \draw (1,1) node [below right] {$c$};
%       \fill [black](0.2,1) circle(2pt);
%       \fill [black](0.6,4) circle(2pt);
%       \fill [black](1,1) circle(2pt);
%     \end{axis}
%   \end{tikzpicture}
%   \captionof{figure}{Problema \ref{p:preguntastermo03}\label{f:preguntastermo03}}
% \end{center}
% %
% \begin{Exercise}
%   \ifthenelse{\equal{\seleccionados}{true}}
%   {\addToList{xyz-preguntas}{\ExerciseHeaderNB}}{}
%   En un calorímetro ideal se mezclan una masa de agua a $\SI{70}{\celsius}$ y otra masa de agua a $\SI{20}{\celsius}$, y se espera hasta que el sistema (la mezcla) alcance el equilibrio térmico. Siendo que el sistema está aislado del entorno, ¿varía su entropía?
% \end{Exercise}
% %
% \begin{Exercise}
%   En los siguientes procesos, ¿el trabajo efectuado por el sistema (definido como un gas que se expande o se contrae) sobre el ambiente es positivo o negativo? \textit{a}) La expansión de una mezcla aire-gasolina quemada en el cilindro de un motor de automóvil; \textit{b}) abrir una botella de champaña; \textit{c}) llenar un tanque de buceo con aire comprimido; \textit{d}) la abolladura parcial de una botella de agua vacía y cerrada, al conducir descendiendo desde las montañas hacia el nivel del mar.
% \end{Exercise}
% %
% \begin{Exercise}
%   ¿En qué situación debe usted efectuar más trabajo: al inflar un globo al nivel del mar o al inflar el mismo globo con el mismo volumen en la cima del Aconcagua?
% \end{Exercise}
% %
% \begin{Exercise}
%   \ifthenelse{\equal{\seleccionados}{true}}
%   {\addToList{xyz-preguntas}{\ExerciseHeaderNB}}{}
%   Cuando se derrite hielo a $\SI{0}{\celsius}$ su volumen disminuye. ¿El cambio de energía interna es mayor, menor o igual que el calor agregado?
% \end{Exercise}
% %
% \begin{Exercise}
%   Un gas ideal se expande mientras que la presión se mantiene constante. Durante este proceso, ¿hay flujo de calor hacia el gas o hacia afuera de este?
% \end{Exercise}
% %
% \begin{Exercise}
%   En un proceso a volumen constante, $dU = nC_VdT$. En cambio, en un proceso a presión constante, no se cumple que $dU = nC_pdT$. ¿Por qué no?
% \end{Exercise}
% %
% \begin{Exercise}
%   Convertir energía mecánica totalmente en calor, ¿viola la segunda ley de la termodinámica? ¿Y convertir calor totalmente en trabajo?
% \end{Exercise}
% %
% \begin{Exercise}\label{p:preguntastermo04}
%   Un sistema termodinámico experimenta un proceso cíclico como se muestra en la figura \ref{f:preguntastermo04}. El ciclo consiste en dos lazos cerrados: el lazo \textit{I} y el \textit{II}. \textit{a}) Durante un ciclo completo, ¿el sistema efectúa trabajo neto positivo o negativo? \textit{b}) Durante un ciclo completo, ¿entra calor al sistema o sale de él? \textit{c}) En cada lazo, \textit{I} y \textit{II}, ¿entra calor en el sistema o sale de él?
% \end{Exercise}
% %
% % It needs \usetikzlibrary{decorations.markings}
% % and the tikzet defined in the preamble.
% \begin{center}
%   \begin{tikzpicture}[scale=0.9]
%     \draw [cyan] plot [smooth cycle, tension=1] coordinates { (0,0.5) (2,2.5) (3,-2) (4.5,-2.4)} [arrow inside={end=stealth,opt={red,scale=2}}{0,0.35,0.5,0.75}];
%     \draw [-latex] (-1,-3.5) -- (-1,3.5) node [left] {$p$};
%     \draw [-latex] (-1,-3.5) -- (5.5,-3.5) node [below] {$V$};
%     \draw (1.35,1) node [] {$I$};
%     \draw (4,-2.5) node [] {$II$};
%   \end{tikzpicture}
%   \captionof{figure}{Problema \ref{p:preguntastermo04}\label{f:preguntastermo04}}
% \end{center}
% %
% \begin{Exercise}
%   \ifthenelse{\equal{\seleccionados}{true}}
%   {\addToList{xyz-preguntas}{\ExerciseHeaderNB}}{}
%   Compare el diagrama $p$-$V$ para el ciclo Otto con el diagrama para la máquina térmica de Carnot. Explique algunas diferencias importantes entre los dos ciclos.
% \end{Exercise}
% %
% \begin{Exercise}
%   ¿Puede un sistema experimentar variaciones de entropía negativas?
% \end{Exercise}
% %
% \begin{Exercise}
%   ¿Un refrigerador lleno de alimentos con una temperatura ambiente de $\SI{20}{\celsius}$ consume más potencia si la temperatura es de $\SI{15}{\celsius}$? ¿O el consumo es el mismo?
% \end{Exercise}
% %
% \begin{Exercise}
%   Explique si en cada uno de los siguientes procesos hay aumentos de entropía o no: la mezcla de agua caliente y fría; expansión libre de un gas; flujo irreversible de calor; producción de calor por fricción mecánica.
% \end{Exercise}
% %
% \begin{Exercise}
%   La expansión libre de un gas ideal es un proceso adiabático, por lo que no hay transferencia de calor. Tampoco se realiza trabajo, de manera que la energía interna no cambia. Por lo tanto, $Q/T = 0$; sin embargo, la entropía es mayor después de la expansión. ¿Por qué la ecuación $\Delta S = \int dQ/T$ no se aplica a esta situación?
% \end{Exercise}
% %



%   \twocolumn[\colorsection{Calorimetría}]
\setcounter{figure}{0}
% %
% \begin{Exercise}
%   \textit{a}) La temperatura corporal normal promedio medida en la boca es $\SI{310}{\kelvin}$. ¿Cuál es la temperatura en grados Celsius y en Fahrenheit? \textit{b}) Durante un ejercicio muy vigoroso, la temperatura del cuerpo puede elevarse hasta $\SI{40}{\celsius}$. ¿Cuál es la temperatura en Kelvin y en grados Fahrenheit? \textit{c}) La temperatura de la superficie del cuerpo es aproximadamente $\SI{7}{\celsius}$ más baja que la temperatura interna. Exprese esta diferencia de temperatura en Kelvin y en grados Fahrenheit. \textit{d}) La sangre almacenada a $\SI{4}{\celsius}$ dura aproximadamente 3 semanas, mientras que la sangre almacenada a $\SI{-160}{\celsius}$ tiene una duración de 5 años. Exprese ambas temperaturas en las escalas Fahrenheit y Kelvin.
% \end{Exercise}
% \begin{Answer}
% 	\begin{minipage}[t]{.4\textwidth}
%     \textit{a}) $\SI{36.9}{\celsius}$ y $\SI{98.4}{^\circ F}$\\ \textit{b}) $\SI{313}{\kelvin}$ y $\SI{104}{^\circ F}$\\ \textit{c}) $\SI{7}{\kelvin}$ y $\SI{13}{^\circ F}$\\ \textit{d}) $\SI{4}{\celsius}$: $\SI{277}{\kelvin}$ y $\SI{39.2}{^\circ F}$; $\SI{-160}{\celsius}$: $\SI{113}{\kelvin}$ y $\SI{-256}{^\circ F}$
%   \end{minipage}
% \end{Answer}
%
\begin{Exercise}
  \textit{a}) Calcule la única temperatura a la que los termómetros Fahrenheit y Celsius coinciden. \textit{b}) Calcule la única temperatura a la que los termómetros Fahrenheit y Kelvin coinciden.
\end{Exercise}
\begin{Answer}
	\begin{minipage}[t]{.4\textwidth}
    \textit{a}) $\SI{-40}{\celsius}=\SI{-40}{^\circ F}$\\ \textit{b}) $\SI{575}{^\circ F}=\SI{575}{\kelvin}$
  \end{minipage}
\end{Answer}
%
\begin{Exercise}
  \textit{a}) Un termómetro de gas a volumen constante tiene una presión de $\SI{1000}{\pascal}$ a $\SI{15}{\celsius}$. Si la presión se incrementa a $\SI{2000}{\pascal}$, ¿cuál es la temperatura en grados Celsius? \textit{b}) Un termómetro de gas registra una presión absoluta de $\SI{325}{mmHg}$, estando en contacto con agua en el punto triple. ¿Qué presión indicará en contacto con agua en su punto de ebullición normal?
\end{Exercise}
\begin{Answer}
	\begin{minipage}[t]{.4\textwidth}
    \textit{a}) $\SI{303}{\celsius}$\\ \textit{b}) $\SI{444}{mmHg}$
  \end{minipage}
\end{Answer}
%
\begin{Exercise}
  Usando un termómetro de gas de volumen constante, un experimentador determinó que la presión del gas cuando el termómetro se encuentra a la temperatura del punto triple del agua ($\SI{0.01}{\celsius}$) es $\SI{4.8E4}{\pascal}$; y en el  punto de ebullición normal del agua ($\SI{100}{\celsius}$) es $\SI{6.5E4}{\pascal}$. \textit{a}) Suponiendo que la presión varía linealmente con la temperatura, use estos datos para calcular la temperatura Celsius en la que la presión del gas sería cero (es decir, obtenga la temperatura Celsius del cero absoluto). \textit{b}) ¿El gas de este termómetro obedece con precisión la ecuación $T_2/T_1=p_2/p_1$? Si es así y la presión a $\SI{100}{\celsius}$ fuera $\SI{6.5E4}{\pascal}$, ¿qué presión habría medido el experimentador a $\SI{0.01}{\celsius}$?
\end{Exercise}
\begin{Answer}
	\begin{minipage}[t]{.4\textwidth}
    \textit{a}) $\SI{-282.4}{\celsius}$\\ \textit{b}) $\SI{4.6E4}{\pascal}$
  \end{minipage}
\end{Answer}
%
\begin{Exercise}\label{p:calorimetria01}
  Un técnico mide el calor específico de un líquido desconocido sumergiendo en él una resistencia eléctrica. La energía eléctrica se convierte en calor transferido al líquido durante $\SI{120}{\second}$ a una tasa constante de $\SI{65.0}{\watt}$. La masa del líquido es $\SI{0.780}{\kilogram}$ y su temperatura aumenta de $\SI{18.55}{\celsius}$ a $\SI{22.54}{\celsius}$. Calcule el calor específico promedio del líquido en este intervalo de temperatura. Suponga que la cantidad de calor que se transfiere al recipiente es despreciable y que no se transfiere calor al entorno.
\end{Exercise}
\begin{Answer}
  $\SI{2.51E3}{\joule.\kilogram^{-1}.\kelvin^{-1}}$
\end{Answer}
%
\begin{Exercise}\label{p:calorimetria02}
  \ifthenelse{\equal{\seleccionados}{true}}
    {\addToList{xyz-calorimetria}{\ExerciseHeaderNB}}{}
  En un experimento se suministra calor a una muestra sólida de $\SI{500}{g}$ a una tasa de $\SI{10.0}{\kilo\joule/\minute}$ mientras se registra su temperatura en función del tiempo. La gráfica de sus datos se muestra en la figura \ref{f:calorimetria02}. \textit{a}) Calcule el calor latente de fusión del sólido. \textit{b}) Determine los calores específicos de los estados sólido y líquido del material.
\end{Exercise}
\begin{Answer}
	\begin{minipage}[t]{.4\textwidth}
    \textit{a}) $ L_f = \SI{3.00E4}{\joule/\kilogram}$\\ \textit{b}) $c_\text{sólido}=\SI{1.33E3}{\joule.\kilogram^{-1}.\kelvin^{-1}}$ y $c_\text{líquido}=\SI{1.00E3}{\joule.\kilogram^{-1}.\kelvin^{-1}}$
  \end{minipage}
\end{Answer}
%
\begin{center}
  \begin{tikzpicture}[scale=0.8]
      \begin{axis}[
                   every major x tick/.append style={thick,blue},
                   clip=false,
                   grid=both,
                   minor x tick num=3,        %un minor tick es decir 0.5
                   minor y tick num=3,
                   xmin=0, xmax=4,           %min y max para los ejes, NO PARA EL DOMINIO
                   ymin=-10, ymax=50,
                   %axis y line=center,        %alinea el eje al centro de la figura
                   %axis x line=middle,        %sino pone 2 ejes x
                   xtick  align=center,
                   xlabel={tiempo~[min]},
                   ylabel={temperatura~[$^\circ\text{C}$]}
                  ];
      \addplot [color=red, very thick] [id=s,samples= 180, domain=0:1]  {-5+15*x};
      \addplot [color=red, very thick] [id=s,samples= 180, domain=1:2.5]  {10};
      \addplot [color=red, very thick] [id=s,samples= 180, domain=2.5:4]  {10+20*(x-2.5)};
      \end{axis}
    \end{tikzpicture}
    \captionof{figure}{Problema \ref{p:calorimetria02}\label{f:calorimetria02}}
  \end{center}
%
\begin{Exercise}
  \ifthenelse{\equal{\seleccionados}{true}}
    {\addToList{xyz-calorimetria}{\ExerciseHeaderNB}}{}
  Una pieza metálica de $\SI{6.00}{\kilogram}$ de cobre sólido a una temperatura inicial $T$ se coloca con $\SI{2.00}{\kilogram}$ de hielo que se encuentran inicialmente a $\SI{-20.0}{\celsius}$. El hielo está en un contenedor aislado de masa despreciable. Después de que se alcanza el equilibrio térmico, se observan $\SI{1.20}{\kilogram}$ de hielo y $\SI{0.80}{\kilogram}$ de agua líquida. ¿Cuál era la temperatura inicial de la pieza de cobre?
\end{Exercise}
\begin{Answer}
  $T=\SI{150}{\celsius}$
\end{Answer}
%
\begin{Exercise}
  \ifthenelse{\equal{\seleccionados}{true}}
  {\addToList{xyz-calorimetria}{\ExerciseHeaderNB}}{}
  Una olla de cobre con una masa de $\SI{0.500}{\kilogram}$ contiene $\SI{0.170}{\kilogram}$ de agua, y ambas están a una temperatura de $\SI{20.0}{\celsius}$. Un bloque de $\SI{0.250}{\kilogram}$ de hierro a $\SI{85.0}{\celsius}$ se deja caer en la olla. Encuentre la temperatura final del sistema, suponiendo que no hay pérdida de calor a los alrededores.
\end{Exercise}
\begin{Answer}
  $\SI{27.5}{\celsius}$
\end{Answer}
%
\begin{Exercise}
  En un recipiente adiabático de masa despreciable, $\SI{0.2}{\kilogram}$ de hielo a una temperatura inicial de $\SI{-40}{\celsius}$ se mezclan con una masa $m$ de agua que tiene una temperatura inicial de $\SI{80}{\celsius}$. Si la temperatura final del sistema es $\SI{20}{\celsius}$, ¿cuál es la masa $m$ del agua que estaba inicialmente a $\SI{80}{\celsius}$?
\end{Exercise}
\begin{Answer}
  $m=\SI{0.4}{\kilogram}$
\end{Answer}
%
\begin{Exercise}
  El calor específico molar de cierta sustancia varía con la temperatura según la siguiente ecuación empírica: $C = a + bT$, donde $a=\SI{29.5}{\joule.\mole^{-1}.\kelvin^{-1}}$ y $b=\SI{8.20E-3}{\joule.\mole^{-1}.\kelvin^{-2}}$. ¿Cuánto calor se necesita para modificar la temperatura de $\SI{3.00}{\mole}$ de la sustancia de $\SI{27.0}{\celsius}$ a $\SI{227}{\celsius}$? (Sugerencia: Integre la ecuación $dQ = nCdT$.)
\end{Exercise}
\begin{Answer}
  $Q=\SI{19700}{\joule}$
\end{Answer}
%
\begin{Exercise}
  Un calorímetro de cobre cuya masa es $\SI{0.446}{\kilogram}$ contiene $\SI{0.095}{\kilogram}$ de hielo, y el sistema está inicialmente en equilibrio a $\SI{0}{\celsius}$. Si se agregan $\SI{0.035}{\kilogram}$ de vapor de agua a $\SI{100.0}{\celsius}$ y $\SI{1}{atm}$ de presión, \textit{a}) ¿qué temperatura final alcanzará el calorímetro y su contenido?, \textit{b}) ¿cuántos kilogramos habrá de hielo, de agua líquida y de vapor a dicha temperatura final?
\end{Exercise}
\begin{Answer}
	\begin{minipage}[t]{.4\textwidth}
    \textit{a}) $\SI{86.2}{\celsius}$\\ \textit{b}) sin hielo, sin vapor, $\SI{0.130}{\kilogram}$ de agua en estado líquido.
  \end{minipage}
\end{Answer}
%
\begin{Exercise}
  Un calorímetro cuyo equivalente en agua es $\SI{20}{\gram}$, contiene $\SI{100}{\gram}$ de agua a $\SI{20}{\celsius}$. Se agregan $\SI{50}{\gram}$ de una sustancia desconocida a una temperatura de $\SI{90}{\celsius}$, obteniéndose una temperatura final de equilibrio de $\SI{24}{\celsius}$. Calcular el calor específico de la sustancia desconocida.
\end{Exercise}
\begin{Answer}
  $\SI{0.145}{cal.\gram^{-1}.\celsius^{-1}}$
\end{Answer}
%
\begin{Exercise}
  \ifthenelse{\equal{\seleccionados}{true}}
  {\addToList{xyz-calorimetria}{\ExerciseHeaderNB}}{}
  Un calorímetro contiene $\SI{40}{\gram}$ de agua a $\SI{22}{\celsius}$ y se le agregan $\SI{50}{\gram}$ de agua a $\SI{50}{\celsius}$, obteniéndose una temperatura final de $\SI{35}{\celsius}$. \textit{a}) Calcular el equivalente en agua del calorímetro. \textit{b}) En un nuevo experimento, este mismo calorímetro contiene $\SI{100}{\gram}$ de agua a una temperatura de $\SI{22}{\celsius}$, y se agregan $\SI{80}{\gram}$ de aluminio a $\SI{90}{\celsius}$. Calcular la temperatura de equilibrio. \textit{Dato:} Calor específico del aluminio: $\SI{0.22}{cal.\gram^{-1}.\celsius^{-1}}$.
\end{Exercise}
\begin{Answer}
	\begin{minipage}[t]{.4\textwidth}
    \textit{a}) $\SI{17.7}{\gram}$\\ \textit{b}) $\SI{30.8}{\celsius}$
  \end{minipage}
\end{Answer}
%

%   \twocolumn[\colorsection{Transmisión del calor}]
\setcounter{figure}{0}
%
\begin{Exercise}
  \ifthenelse{\equal{\seleccionados}{true}}
  {\addToList{xyz-transmision}{\ExerciseHeaderNB}}{}
  Un extremo de una varilla metálica aislada se mantiene a $\SI{100}{\celsius}$, y el otro se mantiene a $\SI{0}{\celsius}$ con una mezcla de hielo y agua. La varilla tiene $\SI{60}{\centi\metre}$ de longitud y el área de su sección transversal es $\SI{1.25}{\square\centi\metre}$. El calor conducido por la varilla funde $\SI{8.5}{\gram}$ de hielo cada $\SI{10}{\minute}$. Calcule la conductividad térmica del metal.
\end{Exercise}
\begin{Answer}
  $\SI{227}{\watt.\metre^{-1}.\kelvin^{-1}}$
\end{Answer}
%
% \begin{Exercise}
%   A través de una ventana de vidrio de $\SI{1}{\square\metre}$ de área y $\SI{5}{\milli\metre}$ de espesor fluye calor a razón de $\SI{1600}{cal/\second}$, siendo la temperatura interior de $\SI{15}{\celsius}$ y la exterior de $\SI{25}{\celsius}$. Si la temperatura exterior aumenta a $\SI{35}{\celsius}$, ¿por cuál de las siguientes ventanas de área $A$ $\si{\square\metre}$ y espesor $w$ $\si{\milli\metre}$ la cambiaría para mantener el mismo flujo calórico?\\
% \begin{tabular*}{0.8\textwidth}{ccc}
%   \textit{a}) $A/w=0.40$ & \textit{b}) $A/w=0.13$ & \textit{c}) $A/w=0.20$\\
%   \textit{d}) $A/w=0.17$ & \textit{e}) $A/w=0.11$ & \textit{f}) $A/w=0.10$\\
% \end{tabular*}
% \end{Exercise}
% \begin{Answer}
%   Opción \textit{f})
% \end{Answer}
%
\begin{Exercise}
  Un método experimental para medir la conductividad térmica de un material aislante consiste en construir una caja del material y medir el aporte de potencia a un calentador eléctrico dentro de la caja, que mantiene el interior a una temperatura medida por encima de la temperatura de la superficie exterior. Suponga que en un aparato como el mencionado se requiere un aporte de potencia de $\SI{180}{\watt}$ para mantener la superficie interior de la caja $\SI{65.0}{\celsius}$ arriba de la temperatura de la superficie exterior. El área total de la caja es de $\SI{2.18}{\square\metre}$, y el espesor de la pared es de $\SI{3.90}{\centi\metre}$. Calcule la conductividad térmica del material en unidades del SI.
\end{Exercise}
\begin{Answer}
  $\SI{0.0495}{\watt.\metre^{-1}.\kelvin^{-1}}$
\end{Answer}
%
\begin{Exercise}\label{p:transmision00}
  \ifthenelse{\equal{\seleccionados}{true}}
  {\addToList{xyz-transmision}{\ExerciseHeaderNB}}{}
  En una casa se tiene una pared de ladrillos de $\SI{3}{\metre} \times \SI{4}{\metre}$, y espesor $\SI{15}{\centi\metre}$, que separa un ambiente a $\SI{25}{\celsius}$ del exterior a $\SI{5}{\celsius}$. Esta pared contiene una ventana que consiste en solo un panel de vidrio de $\SI{1.5}{\metre} \times \SI{1.5}{\metre} \times \SI{5}{\milli\metre}$. \textit{a}) Calcular la corriente de calor total a través del concreto y la ventana, sin incluir efectos de convección. \textit{b}) ¿Cuál es el porcentaje de calor que se pierde a través de la ventana respecto del total? \textit{c}) Repetir los cálculos para una situación donde la ventana es cubierta por una lámina de papel de espesor $\SI{0.750}{\milli\metre}$. \textit{Datos}: conductividad térmica del ladrillo = $\SI{0.6}{\watt.\metre^{-1}.\kelvin^{-1}}$; conductividad térmica del vidrio = $\SI{1}{\watt.\metre^{-1}.\kelvin^{-1}}$; conductividad térmica del papel = $\SI{0.050}{\watt.\metre^{-1}.\kelvin^{-1}}$.
\end{Exercise}
\begin{Answer}
	\begin{minipage}[t]{.4\textwidth}
    \textit{a}) $\SI{9780}{\watt}$\\ \textit{b}) 92\%\\ \textit{c}) $\SI{3030}{\watt}$ ($74.3$\%)
  \end{minipage}
\end{Answer}
%
%No se informan las conductividades, con la intención de que los estudiantes las busquen.
\begin{Exercise}
  \ifthenelse{\equal{\seleccionados}{true}}
  {\addToList{xyz-transmision}{\ExerciseHeaderNB}}{}
  Dos barras, una de latón y otra de cobre, están unidas extremo con extremo. La longitud de la barra de latón es $\SI{0.2}{\metre}$ y la de cobre es $\SI{0.8}{\metre}$. La sección transversal de cada segmento tiene un área de $\SI{0.005}{\square\metre}$. El extremo libre del segmento de latón está en contacto con agua hirviendo y el extremo libre del segmento de cobre se encuentra en contacto con una mezcla de hielo y agua, en ambos casos a presión atmosférica normal. Los lados de las varillas están aislados, por lo que no hay pérdida de calor a los alrededores. \textit{a}) ¿Cuál es la temperatura del punto en el que los segmentos de latón y de cobre se unen? \textit{b}) ¿Qué masa de hielo se funde en $\SI{5}{\minute}$ debido el calor conducido por la varilla compuesta?
\end{Exercise}
\begin{Answer}
	\begin{minipage}[t]{.4\textwidth}
    \textit{a}) $\SI{53.1}{\celsius}$; \textit{b}) $\SI{115}{\gram}$
  \end{minipage}
\end{Answer}
%
\begin{Exercise}\label{p:transmision02}
  El gráfico de la figura \ref{f:transmision02} representa la temperatura en función de la posición dentro de una barra cuya área transversal mide $\SI{3.00}{\centi\metre\squared}$ y su longitud total es $\SI{30}{\centi\metre}$. La barra está compuesta por dos materiales homogéneos, y conecta dos fuentes de calor a temperaturas constantes, transmitiendo $\SI{1.44}{cal/s}$. Determinar las conductividades térmicas de los materiales que forman la barra.
\end{Exercise}
\begin{Answer}
	\begin{minipage}[t]{.4\textwidth}
    $\SI{0.12}{cal/(cm.\celsius.s)}$ y $\SI{0.72}{cal/(cm.\celsius.s)}$
  \end{minipage}
\end{Answer}
%
\begin{center}
  \begin{tikzpicture}[scale=1]
    \begin{axis}[
      % every major x tick/.append style={thick,blue},
      clip=false,
      grid=both,
      minor x tick num=5,
      minor y tick num=3,
      xmin=0, xmax=30,
      ymin=0, ymax=100,
      xtick  align=center,
      xlabel={$x$~[cm]},
      ylabel={$T$~[$^\circ$C]}
      ];
      \draw [color=blue, very thick](0,90)--(15,30);
      \draw [color=blue, very thick](15,30)--(30,20);
    \end{axis}
  \end{tikzpicture}
  \captionof{figure}{Problema \ref{p:transmision02}\label{f:transmision02}}
\end{center}
%
\begin{Exercise}
  Se sueldan varillas de cobre, latón y acero para formar una figura en forma de Y. El área de la sección transversal de cada varilla es de $\SI{2}{\square\centi\metre}$. El extremo libre de la varilla de cobre se mantiene a $\SI{100}{\celsius}$, y los extremos libres de las varillas de latón y acero a $\SI{0}{\celsius}$. Suponga que no hay pérdida de calor por los laterales de las varillas, cuyas longitudes son: $\SI{13}{\centi\metre}$, $\SI{18}{\centi\metre}$ y $\SI{24}{\centi\metre}$ para la de cobre, latón y acero respectivamente. \textit{a}) ¿Qué temperatura tiene el punto de unión? \textit{b}) Calcule la corriente de calor en cada una de las tres varillas.
\end{Exercise}
\begin{Answer}
	\begin{minipage}[t]{.4\textwidth}
    \textit{a}) $\SI{78.4}{\celsius}$\\ \textit{b}) $H_\text{latón} = \SI{9.50}{\watt}$, $H_\text{acero} = \SI{3.28}{\watt}$ y $H_\text{cobre} = \SI{12.8}{\watt}$
  \end{minipage}
\end{Answer}
%
\begin{Exercise}
  Una pared de ladrillos de $\SI{20}{\centi\metre}$ de espesor y conductividad térmica $\SI{5E-4}{cal/(\second.\centi\metre.\celsius)}$, separa una habitación en la que el aire está a una temperatura de $\SI{15}{\celsius}$, del exterior donde el aire se encuentra a una temperatura de $\SI{-5}{\celsius}$. Si el coeficiente de convección interior es de $\SI{1E-4}{cal/(\second.\centi\square\metre.\celsius)}$ y el doble de  éste en el exterior, calcular: \textit{a}) La temperatura de la superficie interior de la pared. \textit{b}) La temperatura de la superficie exterior de la pared.
\end{Exercise}
\begin{Answer}
  \begin{minipage}[t]{.4\textwidth}
    \textit{a}) $\SI{11.4}{\celsius}$\\ \textit{b}) $\SI{-3.16}{\celsius}$
  \end{minipage}
\end{Answer}
%
\begin{Exercise}\label{p:transmision01}
  Una pared exterior está compuesta por una capa externa de madera de $\SI{3.0}{\centi\metre}$ de espesor y una capa interna de espuma de poliestireno de $\SI{2.2}{\centi\metre}$ de espesor. Considere que la conductividad térmica de la madera es $k_\text{m} = \SI{0.08}{\watt.\metre^{-1}.\kelvin^{-1}}$ y la del poliestireno es $k_\text{p} = \SI{0.01}{\watt.\metre^{-1}.\kelvin^{-1}}$. La temperatura del aire en el interior es $\SI{19}{\celsius}$ y la del aire en el exterior es $\SI{-10}{\celsius}$, y los coeficientes de convección del aire en el interior y del aire en el exterior valen $\SI{5}{\watt/(\square\metre.\kelvin)}$ y $\SI{12}{\watt/(\square\metre.\kelvin)}$ respectivamente. \textit{a}) Calcular la rapidez del flujo de calor por metro cuadrado a través de esta pared. \textit{b}) Calcular la temperatura en la superficie de contacto entre la madera y la espuma de poliestireno. \textit{c}) Realizar un gráfico de la temperatura en función de la posición, en la dirección del flujo de calor.
\end{Exercise}
\begin{Answer}
	\begin{minipage}[t]{.4\textwidth}
    \textit{a}) $\SI{10.2}{\watt/\square\metre}$\\ \textit{b}) $\SI{-5.36}{\celsius}$
  \end{minipage}
\end{Answer}
%
\begin{Exercise}
  Un carpintero construye una cabaña rústica que tiene un piso de dimensiones $\SI{3.50}{\metre} \times \SI{3.00}{\metre}$. Sus paredes, que miden $\SI{2.50}{\metre}$ de alto y $\SI{1.80}{\centi\metre}$ de grosor, están hechas de una madera cuya conductividad térmica vale $\SI{0.517}{cal/(\hour.\centi\metre.\celsius)}$, y serán aisladas con un material sintético de conductividad térmica igual a $\SI{0.947}{cal/(\hour.\centi\metre.\celsius)}$. Se desea instalar una estufa que entregue una potencia calorífica de $\SI{1100}{kcal/\hour}$ para mantener el interior a una temperatura de $\SI{25.0}{\celsius}$ cuando la temperatura exterior es $\SI{2.00}{\celsius}$. Despreciando la pérdida de calor a través del techo y del piso, calcule el espesor mínimo necesario del material aislante. Considere que el  coeficiente de convección del aire $\SI{2.00E-4}{cal/(\second.\centi\square\metre.\celsius)}$ tanto en el interior como en el exterior.
\end{Exercise}
\begin{Answer}
  $\SI{0.51}{\centi\metre}$
\end{Answer}
%
\begin{Exercise}
  En un edificio de oficinas se está considerando reemplazar ventanas de un solo panel de vidrio de $\SI{3}{\milli\metre}$ de espesor por ventanas de doble panel de vidrio de $\SI{3}{\milli\metre}$ de espesor, separados por $\SI{5}{\milli\metre}$ de aire estanco. En ambos casos la conductividad térmica del vidrio es $\SI{1}{\watt/(\metre.\kelvin)}$ y en el caso de doble panel la conductividad térmica del aire estanco es $\SI{0.025}{\watt/(\metre.\kelvin)}$. El coeficiente de transmisión interior y exterior vale $\SI{20}{\watt/(\metre\squared.\kelvin)}$ y se puede despreciar la contribución por convección en el aire estanco. Las oficinas se calefaccionan con energía eléctrica cuyo costo es $\SI{116.63}{\$/kWh}$, para mantener la temperatura interior $\SI{10}{\celsius}$ por encima de la temperatura exterior, durante 220 horas mensuales. ¿Cuánto dinero se ahorrarían cada mes por cada metro cuadrado de ventana que se reemplace?
\end{Exercise}
\begin{Answer}
	\begin{minipage}[t]{.4\textwidth}
    \$ 1650
  \end{minipage}
\end{Answer}
%
\begin{Exercise}
  Calcule la tasa de radiación de energía por unidad de área de un cuerpo negro a: \textit{a}) $\SI{273}{\kelvin}$ y \textit{b}) $\SI{2730}{\kelvin}$.
\end{Exercise}
\begin{Answer}
	\begin{minipage}[t]{.4\textwidth}
    \textit{a}) $\SI{315}{\watt/\square\metre}$\\ \textit{b}) $\SI{3.15E6}{\watt/\square\metre}$
  \end{minipage}
\end{Answer}
%
\begin{Exercise}
  \ifthenelse{\equal{\seleccionados}{true}}
  {\addToList{xyz-transmision}{\ExerciseHeaderNB}}{}
  La emisividad del tungsteno es $0.350$. Una esfera de tungsteno con un radio de $\SI{1.5}{\centi\metre}$ se suspende dentro de una cavidad grande, cuyas paredes están a $\SI{290}{\kelvin}$. ¿Qué aporte de potencia se requiere para mantener la esfera a una temperatura de $\SI{3000}{\kelvin}$, si se desprecia la conducción de calor por los soportes?
\end{Exercise}
\begin{Answer}
  $\SI{4540}{\watt}$
\end{Answer}
%
\begin{Exercise}
  Calcular cuánto calor neto pierde cada hora una persona desnuda en forma de radiación, cuando se encuentra en un ambiente a $\SI{15}{\celsius}$. Suponer que la superficie libre que emite (y recibe) calor es de $\SI{2.5}{\metre\squared}$, su temperatura es $\SI{33}{\celsius}$, y se comporta como un cuerpo negro.
\end{Exercise}
\begin{Answer}
	\begin{minipage}[t]{.4\textwidth}
    $\SI{963}{\kilo\joule}$
  \end{minipage}
\end{Answer}
%
\begin{Exercise}
  La tasa de energía radiante que llega del Sol a la atmósfera superior de la Tierra es cercana a $\SI{1.5}{\kilo\watt/\square\metre}$. La distancia promedio de la Tierra al Sol es $\SI{1.5E11}{\metre}$ y el radio del Sol es $\SI{6.96E8}{\metre}$. \textit{a}) Calcule la tasa de radiación de energía por unidad de área de la superficie solar. \textit{b}) Si el Sol irradia como cuerpo negro ideal, ¿qué temperatura tiene en su superficie?
\end{Exercise}
\begin{Answer}
	\begin{minipage}[t]{.4\textwidth}
    \textit{a}) $\approx\SI{70}{\mega\watt/\square\metre}$\\ \textit{b}) $\approx\SI{5900}{\kelvin}$
  \end{minipage}
\end{Answer}
%
%   \twocolumn[\colorsection{Primer principio de la termodinámica}]
\setcounter{figure}{0}
%
\begin{Exercise}
  Un tanque de $\SI{20.0}{\liter}$ contiene $\SI{4.86E-4}{\kilogram}$ de helio a $\SI{18.0}{\celsius}$. La masa molar del helio es $\SI{4.00}{\gram/\mole}$. \textit{a}) ¿Cuántos moles de helio hay en el tanque? \textit{b}) ¿Cuál es la presión en el tanque en pascales y en atmósferas?
\end{Exercise}
\begin{Answer}
	\begin{minipage}[t]{.4\textwidth}
    \textit{a}) $\SI{0.122}{\mole}$\\ \textit{b}) $\SI{14750}{\pascal}$ o $\SI{0.146}{atm}$
  \end{minipage}
\end{Answer}
%
\begin{Exercise}
  \ifthenelse{\equal{\seleccionados}{true}}
  {\addToList{xyz-primerppio}{\ExerciseHeaderNB}}{}
  Un tanque cilíndrico tiene un pistón ajustado que permite modificar el volumen del tanque. Originalmente, el tanque contiene $\SI{0.110}{\cubic\metre}$ de aire a $\SI{0.355}{atm}$ de presión. Se tira lentamente del pistón hasta aumentar el volumen del aire a $\SI{0.390}{\cubic\metre}$. Si la temperatura permanece constante, ¿qué valor final tiene la presión?
\end{Exercise}
\begin{Answer}
  $\SI{0.100}{atm}$
\end{Answer}
%
\begin{Exercise}
  Un matraz de $\SI{1.50}{\liter}$, provisto de una llave de paso, contiene etano gaseoso ($\text{C}_2\text{H}_6$) a $\SI{300}{\kelvin}$ y presión atmosférica ($\SI{101.3}{\kilo\pascal}$). La masa molar del etano es $\SI{30.1}{\gram/\mole}$. El sistema se calienta a $\SI{490}{\kelvin}$, con la llave abierta a la atmósfera. Luego se cierra la llave y el matraz se enfría a su temperatura  original. \textit{a}) Calcule la presión final del etano en el matraz. \textit{b}) ¿Cuántos gramos de etano quedan en el matraz?
\end{Exercise}
\begin{Answer}
	\begin{minipage}[t]{.4\textwidth}
    \textit{a}) $\SI{62}{\kilo\pascal}$\\ \textit{b}) $\SI{1.12}{\gram}$
  \end{minipage}
\end{Answer}
%
\begin{Exercise}
  Dos moles de gas ideal se calientan a presión constante desde $\SI{27.0}{\celsius}$ hasta $\SI{107}{\celsius}$. Calcule el trabajo efectuado por el gas.
\end{Exercise}
\begin{Answer}
  $\SI{1330}{\joule}$
\end{Answer}
%
\begin{Exercise}
  \ifthenelse{\equal{\seleccionados}{true}}
  {\addToList{xyz-primerppio}{\ExerciseHeaderNB}}{}
  Seis moles de gas ideal están en un cilindro provisto en un extremo con un pistón móvil. La temperatura inicial del gas es $\SI{27.0}{\celsius}$ y se desplaza el pistón manteniendo la presión del gas constante. Calcule la temperatura final del gas una vez que haya efectuado $\SI{2.4}{\kilo\joule}$ de trabajo.
\end{Exercise}
\begin{Answer}
  $\SI{75.1}{\celsius}$
\end{Answer}
%
\begin{Exercise}\label{p:primerppio01}
  La gráfica de la figura \ref{f:primerppio01} muestra un diagrama $p$-$V$ del aire en un pulmón cuando una persona inhala y luego exhala una respiración profunda. Estas gráficas, obtenidas en la práctica clínica, normalmente están algo curvadas, pero modelamos una como un conjunto de líneas rectas de la misma forma general. (Importante: La presión indicada es la presión manométrica, no la presión absoluta). \textit{a}) ¿Cuántos joules de trabajo neto realiza el pulmón de esta persona durante una respiración completa? \textit{b}) El proceso que aquí se representa es algo diferente de los que se han estudiado, ya que el cambio de presión se debe a los cambios en la cantidad de gas en el pulmón, y no a los cambios de temperatura. (Piense en su propia respiración, sus pulmones no se expanden porque se han calentado). Si la temperatura del aire en el pulmón permanece en un valor razonable de $\SI{20}{\celsius}$, ¿cuál es el número máximo de moles en el pulmón de esta persona durante una respiración?
\end{Exercise}
\begin{Answer}
	\begin{minipage}[t]{.4\textwidth}
    \textit{a}) $\SI{1}{\joule}$\\ \textit{b}) $\SI{0.06}{\mole}$
  \end{minipage}
\end{Answer}
%
\begin{center}
  \begin{tikzpicture}[scale=1]
    \begin{axis}[
                 every major x tick/.append style={thick,blue},
                 clip=false,
                 grid=both,
                 minor x tick num=3,        %un minor tick es decir 0.5
                 minor y tick num=3,
                 xmin=0, xmax=1.6,           %min y max para los ejes, NO PARA EL DOMINIO
                 ymin=0, ymax=13, 
                 %axis y line=center,        %alinea el eje al centro de la figura
                 %axis x line=middle,        %sino pone 2 ejes x
                 xtick  align=center,
                 xlabel={$V$~[L]},
                 ylabel={$p$~[mmHg]}         
                ];
    \addplot [color=blue, thick] [id=pulmon1,samples= 180, domain=0.1:0.4]  {1+80/3*(x-0.1)};
    \addplot [color=blue, thick] [id=pulmon2,samples= 180, domain=0.4:1.4]  {9+2*(x-0.4)};
    \addplot [color=blue, thick] [id=pulmon3,samples= 180, domain=1:1.4]  {2+90/4*(x-1)};
    \addplot [color=blue, thick] [id=pulmon4,samples= 180, domain=0.1:1]  {1+10/9*(x-0.1)};
    \draw [red, -{Stealth}] (0.15,4)--(0.25,20/3) node[midway,sloped,above] {inhalación};
    \draw [red, -{Stealth}] (0.6,10)--(0.85,10.5) node[midway,sloped,above] {inhalación};
    \draw [red, -{Stealth}] (1.15,7.25)--(1.05,5) node[midway,sloped,above] {exhalación};
    \draw [red, -{Stealth}] (0.7,2.33)--(0.4,2) node[midway,sloped,above] {exhalación};
    \end{axis}
  \end{tikzpicture}
  \captionof{figure}{Problema \ref{p:primerppio01}\label{f:primerppio01}}
\end{center}
%
\begin{Exercise}
  Durante el tiempo en que $\SI{0.305}{\mole}$ de un gas ideal experimentan una compresión isotérmica a  $\SI{22}{\celsius}$, su entorno efectúa $\SI{468}{\joule}$ de trabajo sobre él. \textit{a}) Si la presión final es $\SI{1.76}{atm}$, ¿cuál fue la presión inicial? \textit{b}) Realice una gráfica $p$-$V$ para este proceso.
\end{Exercise}
\begin{Answer}
	\begin{minipage}[t]{.4\textwidth}
    \textit{a}) $\SI{0.941}{atm}$
  \end{minipage}
\end{Answer}
%
\begin{Exercise}
  Cuando se hierve agua a una presión de $\SI{2.00}{atm}$, el calor de vaporización es $\SI{2.20}{\mega\joule/\kilogram}$ y la temperatura del punto de ebullición es $\SI{120}{\celsius}$. A esta presión, $\SI{1.00}{\kilogram}$ de agua ocupa un volumen de $\SI{1.00E-3}{\cubic\metre}$, y $\SI{1.00}{\kilogram}$ de vapor de agua ocupa un volumen de $\SI{0.824}{\cubic\metre}$. Calcule el incremento en la energía interna del agua cuando se forma $\SI{1.00}{\kilogram}$ de vapor de agua a esta temperatura.
\end{Exercise}
\begin{Answer}
	\begin{minipage}[t]{.4\textwidth}
    $\Delta U = \SI{2.03E6}{\joule}$ (es menor al calor recibido)
  \end{minipage}
\end{Answer}
%
\begin{Exercise}\label{p:primerppio02}
  \ifthenelse{\equal{\seleccionados}{true}}
  {\addToList{xyz-primerppio}{\ExerciseHeaderNB}}{}
  Considere el ciclo cerrado $a\rightarrow b \rightarrow c \rightarrow d \rightarrow a$ mostrado en la figura \ref{f:primerppio02}. \textit{a}) Encuentre una expresión para el trabajo total efectuado por el sistema en este proceso. \textit{b}) Encuentre una expresión para el trabajo total efectuado por el sistema si el  ciclo se recorre en sentido opuesto.
\end{Exercise}
\begin{Answer}
	\begin{minipage}[t]{.4\textwidth}
    \textit{a}) $W_{abcda} = (p_1-p_0)(V_1-V_0)$\\ \textit{b}) $W_{adcba} = -W_{abcda}$
  \end{minipage}
\end{Answer}
%
\begin{center}
  \begin{tikzpicture}[scale=1]
    \begin{axis}[
      axis x line=bottom,
      axis y line=left,
      xmin=0, xmax=6,           %min y max para los ejes, NO PARA EL DOMINIO
      ymin=0, ymax=9, 
      xlabel={$V$},
      ylabel={$p$},
      xtick={1,5},
      xticklabels={$V_0$,$V_1$},
      ytick={2,7},
      yticklabels={$p_0$,$p_1$}
      ]
    % \draw [color=blue, very thick][-latex](160pt,20pt)..controls(100pt,25pt) and (60pt,45pt)..(42pt,58pt);
    \draw [color=blue, very thick][-latex](5,2)--(3,2);
    \draw [color=blue, very thick](3,2)--(1,2);
    \draw [color=blue, very thick][-latex] (1,2)--(1,4.5);
    \draw [color=blue, very thick] (1,4.5)--(1,7);
    \draw [color=blue, very thick][-latex] (1,7)--(3,7);
    \draw [color=blue, very thick] (3,7)--(5,7);
    \draw [color=blue, very thick][-latex] (5,7)--(5,4.5);
    \draw [color=blue, very thick] (5,4.5)--(5,2);
    \draw [dashed, thick] (1,0)--(1,2);
    \draw [dashed, thick] (5,0)--(5,2);
    \draw [dashed, thick] (0,2)--(1,2);
    \draw [dashed, thick] (0,7)--(1,7);
    \filldraw(1,2)circle(2pt) (1,7)circle (2pt)(5,7)circle(2pt)(5,2)circle(2pt);
    \draw (1,2) node [below left] {$a$};
    \draw (1,7) node [above left] {$b$};
    \draw (5,7) node [above right] {$c$};
    \draw (5,2) node [below right] {$d$};
    \end{axis}
  \end{tikzpicture}
  \captionof{figure}{Problema \ref{p:primerppio02}\label{f:primerppio02}}
\end{center}
%
\begin{Exercise}\label{p:primerppio03}
  En la figura \ref{f:primerppio03} se muestra el diagrama $p$-$V$ del proceso $a\rightarrow b \rightarrow c$ que implica $\SI{0.0175}{\mole}$ de un gas ideal. \textit{a}) ¿Cuál fue la temperatura más baja que alcanzó el gas en  este proceso? ¿Dónde ocurrió? \textit{b}) ¿Cuánto trabajo realizó o recibió el gas en el proceso $a \rightarrow b$? \textit{c}) ¿Cuánto trabajo realizó o recibió el gas en el proceso $b \rightarrow c$? \textit{d}) Si se entregaron $\SI{215}{\joule}$ de calor al gas durante $a\rightarrow b \rightarrow c$, ¿cuántos de esos Joules se destinaron a la energía interna?
\end{Exercise}
\begin{Answer}
	\begin{minipage}[t]{.4\textwidth}
    \textit{a}) $\SI{278}{\kelvin}$ en el estado $a$\\ \textit{b}) $\SI{0}{\joule}$\\ \textit{c}) realizó $\SI{162}{\joule}$ de trabajo\\ \textit{d}) $\SI{53}{\joule}$
  \end{minipage}
\end{Answer}
%
\begin{center}
  \begin{tikzpicture}[scale=1]
    \begin{axis}[
      every major x tick/.append style={thick,blue},
      clip=false,
      grid=both,
      minor x tick num=3,
      minor y tick num=3,
      xmin=0, xmax=7,
      ymin=0, ymax=0.6,
      xtick  align=center,
      xlabel={$V$~[L]},
      ylabel={$p$~[atm]}
      ];
      \draw [color=blue, very thick][-latex](2,0.2)--(2,0.5);
      \draw [color=blue, very thick][-latex](2,0.5)--(6,0.3);
      \draw (2,0.2) node [below left] {$a$};
      \draw (2,0.5) node [above left] {$b$};
      \draw (6,0.3) node [right] {$c$};
      \fill [black](2,0.2) circle(2pt);
      \fill [black](2,0.5) circle(2pt);
      \fill [black](6,0.3) circle(2pt);
    \end{axis}
  \end{tikzpicture}
  \captionof{figure}{Problema \ref{p:primerppio03}\label{f:primerppio03}}
\end{center}
%
\begin{Exercise}\label{p:primerppio04}
  \ifthenelse{\equal{\seleccionados}{true}}
  {\addToList{xyz-primerppio}{\ExerciseHeaderNB}}{}
  El gráfico en la figura \ref{f:primerppio04} muestra un diagrama $p$-$V$ de $\SI{3.25}{\mole}$ de helio ideal gaseoso. La parte $c \rightarrow a$ de este proceso es isotérmica. \textit{a}) Calcule el calor intercambiado por el helio en los procesos $a \rightarrow b$, $b \rightarrow c$ y $c \rightarrow a$. \textit{b}) Calcule la variación de energía interna del He en esos mismos procesos.
\end{Exercise}
\begin{Answer}
	\begin{minipage}[t]{.4\textwidth}
    \textit{a}) $Q_{ab}=\SI{60.0}{\kilo\joule}$, $Q_{bc}=\SI{-36.0}{\kilo\joule}$ y $Q_{ca}=\SI{-11.1}{\kilo\joule}$\\ \textit{b}) $\Delta U_{ab}=\SI{36.0}{\kilo\joule}$, $\Delta U_{bc}=\SI{-36.0}{\kilo\joule}$ y $\Delta U_{ca}=\SI{0}{\kilo\joule}$
  \end{minipage}
\end{Answer}
%
\begin{center}
  \begin{tikzpicture}[scale=1]
    \begin{axis}[
      %ticks=none,
      axis x line=bottom,
      axis y line=left,
      xmin=0, xmax=5,           %min y max para los ejes, NO PARA EL DOMINIO
      ymin=1, ymax=9, 
      xlabel={$V$~[$\si{\cubic\metre}$]},
      ylabel={$p$~[$10^5 \si{\pascal}$]},
      xtick={1,4},
      xticklabels={0.010,0.040},
      ytick={2},
      yticklabels={2.0}
      %yticklabels={},
      %extra x ticks={0.04},
      %extra y ticks={2}
      ];
    \addplot [-latex, color=blue, very thick] [samples= 180, domain=1:2.5]  {8};
    \addplot [color=blue, very thick] [samples= 180, domain=2.5:4]  {8};
    \addplot [color=blue, very thick] [samples= 180, domain=1:2.5]  {8/x};
    \addplot [latex-, color=blue, very thick] [samples= 180, domain=2.5:4]  {8/x};
    \addplot[color = black, dashed, thick] coordinates {(4, 0) (4, 2) (0, 2)};
    \addplot[color = black, dashed, thick] coordinates {(1, 0) (1, 8)};
    \draw [color=blue, very thick][-latex](4,8)--(4,5);
    \draw [color=blue, very thick](4,5)--(4,2);
    \draw (1,8) node [above left] {$a$};
    \draw (4,8) node [above right] {$b$};
    \draw (4,2) node [right] {$c$};
    \fill [black](1,8) circle(2pt);
    \fill [black](4,8) circle(2pt);
    \fill [black](4,2) circle(2pt);
    \end{axis}
  \end{tikzpicture}
  \captionof{figure}{Problema \ref{p:primerppio04}\label{f:primerppio04}}
\end{center}
%
\begin{Exercise}\label{p:primerppio05}
  \textit{a}) Una tercera parte de un mol de gas He evoluciona a lo largo de la trayectoria $a \rightarrow b \rightarrow c$ representada por la línea continua en la figura \ref{f:primerppio05}. Suponga que el gas se puede tratar como ideal. ¿Cuánto calor intercambia gas? \textit{b}) Si, en vez de ello, el gas pasa  del estado $a$ al estado $c$ evolucionando a lo largo de la línea horizontal punteada en la figura, ¿cuánto calor intercambia el gas en este caso?
\end{Exercise}
\begin{Answer}
	\begin{minipage}[t]{.4\textwidth}
    \textit{a}) $\SI{3200}{\joule}$\\ \textit{b}) $\SI{2000}{\joule}$
  \end{minipage}
\end{Answer}
%
\begin{center}
  \begin{tikzpicture}[scale=1]
    \begin{axis}[
      every major x tick/.append style={thick,blue},
      clip=false,
      grid=both,
      minor x tick num=3,
      minor y tick num=3,
      xmin=0, xmax=1.1,
      ymin=0, ymax=4.5,
      xtick  align=center,
      xlabel={$V$~[$10^{-2} \si{\cubic\metre}$]},
      ylabel={$p$~[$10^5 \si{\pascal}$]}
      ];
      \draw [color=blue, very thick][-latex](0.2,1)--(0.4,2.5);
      \draw [color=blue, very thick](0.4,2.5)--(0.6,4);
      \draw [color=blue, very thick][-latex](0.6,4)--(0.8,2.5);
      \draw [color=blue, very thick](0.8,2.5)--(1,1);
      \draw [color=blue, dashed, very thick][-latex](0.2,1)--(0.6,1);
      \draw [color=blue, dashed, very thick](0.6,1)--(1,1);
      \draw (0.2,1) node [below left] {$a$};
      \draw (0.6,4) node [above left] {$b$};
      \draw (1,1) node [below right] {$c$};
      \fill [black](0.2,1) circle(2pt);
      \fill [black](0.6,4) circle(2pt);
      \fill [black](1,1) circle(2pt);
    \end{axis}
  \end{tikzpicture}
  \captionof{figure}{Problema \ref{p:primerppio05}\label{f:primerppio05}}
\end{center}
%
\begin{Exercise}
  \ifthenelse{\equal{\seleccionados}{true}}
  {\addToList{xyz-primerppio}{\ExerciseHeaderNB}}{}
  Se colocan $\SI{0.20}{\mole}$ de un gas ideal diatómico en un recipiente a $\SI{3.0}{atm}$ y $\SI{500}{\kelvin}$. Se efectúa con el gas el siguiente ciclo: \textit{i}) Se expande isotérmicamente desde el estado \textit{A} hasta duplicar su volumen llegando al estado \textit{B}. \textit{ii}) A volumen constante se reduce su temperatura hasta $\SI{300}{\kelvin}$, alcanzando el estado \textit{C}. \textit{iii}) A presión constante se reduce su volumen hasta el volumen inicial (estado \textit{D}). \textit{iv}) El gas aumenta su temperatura a volumen constante hasta el estado \textit{A}. Para este ciclo se pide: \textit{a}) Realizar un diagrama $p-V$ indicando en qué procesos el gas realiza o recibe trabajo y en cuáles absorbe o cede calor. \textit{b}) Calcular el trabajo neto realizado por el gas. \textit{c}) Calcular el cociente entre el trabajo realizado neto y el calor absorbido por el gas.
\end{Exercise}
\begin{Answer}
	\begin{minipage}[t]{.4\textwidth}
    \textit{b}) $W = \SI{327}{\joule}$\\ \textit{c}) $0.16$
  \end{minipage}
\end{Answer}
%
\begin{Exercise}
  Un mol de un gas ideal diatómico se comprime lentamente a un tercio de su volumen original. En esta compresión, la magnitud del trabajo realizado sobre el gas es $\SI{600}{\joule}$. \textit{a}) Si el proceso es isotérmico, ¿cuál es el valor del calor $Q$ para el gas? ¿El flujo de calor es hacia adentro o hacia afuera del gas? \textit{b}) Si el proceso es isobárico, ¿cuál es el cambio en la energía interna del gas? ¿Aumenta o disminuye su energía interna?
\end{Exercise}
\begin{Answer}
	\begin{minipage}[t]{.4\textwidth}
    \textit{a}) $Q = \SI{-600}{\joule}$, el gas cede calor\\ \textit{b}) $\Delta U = \SI{-1500}{\joule}$, disminuye
  \end{minipage}
\end{Answer}
%
\begin{Exercise}
  Un cilindro con pistón contiene $\SI{0.15}{\mole}$ de nitrógeno a $\SI{0.18}{\mega\pascal}$ y $\SI{300}{\kelvin}$, que se puede tratar como un gas ideal. Primero, el gas se comprime isobáricamente a la mitad de su volumen original. Luego se expande adiabáticamente hasta su volumen original. Por último, se calienta isocóricamente hasta su presión original. Calcule el trabajo del gas en este ciclo.
\end{Exercise}
\begin{Answer}
	\begin{minipage}[t]{.4\textwidth}
    $W = \SI{-74.4}{\joule}$
  \end{minipage}
\end{Answer}
%

%   \twocolumn[\colorsection{Segundo principio de la termodinámica}]
\setcounter{figure}{0}
%
\begin{Exercise}
  En un calorímetro ideal (adiabático y de capacidad calorífica despreciable) se introducen $\SI{50}{\gram}$ de agua a $\SI{15}{\celsius}$ y $\SI{50}{\gram}$ de agua a $\SI{5}{\celsius}$, y se espera hasta que el sistema (la mezcla) alcance el equilibrio térmico. \textit{a}) Calcule la variación de entropía del sistema. \textit{b}) ¿Cuál es la variación de entropía del Universo en este proceso?
\end{Exercise}
\begin{Answer}
	\begin{minipage}[t]{.4\textwidth}
    \textit{a}) $\SI{0.016}{cal/\kelvin}$\\ \textit{b}) $\SI{0.016}{cal/\kelvin}$
  \end{minipage}
\end{Answer}
%
\begin{Exercise}
  Un bloque de hielo de $\SI{4.5}{\kilogram}$ a $\SI{0}{\celsius}$ cae en el océano y se funde. La temperatura media del océano es $\SI{3.5}{\celsius}$, incluyendo las aguas profundas. ¿En qué medida la fusión de este hielo cambia la entropía del mundo? ¿La aumenta o la disminuye? (La variación de la temperatura del océano mientras este bloque de hielo se funde es despreciable.)
\end{Exercise}
\begin{Answer}
	\begin{minipage}[t]{.4\textwidth}
    $\Delta S = \SI{71.3}{\joule/\kelvin}$
  \end{minipage}
\end{Answer}
%
\begin{Exercise}
  Un motor diesel realiza $\SI{2200}{\joule}$ de trabajo mecánico y cede $\SI{4300}{\joule}$ de calor en cada ciclo. \textit{a}) ¿Cuánto calor debe suministrarse al motor en cada ciclo? \textit{b}) Calcule la eficiencia térmica del motor.
\end{Exercise}
\begin{Answer}
	\begin{minipage}[t]{.4\textwidth}
    \textit{a}) $\SI{6500}{\joule}$\\ \textit{b}) 34\%
  \end{minipage}
\end{Answer}
%
\begin{Exercise}\label{p:segundoppio01}
  \ifthenelse{\equal{\seleccionados}{true}}
  {\addToList{xyz-segundoppio}{\ExerciseHeaderNB}}{}
  La figura \ref{f:segundoppio01} muestra un esquema de una máquina térmica cuyo depósito de alta temperatura está a una temperatura $T_c = \SI{620}{\kelvin}$, recibe $Q_\text{abs} = \SI{550}{\joule}$ de calor a esta temperatura en cada ciclo y cede $Q_\text{ced} = \SI{-395}{\joule}$ al depósito frío, a una temperatura $T_f = \SI{378}{\kelvin}$. \textit{a}) ¿Cuánto trabajo mecánico realiza la máquina en cada ciclo? \textit{b}) ¿Cuál es la eficiencia térmica de la máquina? \textit{c}) ¿Se trata de una máquina ideal o real?
\end{Exercise}
\begin{Answer}
	\begin{minipage}[t]{.4\textwidth}
    \textit{a}) $W = \SI{155}{\joule}$\\ \textit{b}) $e = 0.282$\\ \textit{c}) la máquina es real
  \end{minipage}
\end{Answer}
%
\begin{center}
  \begin{tikzpicture}[scale=1]
      
    % It needs:
    % \documentclass[dvipsnames]{article}
    % \usepackage{xcolor}
    % \usetikzlibrary{arrows.meta}
    
    \draw [rounded corners, thick, fill=pink] (0,4) rectangle (3,5);
    \draw [rounded corners, thick, fill=cyan] (0,0) rectangle (3,1);
    \draw [thick, fill=LimeGreen] (1.5,2.5) circle [radius=0.5];
    \draw (1.5,4.5) node {$T_c$};
    \draw (1.5,0.5) node {$T_f$};
    \draw (1.5,2.5) node {gas};

    \draw [color=pink] [-{Triangle[width=25pt,length=8pt]}, line width=18pt](1.5,3.95) -- (1.5, 3);
    \draw (1.5,3.5) node {$Q_c$};
    \draw [color=cyan] [-{Triangle[width=25pt,length=8pt]}, line width=18pt](1.5,1.97) -- (1.5, 1);
    \draw (1.5,1.5) node {$Q_f$};
    \draw [color=LimeGreen] [-{Triangle[width=25pt,length=8pt]}, line width=18pt](2.02,2.5) -- (3.5, 2.5);
    \draw (2.75,2.5) node {$W$};

  \end{tikzpicture}
  \captionof{figure}{Problema \ref{p:segundoppio01}\label{f:segundoppio01}}
\end{center}
%
\begin{Exercise}\label{p:segundoppio02}
  En la tabla \ref{t:segundoppio02} se muestra el calor absorbido, el calor cedido (en valor absoluto) y el trabajo realizado en cada ciclo de cuatro máquinas térmicas diseñadas para funcionar entre dos reservorios a temperaturas constantes de $\SI{800}{\kelvin}$ y $\SI{400}{\kelvin}$. ¿Cuáles de estas máquinas son posibles, debido a que no violan ningún principio termodinámico?\\
  \begin{table}[h]
    \caption{Ejercicio \ref{p:segundoppio02}}\label{t:segundoppio02}
    \centering
    \begin{tabular}{cccc}
        \hline
        \textbf{Máquina} & \textbf{$Q$ abs. [J]} & \textbf{$Q$ ced. [J]} & \textbf{$W$ [J]}\\
        \hline
        \textit{a} & 1000 & 600 & 600\\
        \textit{b} & 1000 & 400 & 600\\
        \textit{c} & 1000 & 500 & 500\\
        \textit{d} & 1000 & 700 & 300\\
        \hline
    \end{tabular}
  \end{table}
\end{Exercise}
\begin{Answer}
	\begin{minipage}[t]{.4\textwidth}
    Máquinas \textit{c} y \textit{d}
  \end{minipage}
\end{Answer}
%
\begin{Exercise}
  Un refrigerador tiene un coeficiente de rendimiento de $2.10$. Durante cada ciclo, absorbe $\SI{3.4E4}{\joule}$ de calor del depósito frío. \textit{a}) ¿Cuánta energía mecánica se requiere en cada ciclo para operar el refrigerador? \textit{b}) Durante cada ciclo, ¿cuánto calor se desecha al depósito caliente?
\end{Exercise}
\begin{Answer}
	\begin{minipage}[t]{.4\textwidth}
    \textit{a}) $\SI{1.62E4}{\joule}$\\ \textit{b}) $\SI{5.02E4}{\joule}$
  \end{minipage}
\end{Answer}
%
\begin{Exercise}
  Un acondicionador de aire tiene un coeficiente de rendimiento de $2.9$ en un día caluroso, y utiliza $\SI{850}{\watt}$ de potencia eléctrica. \textit{a}) ¿Cuántos Joules de calor extrae el sistema de aire acondicionado de la habitación en cada minuto? \textit{b}) ¿Cuántos Joules de calor entrega el sistema de aire acondicionado al aire caliente del exterior en cada minuto?
\end{Exercise}
\begin{Answer}
	\begin{minipage}[t]{.4\textwidth}
    \textit{a}) $\SI{148}{\kilo\joule}$\\ \textit{b}) $\SI{199}{\kilo\joule}$
  \end{minipage}
\end{Answer}
%
\begin{Exercise}
  \ifthenelse{\equal{\seleccionados}{true}}
  {\addToList{xyz-segundoppio}{\ExerciseHeaderNB}}{}
  Una máquina de Carnot opera entre $\SI{500}{\celsius}$ y $\SI{100}{\celsius}$ con un suministro de calor de $\SI{250}{\joule}$ por ciclo. \textit{a}) ¿Cuánto calor se entrega al depósito frío en cada ciclo? \textit{b}) ¿Qué número mínimo de ciclos se requieren para que la máquina levante una piedra de $\SI{500}{\kilogram}$ a una altura de $\SI{100}{\metre}$?
\end{Exercise}
\begin{Answer}
	\begin{minipage}[t]{.4\textwidth}
    \textit{a}) $\SI{121}{\joule}$\\ \textit{b}) $\SI{3.79E3}{ciclos}$
  \end{minipage}
\end{Answer}
%
\begin{Exercise}
  Una máquina de Carnot tiene una eficiencia del 59\% y realiza $\SI{25}{\kilo\joule}$ de trabajo en cada ciclo. \textit{a}) ¿Cuánto calor extrae la máquina de su fuente de calor en cada ciclo? \textit{b}) Suponga que la máquina expulsa calor a una temperatura ambiente de $\SI{20}{\celsius}$. ¿Cuál es la temperatura de su fuente de calor?
\end{Exercise}
\begin{Answer}
  \begin{minipage}[t]{.4\textwidth}
    \textit{a}) $\SI{42.37}{\kilo\joule}$; \textit{b}) $\SI{442}{\celsius}$
  \end{minipage}
\end{Answer}
%
\begin{Exercise}\label{p:segundoppio05}
  Calcule la eficiencia térmica de una máquina en la que $n$ moles de un gas ideal diatómico realizan el ciclo $1 \rightarrow 2 \rightarrow 3 \rightarrow 4 \rightarrow 1$ que se muestra en la figura \ref{f:segundoppio05}.
\end{Exercise}
\begin{Answer}
  \begin{minipage}[t]{.4\textwidth}
    $e = 2/19$
  \end{minipage}
\end{Answer}
%
\begin{center}
  \begin{tikzpicture}[scale=0.9]
    \begin{axis}[
      % ticks=none,
      axis x line=bottom,
      axis y line=left,
      xmin=0.5, xmax=2.5,
      ymin=0.5, ymax=2.5, 
      xlabel={$V$},
      ylabel={$p$},
      xtick={1,2},
      xticklabels={$V_0$,$2V_0$},
      ytick={1,2},
      yticklabels={$p_0$,$2p_0$}
      ];
    \addplot[color = black, dashed, thick] coordinates {(0, 2) (1, 2)};
    \addplot[color = black, dashed, thick] coordinates {(0, 1) (1, 1)};
    \addplot[color = black, dashed, thick] coordinates {(1, 0) (1, 1)};
    \addplot[color = black, dashed, thick] coordinates {(2, 0) (2, 1)};
    \draw [color=blue, very thick][-latex](1,2)--(1.5,2);
    \draw [color=blue, very thick](1.5,2)--(2,2);
    \draw [color=blue, very thick][-latex](2,2)--(2,1.5);
    \draw [color=blue, very thick](2,1.5)--(2,1);
    \draw [color=blue, very thick][-latex](2,1)--(1.5,1);
    \draw [color=blue, very thick](1.5,1)--(1,1);
    \draw [color=blue, very thick][-latex](1,1)--(1,1.5);
    \draw [color=blue, very thick](1,1.5)--(1,2);
    \draw (1,1) node [below left] {1};
    \draw (1,2) node [above left] {2};
    \draw (2,2) node [above right] {3};
    \draw (2,1) node [below right] {4};
    \fill [black](1,1) circle(2pt);
    \fill [black](1,2) circle(2pt);
    \fill [black](2,2) circle(2pt);
    \fill [black](2,1) circle(2pt);
    \end{axis}
  \end{tikzpicture}
  \captionof{figure}{Problema \ref{p:segundoppio05}\label{f:segundoppio05}}
\end{center}
%
\begin{Exercise}\label{p:segundoppio06}
  Calcule la eficiencia térmica de una máquina en la que $n$ moles de un gas ideal diatómico realizan el ciclo mostrado en la figura \ref{f:segundoppio06}.
\end{Exercise}
\begin{Answer}
	\begin{minipage}[t]{.4\textwidth}
    $e = \ln{(3)}/(2\ln{(3)}+7/2)$
  \end{minipage}
\end{Answer}
%
\begin{center}
  \begin{tikzpicture}[scale=0.9]
    \begin{axis}[
      % ticks=none,
      axis x line=bottom,
      axis y line=left,
      xmin=0.5, xmax=2.5,
      ymin=0.5, ymax=3.5, 
      xlabel={$T$},
      ylabel={$p$},
      xtick={1,2},
      xticklabels={$T_0$,$2T_0$},
      ytick={1,3},
      yticklabels={$p_0$,$3p_0$}
      %extra x ticks={0.04},
      %extra y ticks={2}
      ];
    \addplot[color = black, dashed, thick] coordinates {(0, 3) (1, 3)};
    \addplot[color = black, dashed, thick] coordinates {(0, 1) (1, 1)};
    \addplot[color = black, dashed, thick] coordinates {(1, 0) (1, 1)};
    \addplot[color = black, dashed, thick] coordinates {(2, 0) (2, 1)};
    % \addplot[color = black, dashed, thick] coordinates {(9, 0) (9, 1.5)};
    \draw [color=blue, very thick][-latex](1,3)--(1.5,3);
    \draw [color=blue, very thick](1.5,3)--(2,3);
    \draw [color=blue, very thick][-latex](2,3)--(2,2);
    \draw [color=blue, very thick](2,2)--(2,1);
    \draw [color=blue, very thick][-latex](2,1)--(1.5,1);
    \draw [color=blue, very thick](1.5,1)--(1,1);
    \draw [color=blue, very thick][-latex](1,1)--(1,2);
    \draw [color=blue, very thick](1,2)--(1,3);
    \fill [black](1,1) circle(2pt);
    \fill [black](1,3) circle(2pt);
    \fill [black](2,3) circle(2pt);
    \fill [black](2,1) circle(2pt);
    \end{axis}
  \end{tikzpicture}
  \captionof{figure}{Problema \ref{p:segundoppio06}\label{f:segundoppio06}}
\end{center}
%
\begin{Exercise}
  Un cilindro contiene oxígeno a una presión de $\SI{2.00}{atm}$, en un volumen de $\SI{4.00}{\liter}$ y a una temperatura de $\SI{300}{\kelvin}$. Suponga que el oxígeno se puede tratar como gas ideal mientras se somete a los siguientes procesos: \textit{i}) Calentamiento a presión constante desde el estado inicial (estado 1) al estado 2, donde $T = \SI{450}{\kelvin}$. \textit{ii}) Enfriamiento a volumen constante hasta $\SI{250}{\kelvin}$ (estado 3). \textit{iii}) Compresión a temperatura constante hasta un volumen de $\SI{4.00}{\liter}$ (estado 4). \textit{iv}) Calentamiento a volumen constante hasta $\SI{300}{\kelvin}$, regresando el sistema al estado 1. Determine la eficiencia de este dispositivo como máquina térmica y compárela con la de una máquina de ciclo de Carnot que opera entre las mismas temperaturas mínima y máxima de $\SI{250}{\kelvin}$ y $\SI{450}{\kelvin}$.
\end{Exercise}
\begin{Answer}
	\begin{minipage}[t]{.4\textwidth}
    $7.5\%$ (el máximo posible es $44.4\%$)
  \end{minipage}
\end{Answer}
%
\begin{Exercise}\label{p:segundoppio04}
  Una máquina térmica opera siguiendo aproximadamente el ciclo de la figura \ref{f:segundoppio04}. Dos moles de helio gaseoso, que puede ser considerado como gas ideal, son utilizados como la sustancia de trabajo en esta máquina, que alcanza una temperatura máxima de $\SI{327}{\celsius}$ cuando llega al equilibrio con la fuente de temperatura alta. El proceso $bc$ es isotérmico reversible manteniendo al gas en contacto con la fuente de temperatura, la presión en los estados $a$ y $c$ es $\SI{100}{\kilo\pascal}$, y en el estado $b$ es $\SI{300}{\kilo\pascal}$. La temperatura de la fuente fría es la misma que alcanza el gas cuando llega al estado \textit{a}. \textit{a}) ¿Cuánto calor entra en el gas y cuánto sale del gas en cada ciclo? \textit{b}) ¿Cuánto trabajo efectúa la máquina en cada ciclo? \textit{c}) ¿Qué eficiencia tiene esta máquina? \textit{d}) Calcule el porcentaje que representa la eficiencia de esta máquina respecto de la máxima eficiencia posible que puede lograrse con los depósitos caliente y frío que se usan en este ciclo ($e_\text{máquina}/e_\text{máximo}\times 100\%$). \textit{e}) Si se invierte el sentido del ciclo para transformarla en una máquina refrigeradora, ¿la máquina que se obtiene es posible? (Verificar que se cumplan ambos principios de la termodinámica.)
\end{Exercise}
\begin{Answer}
	\begin{minipage}[t]{.4\textwidth}
    \textit{a}) Entran $\SI{20.9}{\kilo\joule}$ y salen $\SI{16.6}{\kilo\joule}$\\ \textit{b}) $\SI{4.3}{\kilo\joule}$\\ \textit{c}) 21\%\\ \textit{d}) La eficiencia de esta máquina es 31\% del máximo posible.
  \end{minipage}
\end{Answer}
%
\begin{center}
  \begin{tikzpicture}[scale=0.9]
    \begin{axis}[
      ticks=none,
      axis x line=bottom,
      axis y line=left,
      xmin=0.01, xmax=0.12,
      ymin=50, ymax=350, 
      xlabel={$V$},
      ylabel={$p$},
      ];
    \addplot [color=blue, very thick] [samples= 180, domain=0.033:0.066]  {100};
    \addplot [latex-, color=blue, very thick] [samples= 180, domain=0.066:0.1]  {100};
    \addplot [-latex, color=blue, very thick] [samples= 180, domain=0.033:0.066]  {100*0.1/x};
    \addplot [color=blue, very thick] [samples= 180, domain=0.066:0.1]  {100*0.1/x};
    \draw [color=blue, very thick][-latex](0.033,100)--(0.033,200);
    \draw [color=blue, very thick](0.033,200)--(0.033,300);
    \draw (0.033,100) node [left] {$a$};
    \draw (0.033,300) node [left] {$b$};
    \draw (0.1,100) node [above right] {$c$};
    \fill [black](0.033,100) circle(2pt);
    \fill [black](0.033,300) circle(2pt);
    \fill [black](0.1,100) circle(2pt);
    \end{axis}
  \end{tikzpicture}
  \captionof{figure}{Problema \ref{p:segundoppio04}\label{f:segundoppio04}}
\end{center}
%
\begin{Exercise}\label{p:segundoppio03}
  \ifthenelse{\equal{\seleccionados}{true}}
  {\addToList{xyz-segundoppio}{\ExerciseHeaderNB}}{}
  El diagrama $p$-$V$ de la figura \ref{f:segundoppio03} muestra un ciclo de una máquina térmica que usa $\SI{0.250}{\mole}$ de un gas ideal para el cual el coeficiente de dilatación adiabática es $\gamma = 1.40$. La parte curva $ab$ del ciclo corresponde a un proceso adiabático. \textit{a}) ¿Cuánto calor absorbe este gas por ciclo, y en qué parte del ciclo ocurre? \textit{b}) ¿Cuánto calor cede este gas por ciclo, y en qué parte del ciclo ocurre? \textit{c}) ¿Cuánto trabajo realiza esta máquina en un ciclo? \textit{d}) ¿Cuál es la eficiencia térmica de la máquina?
\end{Exercise}
\begin{Answer}
	\begin{minipage}[t]{.4\textwidth}
    \textit{a}) $\SI{5480}{\joule}$\\ \textit{b}) $\SI{3720}{\joule}$\\ \textit{c}) $\SI{1760}{\joule}$\\ \textit{d}) 32\%
  \end{minipage}
\end{Answer}
%
\begin{center}
  \begin{tikzpicture}[scale=0.9]
    \begin{axis}[
      %ticks=none,
      axis x line=bottom,
      axis y line=left,
      xmin=0, xmax=10,
      ymin=0, ymax=15, 
      xlabel={$V$~[$\si{\cubic\metre}$]},
      ylabel={$p$~[atm]},
      xtick={2,9},
      xticklabels={0.002,0.009},
      ytick={1.5},
      yticklabels={1.5}
      %yticklabels={},
      %extra x ticks={0.04},
      %extra y ticks={2}
      ];
    \addplot [color=blue, very thick] [samples= 180, domain=2:5.5]  {1.5};
    \addplot [latex-, color=blue, very thick] [samples= 180, domain=5.5:9]  {1.5};
    \addplot [-latex, color=blue, very thick] [samples= 180, domain=2:4.5]  {1.5*(9/x)^1.5};
    \addplot [color=blue, very thick] [samples= 180, domain=4.5:9]  {1.5*(9/x)^1.5};
    \addplot[color = black, dashed, thick] coordinates {(2, 0) (2, 1.5) (0, 1.5)};
    \addplot[color = black, dashed, thick] coordinates {(9, 0) (9, 1.5)};
    \draw [color=blue, very thick][-latex](2,1.5)--(2,7.5);
    \draw [color=blue, very thick](2,7.5)--(2,14.3);
    \draw (2,14.3) node [left] {$a$};
    \draw (9,1.5) node [above right] {$b$};
    \draw (2,1.5) node [above left] {$c$};
    \fill [black](2,14.3) circle(2pt);
    \fill [black](9,1.5) circle(2pt);
    \fill [black](2,1.5) circle(2pt);
    \end{axis}
  \end{tikzpicture}
  \captionof{figure}{Problema \ref{p:segundoppio03}\label{f:segundoppio03}}
\end{center}
%
\begin{Exercise}
  Una planta generadora de energía eléctrica de $\SI{1000}{\mega\watt}$, alimentada con carbón, tiene una eficiencia térmica del 40\%. \textit{a}) ¿Cuál es la rapidez de suministro de calor a la planta? \textit{b}) La planta quema carbón de piedra (antracita), que tiene un calor de combustión de $\SI{2.65E7}{\joule/\kilogram}$. ¿Cuánto carbón consume la planta al día, si opera de manera continua? \textit{c}) El depósito frío hacia donde la planta cede calor es un río cercano. La temperatura del río es $\SI{18.0}{\celsius}$ antes de llegar a la planta de energía y $\SI{18.5}{\celsius}$ después de recibir el calor de desecho de la planta. Calcule el caudal del río en $\si{\cubic\metre/\second}$.
\end{Exercise}
\begin{Answer}
	\begin{minipage}[t]{.4\textwidth}
    \textit{a}) $\SI{2500}{\mega\watt}$\\ \textit{b}) $\SI{8150}{ton}$\\ \textit{c}) $\SI{720}{\cubic\metre/\second}$
  \end{minipage}
\end{Answer}
%
\begin{Exercise}\label{p:segundoppio07}
  La figura \ref{f:segundoppio07} muestra el ciclo de Stirling idealizado, donde el proceso $1 \rightarrow 2$ es una expansión isotérmica a alta temperatura ($T_\text{c}$) y el proceso $3 \rightarrow 4$ es una compresión isotérmica a baja temperatura ($T_\text{f}$). \textit{a}) En un motor Stirling, las transferencias de calor en $4 \rightarrow 1$ y $2 \rightarrow 3$ no implican fuentes de calor externas, sino que usan regeneración: la misma sustancia que transfiere calor al gas del interior del cilindro en el proceso $4 \rightarrow 1$ absorbe calor del gas en el proceso $2 \rightarrow 3$. Por lo tanto, los calores transferidos $Q_{41}$ y $Q_{23}$ no afectan a la eficiencia del motor. Explique esta afirmación demostrando que se cumple $Q_{41} = - Q_{23}$. \textit{b}) Deduzca la eficiencia de este ciclo en términos de las temperaturas $T_\text{c}$ y $T_\text{f}$, teniendo en cuenta que representa a un motor Stirling que utiliza regeneración.
\end{Exercise}
\begin{Answer}
	\begin{minipage}[t]{.4\textwidth}
    \textit{b}) $e = 1 - T_f/T_c$
  \end{minipage}
\end{Answer}
%
\begin{center}
  \begin{tikzpicture}[scale=0.9]
    \begin{axis}[
      % ticks=none,
      axis x line=bottom,
      axis y line=left,
      xmin=0, xmax=5,
      ymin=0, ymax=450, 
      xlabel={$V$},
      ylabel={$p$},
      xtick={2,4},
      xticklabels={$V_1$,$V_2$},
      ytick=\empty,
      yticklabels={}
      %extra x ticks={0.04},
      %extra y ticks={2}
      ];
    \addplot [-latex, color=red, very thick] [samples= 180, domain=2:3] {600/x};
    \addplot [color=red, very thick] [samples= 180, domain=3:4] {600/x};
    \addplot [color=red, dashed, thick] [samples= 180, domain=1.4:2] {600/x};
    \addplot [color=red, dashed, thick] [samples= 180, domain=4:5] {600/x};
    \addplot [latex-, color=blue, very thick] [samples= 180, domain=3:4] {300/x};
    \addplot [color=blue, very thick] [samples= 180, domain=2:3] {300/x};
    \addplot [color=blue, dashed, thick] [samples= 180, domain=0.7:2] {300/x};
    \addplot [color=blue, dashed, thick] [samples= 180, domain=4:5] {300/x};
    
    \addplot[-latex, color = black, very thick] coordinates {(2, 150) (2, 225)};
    \addplot[color = black, very thick] coordinates {(2, 225) (2, 300)};
    \addplot[-latex, color = black, very thick] coordinates {(4, 150) (4, 112)};
    \addplot[color = black, very thick] coordinates {(4, 112) (4, 75)};

    \addplot[color = black, dashed, thick] coordinates {(2, 0) (2, 150)};
    \addplot[color = black, dashed, thick] coordinates {(4, 0) (4, 75)};
    
    \draw (2,150) node [below left] {4};
    \draw (2,300) node [below left] {1};
    \draw (4,150) node [above right] {2};
    \draw (4,75) node [below right] {3};
    \fill [black](2,150) circle(2pt);
    \fill [black](2,300) circle(2pt);
    \fill [black](4,150) circle(2pt);
    \fill [black](4,75) circle(2pt);
    \end{axis}
  \end{tikzpicture}
  \captionof{figure}{Problema \ref{p:segundoppio07}\label{f:segundoppio07}}
\end{center}
%
\begin{Exercise}\label{p:segundoppio08}
  El ciclo de la figura \ref{f:segundoppio08} aproxima el funcionamiento de una máquina térmica constituida por $\SI{0.55}{\mole}$ de un gas ideal diatómico que intercambia calor con dos depósitos que mantienen sus temperaturas constantes. El gas comienza en el estado $a$ con un volumen de $\SI{2.3}{\liter}$ y a la temperatura del reservorio caliente, $\SI{520}{\kelvin}$. Se expande adiabáticamente hasta un volumen de $\SI{9.0}{\liter}$, llegando al estado $b$ cuya temperatura es la del reservorio frío, $\SI{300}{\kelvin}$. Luego se comprime isotérmicamente hasta el estado $c$ con un volumen de $\SI{1.5}{\liter}$. Por último evoluciona sobre la trayectoria rectilínea que une los estados $c$ y $a$ para completar el ciclo. Calcular la eficiencia térmica de esta máquina. 
\end{Exercise}
\begin{Answer}
	\begin{minipage}[t]{.4\textwidth}
    $0.254$
  \end{minipage}
\end{Answer}
%
\begin{center}
  \begin{tikzpicture}[scale=0.9]
    \begin{axis}[
      ticks=none,
      axis x line=bottom,
      axis y line=left,
      xmin=0, xmax=10, 
      ymin=0, ymax=12.5, 
      xlabel={$V$},
      ylabel={$p$},
      % xtick={2,9},
      % xticklabels={1.5,9},
      % ytick={1.5},
      % yticklabels={1.5}
      %yticklabels={},
      %extra x ticks={0.04},
      %extra y ticks={2}
      ];
    \addplot [-latex, color=blue, very thick] [samples= 180, domain=2.3:4.5]  {1.5*(9/x)^1.5};
    \addplot [color=blue, very thick] [samples= 180, domain=4.5:9]  {1.5*(9/x)^1.5};
    \addplot [latex-, color=blue, very thick] [samples= 180, domain=3:9]  {1.5*9/x};
    \addplot [color=blue, very thick] [samples= 180, domain=1.5:3]  {1.5*9/x};
    % \addplot[color = black, dashed, thick] coordinates {(2, 0) (2, 1.5) (0, 1.5)};
    % \addplot[color = black, dashed, thick] coordinates {(9, 0) (9, 1.5)};
    \draw [color=blue, very thick][-latex](1.5,9)--(2,10.63);
    \draw [color=blue, very thick](2,10.63)--(2.3,11.61);
    \draw (2.3,11.61) node [right] {$a$};
    \draw (9,1.5) node [above right] {$b$};
    \draw (1.5,9) node [left] {$c$};
    \fill [black](2.3,11.61) circle(2pt);
    \fill [black](9,1.5) circle(2pt);
    \fill [black](1.5,9) circle(2pt);
    \end{axis}
  \end{tikzpicture}
  \captionof{figure}{Problema \ref{p:segundoppio08}\label{f:segundoppio08}}
\end{center}
%

%   \twocolumn[\colorsection{Preguntas para el análisis}]
\textit{En esta sección se requiere brindar respuestas argumentadas.}
\setcounter{figure}{0}
%
\begin{Exercise}
  Explique por qué no tendría sentido utilizar un termómetro de vidrio de tamaño normal, para medir la temperatura del agua caliente contenida en un dedal.
\end{Exercise}
%
\begin{Exercise}
  Si usted calienta el aire dentro de un recipiente rígido y sellado hasta que su temperatura en la escala Kelvin se duplique, la presión del aire en el recipiente también se duplica. ¿Esto es cierto si se duplica la temperatura Celsius del aire en el recipiente?
\end{Exercise}
%
\begin{Exercise}
  \textit{a}) Suponga que en el problema \ref{p:calorimetria01} no es posible despreciar la transferencia de calor del líquido al recipiente en ese experimento. ¿El resultado obtenido en dicho experimento, es mayor o menor que el calor específico promedio real del líquido? \textit{b}) Ahora suponga que se puede despreciar el intercambio de calor con el recipiente, pero no es posible despreciar el intercambio de calor con el entorno. ¿El resultado obtenido en dicho experimento, es mayor o menor que el calor específico promedio real del líquido?
\end{Exercise}
%
\begin{Exercise}
  Si en el problema \ref{p:transmision00} se tuviera en cuenta la convección del aire a ambos lados de la pared, ¿la corriente de calor sería mayor, menor o la misma que la calculada en dicho problema?
\end{Exercise}
%
\begin{Exercise}
  Se desea cubrir las paredes metálicas de un horno con un material aislante. ¿Cuáles de los siguientes valores cambian y cuáles se mantienen igual si en lugar de aplicar el revestimiento en el interior del horno se lo aplica sobre el exterior de las paredes?: \textit{i}) la potencia transmitida a través de las paredes; \textit{ii}) la temperatura de la superficie interior; \textit{iii}) la temperatura de la superficie exterior. ¿Sus respuestas son las mismas si se incluye o no convección en los cálculos?
\end{Exercise}
%
\begin{Exercise}\label{p:preguntastermo01}
  \ifthenelse{\equal{\seleccionados}{true}}
  {\addToList{xyz-preguntas}{\ExerciseHeaderNB}}{}
  En la figura \ref{f:preguntastermo01} se muestra la evolución de la temperatura en función del calor intercambiado con el entorno para una muestra de un líquido desconocido. Si $\text{C}_{\text{l}}$ es el calor específico de este líquido y $\text{C}_{\text{s}}$ es el calor específico de la misma sustancia en estado sólido, ¿cuál de las siguientes opciones es la única correcta?\\
\renewcommand{\arraystretch}{1.5}
  \begin{tabular}{p{2.5cm} p{2.5cm} p{2.5cm}}
     \textit{a}) $\text{C}_{\text{s}}=\text{C}_{\text{l}}$ & \textit{b}) $\text{C}_{\text{s}}=3\text{C}_{\text{l}}$ & \textit{c}) $\text{C}_{\text{s}}=\text{C}_{\text{l}}/3$ \\
     \textit{d}) $\text{C}_{\text{s}}=9\text{C}_{\text{l}}$ & e) $\text{C}_{\text{s}}=\text{C}_{\text{l}}/9$ \\
  \end{tabular} \\
\end{Exercise}
%
\begin{center}
    \begin{tikzpicture}[scale=0.8]
        \begin{axis}[
            ticks= none,
            every major x tick/.append style={thick,blue},
            clip=false,
            grid=both,
            minor x tick num=2,        
            minor y tick num=2,
            xmin=0, xmax=7,   
            ymin=0, ymax=6.5,
            %xtick  align=center,
            xlabel={$|Q|$},
            ylabel={$T$}
        ];
        \addplot [color=red, thick] [id=p36a,samples= 180, domain=0:1]  {5-3*x};
        \addplot [color=red, thick] [id=p36b,samples= 180, domain=1:3]  {2};
        \addplot [color=red, thick] [id=p36c,samples= 180, domain=3:7]  {2-1/3*(x-3)};
        \end{axis}
    \end{tikzpicture}
    \captionof{figure}{Pregunta \ref{p:preguntastermo01}\label{f:preguntastermo01}}
\end{center}
%
\begin{Exercise}
  ¿Es correcto afirmar que si se disminuye el área de un cuerpo, el calor intercambiado disminuye en la misma proporción independientemente del mecanismo considerado?
\end{Exercise}
%
\begin{Exercise}
  \textit{a}) Un bloque de metal frío se siente más frío que uno de madera a la misma temperatura. ¿Por qué? \textit{b}) Un bloque de metal caliente se siente más caliente que uno de madera a la misma temperatura. ¿Por qué? \textit{c}) ¿Hay alguna temperatura a la que ambos bloques se sientan igualmente calientes o fríos? ¿Cuál es esta?
\end{Exercise}
%
\begin{Exercise}
  En algunas situaciones suele considerarse que uno de los mecanismos de transmisión del calor es ``más importante'' que los otros. \textit{a}) Mencione un caso en que la conducción es el mecanismo primordial y de algunas razones para que esta aproximación pueda considerarse correcta. \textit{b}) Ídem \textit{a}) para convección. \textit{c}) Ídem \textit{a}) para radiación.
\end{Exercise}
%
\begin{Exercise}\label{p:preguntastermo02}
  \ifthenelse{\equal{\seleccionados}{true}}
  {\addToList{xyz-preguntas}{\ExerciseHeaderNB}}{}
  El gráfico de la figura \ref{f:preguntastermo02} representa la temperatura en función a la distancia a la fuente caliente dentro de una pared formada por dos materiales, el material $A$ con un espesor $l$ y el material $B$ con espesor $2l$. Para el gráfico en cuestión analice las siguientes afirmaciones y diga si son verdaderas o falsas y por qué: \textit{a}) La conductividad térmica de $A$ es mayor que la de $B$. \textit{b}) La temperatura de unión en la superficie de contacto antre $A$ y $B$ es menor que el promedio de las temperaturas en los lados de la pared. \textit{c}) El flujo calórico es mayor en el material de mayor conductividad.
\end{Exercise}
%
\begin{center}
  \begin{tikzpicture}[scale=0.9]
    \begin{axis}[
      %ticks=none,
      axis x line=bottom,
      axis y line=left,
      xmin=0, xmax=3.5,
      ymin=0, ymax=5,
      xlabel={Posición},
      ylabel={Temperatura},
      xtick={1,3},
      xticklabels={$l$,$3l$},
      ytick={},
      yticklabels={}
      ];
    \draw [color=blue, very thick](0,0.5)--(0,4.5);
    \draw [color=blue, very thick](1,0.5)--(1,4.5);
    \draw [color=blue, very thick](3,0.5)--(3,4.5);

    % No puedo usar foreach adentro de axis, y si lo uso afuera no
    % respeta la escala.
    \draw [color=blue!30](0,0.75)--(0.5,0.5);
    \draw [color=blue!30](0,1)--(1,0.5);
    \draw [color=blue!30](0,1.25)--(1,0.75);
    \draw [color=blue!30](0,1.5)--(1,1);
    \draw [color=blue!30](0,1.75)--(1,1.25);
    \draw [color=blue!30](0,2)--(1,1.5);
    \draw [color=blue!30](0,2.25)--(1,1.75);
    \draw [color=blue!30](0,2.5)--(1,2);
    \draw [color=blue!30](0,2.75)--(1,2.25);
    \draw [color=blue!30](0,3)--(1,2.5);
    \draw [color=blue!30](0,3.25)--(1,2.75);
    \draw [color=blue!30](0,3.5)--(1,3);
    \draw [color=blue!30](0,3.75)--(1,3.25);
    \draw [color=blue!30](0,4)--(1,3.5);
    \draw [color=blue!30](0,4.25)--(1,3.75);
    \draw [color=blue!30](0,4.5)--(1,4);
    \draw [color=blue!30](0.5,4.5)--(1,4.25);

    \draw [color=blue!30](2.875,4.35)--(3,4.25);
    \draw [color=blue!30](2.5625,4.35)--(3,4);
    \draw [color=blue!30](2.25,4.35)--(3,3.75);
    \draw [color=blue!30](1.9375,4.35)--(3,3.5);
    \draw [color=blue!30](1.625,4.35)--(3,3.25);
    \draw [color=blue!30](1.3125,4.35)--(3,3);
    \draw [color=blue!30](1,4.35)--(3,2.75);
    \draw [color=blue!30](1,4.10)--(3,2.5);
    \draw [color=blue!30](1,3.85)--(3,2.25);
    \draw [color=blue!30](1,3.60)--(3,2.0);
    \draw [color=blue!30](1,3.35)--(3,1.75);
    \draw [color=blue!30](1,3.10)--(3,1.5);
    \draw [color=blue!30](1,2.85)--(3,1.25);
    \draw [color=blue!30](1,2.6)--(3,1.);
    \draw [color=blue!30](1,2.35)--(3,0.75);
    \draw [color=blue!30](1,2.10)--(3,0.5);
    \draw [color=blue!30](1,1.85)--(2.6875,0.5);
    \draw [color=blue!30](1,1.6)--(2.375,0.5);
    \draw [color=blue!30](1,1.35)--(2.0625,0.5);
    \draw [color=blue!30](1,1.1)--(1.75,0.5);
    \draw [color=blue!30](1,0.85)--(1.4375,0.5);

    \draw [color=black, very thick](0,4)--(1,3.5);
    \draw [color=black, very thick](1,3.5)--(3,0.75);

    \draw (0.5,3) node [font=\fontsize{17}{0}] {$A$};
    \draw (2.5,3) node [font=\fontsize{17}{0}] {$B$};
    \end{axis}
    % \foreach \y in {1,...,5}
    %     \draw [color=blue!30](0,\y)--(1.8,\y-0.5);
  \end{tikzpicture}
  \captionof{figure}{Problema \ref{p:preguntastermo02}\label{f:preguntastermo02}}
\end{center}
%
\textbf{Se agregarán más preguntas próximamente.}
% \begin{Exercise}\label{p:preguntastermo03}
%   En la figura \ref{f:preguntastermo03} se muestran dos evoluciones de un mismo gas ideal. ¿Cómo se comparan el calor $Q_{a\rightarrow b \rightarrow c}$ a lo largo de la línea sólida con el calor $Q_{a\rightarrow c}$ a lo largo de la línea punteada?
% \end{Exercise}
% %
% \begin{center}
%   \begin{tikzpicture}[scale=0.9]
%     \begin{axis}[
%       ticks=none,
%       axis x line=bottom,
%       axis y line=left,
%       clip=false,
%       xmin=0, xmax=1.1,
%       ymin=0, ymax=4.5,
%       xtick  align=center,
%       xlabel={$V$},
%       ylabel={$p$}
%       ];
%       \draw [color=blue, very thick][-latex](0.2,1)--(0.4,2.5);
%       \draw [color=blue, very thick](0.4,2.5)--(0.6,4);
%       \draw [color=blue, very thick][-latex](0.6,4)--(0.8,2.5);
%       \draw [color=blue, very thick](0.8,2.5)--(1,1);
%       \draw [color=blue, dashed, very thick][-latex](0.2,1)--(0.6,1);
%       \draw [color=blue, dashed, very thick](0.6,1)--(1,1);
%       \draw (0.2,1) node [below left] {$a$};
%       \draw (0.6,4) node [above left] {$b$};
%       \draw (1,1) node [below right] {$c$};
%       \fill [black](0.2,1) circle(2pt);
%       \fill [black](0.6,4) circle(2pt);
%       \fill [black](1,1) circle(2pt);
%     \end{axis}
%   \end{tikzpicture}
%   \captionof{figure}{Problema \ref{p:preguntastermo03}\label{f:preguntastermo03}}
% \end{center}
% %
% \begin{Exercise}
%   \ifthenelse{\equal{\seleccionados}{true}}
%   {\addToList{xyz-preguntas}{\ExerciseHeaderNB}}{}
%   En un calorímetro ideal se mezclan una masa de agua a $\SI{70}{\celsius}$ y otra masa de agua a $\SI{20}{\celsius}$, y se espera hasta que el sistema (la mezcla) alcance el equilibrio térmico. Siendo que el sistema está aislado del entorno, ¿varía su entropía?
% \end{Exercise}
% %
% \begin{Exercise}
%   En los siguientes procesos, ¿el trabajo efectuado por el sistema (definido como un gas que se expande o se contrae) sobre el ambiente es positivo o negativo? \textit{a}) La expansión de una mezcla aire-gasolina quemada en el cilindro de un motor de automóvil; \textit{b}) abrir una botella de champaña; \textit{c}) llenar un tanque de buceo con aire comprimido; \textit{d}) la abolladura parcial de una botella de agua vacía y cerrada, al conducir descendiendo desde las montañas hacia el nivel del mar.
% \end{Exercise}
% %
% \begin{Exercise}
%   ¿En qué situación debe usted efectuar más trabajo: al inflar un globo al nivel del mar o al inflar el mismo globo con el mismo volumen en la cima del Aconcagua?
% \end{Exercise}
% %
% \begin{Exercise}
%   \ifthenelse{\equal{\seleccionados}{true}}
%   {\addToList{xyz-preguntas}{\ExerciseHeaderNB}}{}
%   Cuando se derrite hielo a $\SI{0}{\celsius}$ su volumen disminuye. ¿El cambio de energía interna es mayor, menor o igual que el calor agregado?
% \end{Exercise}
% %
% \begin{Exercise}
%   Un gas ideal se expande mientras que la presión se mantiene constante. Durante este proceso, ¿hay flujo de calor hacia el gas o hacia afuera de este?
% \end{Exercise}
% %
% \begin{Exercise}
%   En un proceso a volumen constante, $dU = nC_VdT$. En cambio, en un proceso a presión constante, no se cumple que $dU = nC_pdT$. ¿Por qué no?
% \end{Exercise}
% %
% \begin{Exercise}
%   Convertir energía mecánica totalmente en calor, ¿viola la segunda ley de la termodinámica? ¿Y convertir calor totalmente en trabajo?
% \end{Exercise}
% %
% \begin{Exercise}\label{p:preguntastermo04}
%   Un sistema termodinámico experimenta un proceso cíclico como se muestra en la figura \ref{f:preguntastermo04}. El ciclo consiste en dos lazos cerrados: el lazo \textit{I} y el \textit{II}. \textit{a}) Durante un ciclo completo, ¿el sistema efectúa trabajo neto positivo o negativo? \textit{b}) Durante un ciclo completo, ¿entra calor al sistema o sale de él? \textit{c}) En cada lazo, \textit{I} y \textit{II}, ¿entra calor en el sistema o sale de él?
% \end{Exercise}
% %
% % It needs \usetikzlibrary{decorations.markings}
% % and the tikzet defined in the preamble.
% \begin{center}
%   \begin{tikzpicture}[scale=0.9]
%     \draw [cyan] plot [smooth cycle, tension=1] coordinates { (0,0.5) (2,2.5) (3,-2) (4.5,-2.4)} [arrow inside={end=stealth,opt={red,scale=2}}{0,0.35,0.5,0.75}];
%     \draw [-latex] (-1,-3.5) -- (-1,3.5) node [left] {$p$};
%     \draw [-latex] (-1,-3.5) -- (5.5,-3.5) node [below] {$V$};
%     \draw (1.35,1) node [] {$I$};
%     \draw (4,-2.5) node [] {$II$};
%   \end{tikzpicture}
%   \captionof{figure}{Problema \ref{p:preguntastermo04}\label{f:preguntastermo04}}
% \end{center}
% %
% \begin{Exercise}
%   \ifthenelse{\equal{\seleccionados}{true}}
%   {\addToList{xyz-preguntas}{\ExerciseHeaderNB}}{}
%   Compare el diagrama $p$-$V$ para el ciclo Otto con el diagrama para la máquina térmica de Carnot. Explique algunas diferencias importantes entre los dos ciclos.
% \end{Exercise}
% %
% \begin{Exercise}
%   ¿Puede un sistema experimentar variaciones de entropía negativas?
% \end{Exercise}
% %
% \begin{Exercise}
%   ¿Un refrigerador lleno de alimentos con una temperatura ambiente de $\SI{20}{\celsius}$ consume más potencia si la temperatura es de $\SI{15}{\celsius}$? ¿O el consumo es el mismo?
% \end{Exercise}
% %
% \begin{Exercise}
%   Explique si en cada uno de los siguientes procesos hay aumentos de entropía o no: la mezcla de agua caliente y fría; expansión libre de un gas; flujo irreversible de calor; producción de calor por fricción mecánica.
% \end{Exercise}
% %
% \begin{Exercise}
%   La expansión libre de un gas ideal es un proceso adiabático, por lo que no hay transferencia de calor. Tampoco se realiza trabajo, de manera que la energía interna no cambia. Por lo tanto, $Q/T = 0$; sin embargo, la entropía es mayor después de la expansión. ¿Por qué la ecuación $\Delta S = \int dQ/T$ no se aplica a esta situación?
% \end{Exercise}
% %


  \twocolumn[\colorsectionnonumber{Respuestas de la unidad I}]
      \shipoutAnswer

\end{document}
